\documentclass{article}
\usepackage[margin=1in]{geometry}

% ==== PACKAGES ====
\usepackage{amsmath, amssymb, amsthm}
\usepackage{mathtools}
\usepackage[longnamesfirst,sort&compress]{natbib}
\usepackage{graphicx}
\usepackage{algorithm, algorithmic}
\usepackage{hyperref}
\usepackage{cleveref}
\usepackage{booktabs}
\usepackage{enumitem}

% Citation punctuation style matching original
\bibpunct{(}{)}{;}{a}{,}{,}

% ==== CUSTOM COMMANDS ====
\newcommand{\E}{\mathbb{E}}

\renewcommand{\P}{\mathbb{P}}
\newcommand{\R}{\mathbb{R}}
\newcommand{\diff}{\mathrm{d}}
\newcommand{\pard}[2]{\frac{\partial #1}{\partial #2}}

% Theorem environments
\theoremstyle{definition}
\newtheorem{definition}{Definition}[section]
\newtheorem{theorem}{Theorem}[section]
\newtheorem{proposition}{Proposition}[section]
\newtheorem{lemma}{Lemma}[section]
\newtheorem{remark}{Remark}[section]

% Proof environment for appendix
\newenvironment{delayedproof}[1]
 {\begin{proof}[\raisedtarget{#1}Proof of \Cref{#1}]}
 {\end{proof}}
\newcommand{\raisedtarget}[1]{%
  \raisebox{\fontcharht\font`P}[0pt][0pt]{\hypertarget{#1}{}}%
}
\newcommand{\proofref}[1]{\hyperlink{#1}{proof}}

\title{First-Hitting Time Analysis for Log-Health Process with Constant Jump Intensities}
\author{[Author Names]}
\date{\today}

\begin{document}
\maketitle

\begin{abstract}
We develop an optimal allocation framework for long-short cryptocurrency positions on decentralized finance (DeFi) lending platforms by analyzing first-hitting time distributions for log-health processes under constant-intensity jump-diffusion dynamics. Moving beyond the complexity of Hawkes processes, we employ a spectrally negative Lévy process with constant-intensity compound Poisson jumps to model wrong-way risk in cryptocurrency markets. The approach provides semi-analytical solutions via Laplace transform methods and Gaver-Stehfest inversion, enabling practical computation of margin call probabilities. Our framework balances expected returns against liquidation risk through optimization of position weights subject to collateral constraints.

\textbf{Keywords:} DeFi, long-short strategies, liquidation risk, Lévy processes, first passage times, cryptocurrency
\end{abstract}

% ====================================================================
\section{Introduction}
% ====================================================================

Decentralized finance (DeFi) lending platforms such as AAVE and Compound have emerged as powerful tools for implementing sophisticated long-short cryptocurrency strategies. With combined total value locked exceeding \$10 billion, these protocols enable traders to simultaneously hold long positions in promising assets while borrowing and shorting others, all while earning interest on collateralized deposits. This creates unique opportunities for alpha generation that are unavailable through traditional centralized exchanges.

The core challenge in DeFi long-short positioning lies in managing liquidation risk under extreme market volatility. Unlike traditional margin trading, DeFi platforms require significant overcollateralization (typically 120-150\%) and impose automatic liquidation when health factors fall below unity. Given the notorious volatility of cryptocurrency markets, where daily moves of 10-20\% are common and flash crashes can exceed 50\%, the probability and timing of liquidation events become critical factors in position sizing and risk management.

Wrong-way risk represents the most significant threat to DeFi portfolios: adverse price movements in either the collateral asset (downward) or borrowed asset (upward) directly deteriorate the health factor, with large jumps potentially triggering immediate liquidation. Traditional portfolio theory, which assumes smooth price evolution and continuous rebalancing opportunities, fails to capture the discrete, path-dependent nature of liquidation risk in DeFi environments.

The academic literature on first-hitting times for jump-diffusion processes suggests that Hawkes processes, with their ability to model clustered jumps and cross-excitation between assets, would be ideal for capturing wrong-way risk dynamics. However, our initial investigations revealed that Hawkes-based models, while theoretically appealing, present severe computational challenges. The resulting characteristic function analysis requires solving high-dimensional Riccati systems that are numerically unstable and computationally intensive, making them impractical for real-time risk management applications.

This paper advocates for a more pragmatic approach: modeling the log-health process as a spectrally negative Lévy process with constant jump intensities. While this approach cannot capture the dynamic clustering effects of Hawkes processes, it offers substantial practical advantages:

\begin{enumerate}
    \item \textbf{Analytical tractability}: Explicit Laplace transform representations for first-hitting time distributions
    \item \textbf{Numerical stability}: Robust inversion algorithms that converge reliably across parameter ranges
    \item \textbf{Computational efficiency}: Fast evaluation suitable for real-time portfolio optimization
    \item \textbf{Parameter parsimony}: Fewer parameters reduce calibration complexity and overfitting risk
    \item \textbf{Practical implementation}: Semi-analytical solutions enable deployment in production trading systems
\end{enumerate}

Our framework demonstrates that effective liquidation risk management in DeFi applications need not require the most sophisticated stochastic models. The constant-intensity jump-diffusion approach captures the essential wrong-way risk characteristics while maintaining the computational tractability necessary for practical portfolio management. This represents a conscious trade-off between theoretical completeness and practical utility, prioritizing robust, implementable solutions over academic elegance.

% ====================================================================
\section{Portfolio Setup and Log-Health Process}
% ====================================================================

\subsection{Long-Short Position Construction}

Consider a long-short position on a DeFi lending platform where a user deposits collateral asset $X$ and borrows asset $Y$. The position consists of:
\begin{itemize}
    \item \textbf{Long position}: $w_X$ units of asset $X(t)$ deposited as collateral
    \item \textbf{Short position}: $w_Y$ units of asset $Y(t)$ borrowed for short selling
\end{itemize}

We implement a \textbf{net exposure constraint}:
\begin{equation}
w_X - w_Y = 1, \quad w_X > 0, \quad w_Y > 0 \label{eq:net_exposure}
\end{equation}

This ensures unit net exposure while allowing flexible allocation between long and short positions based on relative expected returns and risk characteristics.

\subsection{Health Factor and Liquidation Mechanics}

The \textbf{log-health process} serves as the primary risk metric:
\begin{equation}
h(t) = \log\left( \frac{b_X w_X X(t)}{w_Y Y(t)} \right) \label{eq:log_health}
\end{equation}

where $b_X \in (0,1]$ is the collateral factor (haircut) applied to the long position. A margin call is triggered when $h(t)$ first hits zero, corresponding to the health factor $H(t) = e^{h(t)} \leq 1$.

The liquidation time is formulated as the first hitting time:
\begin{equation}
\tau_0 = \inf\{t \geq 0 : h(t) \leq 0\} \label{eq:liquidation_time}
\end{equation}

% ====================================================================
\section{Log-Health Process and Laplace Exponent Analysis}
% ====================================================================

\subsection{Asset Price Dynamics}

The asset prices follow jump-diffusion dynamics with constant-intensity compound Poisson jumps:
\begin{align}
\frac{dX(t)}{X(t^-)} &= \mu_X dt + \sigma_X dB_X(t) + dJ_X(t) \label{eq:X_dynamics}\\
\frac{dY(t)}{Y(t^-)} &= \mu_Y dt + \sigma_Y dB_Y(t) + dJ_Y(t) \label{eq:Y_dynamics}
\end{align}

where:
\begin{itemize}
    \item $B_X(t), B_Y(t)$ are Brownian motions with correlation $\rho$
    \item $J_X(t)$ has constant intensity $\hat\lambda_X$ and i.i.d. jumps $U_X = -(\delta_X + E_X)$, $E_X \sim \mathrm{Exp}(\eta_X)$ representing negative jumps in the long asset
    \item $J_Y(t)$ has constant intensity $\hat\lambda_Y$ and i.i.d. jumps $U_Y = +(\delta_Y + E_Y)$, $E_Y \sim \mathrm{Exp}(\eta_Y)$ representing positive jumps in the short asset
\end{itemize}

This jump structure captures wrong-way risk: downward jumps in collateral and upward jumps in borrowed assets are both adverse to the portfolio.

\subsection{Wealth Process Dynamics}

The wealth process dynamics are:
\begin{equation}
\frac{dV(t)}{V(t^{-})} = w_X \frac{dX(t)}{X(t^-)}  - w_Y \frac{dY(t)}{Y(t^-)} 
\end{equation}
Using equations 4 and 5 and grouping the terms based on their types:
\begin{equation}
\frac{dV(t)}{V(t^{-})}= (w_X \mu_X - w_Y \mu_y) dt + w_X \sigma_X dB_X(t) - w_Y \sigma_Y dB_Y(t) + w_X dJ_X(t) - w_Y dJ_Y(t)
\end{equation}
The solution to the SDE will then be:
\begin{equation}
V(T) = V(0) e^{(\psi T + \sigma_W W(T))} \prod_{i=1}^{N_X (T)} (1 + w_X U_X^i) \prod_{j=1}^{N_Y (T)} (1 - w_Y U_Y^j)
\end{equation}
Where the drift adjustment term $\psi$ is equal to:
\begin{equation}
\psi = w_X \mu_X - w_Y \mu_Y - \frac{1}{2} ( {w^2_X} {\sigma_X}^2 + {w^2_Y} {\sigma_Y}^2 - 2w_X w_Y \rho \sigma_X \sigma_Y ) 
+ \lambda_X w_X \mathbb{E} [U_X] - \lambda_Y w_Y \mathbb{E} [U_Y]
\end{equation}
The first moment of the wealth process is:
\begin{equation}
\mathbb{E}[V(T)] = V(0) e^{\Theta T}
\end{equation}
With the growth rate $\Theta$:
\begin{equation}
\Theta = w_X \mu_X - w_Y \mu_Y + \lambda_X w_X \mathbb{E} [U_X] - \lambda_Y w_Y \mathbb{E} [U_Y]
\end{equation}
Now, the second moment of the wealth process is:
\begin{equation}
\text{Var}(V(T)) = \mathbb{E}[V(T)^2] - (\mathbb{E}[V(T)])^2 = V(0)^2 e^{2 \Theta T} (e^{\xi T} - 1)
\end{equation}
Where the excess variance rate $\xi$ represents:
\begin{equation}
\xi = w^2_X \sigma^2_X  + w^2_Y \sigma^2_Y - 2w_Xw_Y \rho \sigma_X \rho \sigma_Y +
\lambda_X w^2_X \mathbb{E}[U^2_X] + \lambda_Y w^2_Y \mathbb{E}[U^2_Y]
\end{equation}
The second moment of the jumps is given by:
\begin{equation}
\mathbb{E}[U^2_X] = \frac{1}{\eta^2_X} + \left( \delta_X + \frac{1}{\eta_X} \right)
\end{equation}
\begin{equation}
\mathbb{E}[U^2_Y] = \frac{1}{\eta^2_Y} + \left( \delta_Y + \frac{1}{\eta_Y} \right)
\end{equation}

\subsection{Log-Health Process Dynamics}

\begin{proposition}[Log-Health Process with Inherited Jumps]
\label{prop:log_health_dynamics}
Given the asset price processes $X(t)$ and $Y(t)$ follow the dynamics in equations \eqref{eq:X_dynamics} and \eqref{eq:Y_dynamics}, and position weights satisfying the net exposure constraint $w_X - w_Y = 1$, the log-health process
\begin{equation}
h(t) = \log\left( \frac{b_X w_X X(t)}{w_Y Y(t)} \right)
\end{equation}
evolves as a spectrally negative Lévy process:
\begin{equation}
h(t) = h_0 + \mu_h t + \sigma_h B_h(t) + J_h^X(t) + J_h^Y(t) \label{eq:log_health_full}
\end{equation}
where $J_h^X(t)$ and $J_h^Y(t)$ are compound Poisson processes inherited from the asset jump processes, with:
\begin{align}
J_h^X(t) &= \sum_{i=1}^{N_X(t)} U_X^i, \quad U_X^i = -(\delta_X + E_X^i) \label{eq:jump_X_health}\\
J_h^Y(t) &= \sum_{j=1}^{N_Y(t)} U_Y^j, \quad U_Y^j = +(\delta_Y + E_Y^j) \label{eq:jump_Y_health}
\end{align}
and Brownian motion and drift parameters:
\begin{align}
\mu_h &= \mu_X - \mu_Y - \frac{1}{2}(\sigma_X^2 + \sigma_Y^2 - 2\rho\sigma_X\sigma_Y) \label{eq:drift_h}\\
\sigma_h^2 &= \sigma_X^2 + \sigma_Y^2 - 2\rho\sigma_X\sigma_Y \label{eq:vol_h}\\
dB_h(t) &= \frac{\sigma_X dB_X(t) - \sigma_Y dB_Y(t)}{\sigma_h} \label{eq:brownian_h}
\end{align}
\end{proposition}

\begin{proof}
See \proofref{prop:log_health_dynamics} in Appendix \ref{sec:proofs}.
\end{proof}

\subsection{Laplace Exponent}

\begin{proposition}[Laplace Exponent of Log-Health Process]
\label{prop:laplace_exponent}
For the log-health process $h(t) = h_0 + \mu_h t + \sigma_h B_h(t) + J_h^X(t) + J_h^Y(t)$ from Proposition \ref{prop:log_health_dynamics}, the Laplace exponent is:
\begin{equation}
\psi(\theta) = \frac{1}{2}\sigma_h^2 \theta^2 - \mu_h \theta + \hat\lambda_X \left( e^{-\theta \delta_X} \frac{\eta_X}{\eta_X+\theta} - 1 \right) + \hat\lambda_Y \left( e^{-\theta \delta_Y} \frac{\eta_Y}{\eta_Y+\theta} - 1 \right) \label{eq:laplace_exponent_sec3}
\end{equation}
where the jump terms arise from the shifted exponential distributions:
\begin{align}
U_X^i &= -(\delta_X + E_X^i), \quad E_X^i \sim \mathrm{Exp}(\eta_X)\\
U_Y^j &= +(\delta_Y + E_Y^j), \quad E_Y^j \sim \mathrm{Exp}(\eta_Y)
\end{align}

Furthermore, $\psi(\theta)$ satisfies:
\begin{enumerate}
    \item $\psi(0) = 0$ and $\psi(\theta)$ is convex in $\theta$
    \item $\psi'(0) = -\mu_h - \hat\lambda_X(\delta_X + 1/\eta_X) - \hat\lambda_Y(\delta_Y + 1/\eta_Y)$
    \item The sign of $\psi'(0)$ determines hitting behavior:
    \begin{itemize}
        \item If $\psi'(0) \geq 0$: almost sure hitting ($R = 0$)
        \item If $\psi'(0) < 0$: defective hitting with eventual probability $e^{-Rh_0}$ where $R > 0$ solves $\psi(R) = 0$
    \end{itemize}
\end{enumerate}
\end{proposition}

\begin{proof}
See \proofref{prop:laplace_exponent} in Appendix \ref{sec:proofs}.
\end{proof}

\subsection{Root Function}

The root function $\Phi(q)$ serves as the cornerstone of first-hitting time analysis for spectrally negative Lévy processes. For $q \geq 0$, it is defined as the largest root of $\psi(\theta) = q$:
\begin{equation}
\Phi(q) = \sup\{\theta \geq 0 : \psi(\theta) = q\} \label{eq:root_function_sec3}
\end{equation}

The function satisfies several key properties: $\Phi(0) = R$ corresponds to the adjustment coefficient that determines whether hitting behavior is defective or almost sure, $\Phi(q)$ is strictly increasing and smooth for $q \geq 0$, and $\Phi(q) \to \infty$ as $q \to \infty$ reflecting finite hitting times.

The root function enables the computation of liquidation probabilities $\mathbb{P}(\tau_0 \leq T)$ through Laplace inversion of $e^{-h_0 \Phi(q)}/q$ and facilitates risk-adjusted position sizing through optimal $h_0$ selection that balances expected returns against liquidation costs driven by $\Phi(q)$.

% ====================================================================
\section{First-Hitting Time Analysis}
% ====================================================================

\subsection{First-Hitting Time Problem}

The liquidation event occurs when the log-health process $h(t)$ first reaches zero, corresponding to a health factor $H(t) = e^{h(t)} \leq 1$. This transforms the liquidation risk problem into analyzing the first-hitting time:
\begin{equation}
\tau_0 = \inf\{t \geq 0 : h(t) \leq 0\}
\end{equation}

The distribution of $\tau_0$ determines the timing of liquidation events, which is fundamental for position sizing, risk management, and portfolio optimization in DeFi environments. Unlike traditional finance where positions can be continuously monitored and adjusted, DeFi liquidations occur automatically through smart contracts, making the precise timing distribution critical for risk assessment.

\subsection{Almost Sure vs. Defective Hitting}

The nature of first-hitting behavior fundamentally depends on whether liquidation is inevitable or merely possible, determined by the long-run behavior of the log-health process. When the log-health process has a negative drift that dominates stabilizing effects, liquidation becomes almost sure with $\mathbb{P}(\tau_0 < \infty) = 1$ for all initial health levels $h_0 > 0$. In this regime, the question is not whether liquidation will occur, but when.

Conversely, when favorable drift components can dominate adverse jump effects, hitting becomes defective with $\mathbb{P}(\tau_0 < \infty) = e^{-R h_0} < 1$ for some adjustment coefficient $R > 0$. This regime allows for genuine long-term positions where liquidation is not inevitable. Here, risk management requires conditioning on eventual liquidation:
\begin{equation}
F_{\mathrm{cond}}(T,h_0) = \mathbb{P}(\tau_0 \leq T \mid \tau_0 < \infty) = e^{R h_0} F(T,h_0)
\end{equation}

The distinction between these regimes is crucial for DeFi strategy selection. Almost sure hitting regimes are unsuitable for long-term positions but may support high-frequency strategies. Defective hitting regimes enable traditional investment approaches where positions can potentially be held indefinitely.

\subsection{Laplace Transform of First-Hitting Time}

The mathematical analysis relies on Laplace transform methods. The fundamental result links the root function to liquidation timing:

\begin{theorem}[First-Hitting Time Laplace Transform]
\label{thm:fht_laplace_section3}
For the first hitting time $\tau_0 = \inf\{t \geq 0 : h(t) \leq 0\}$ with $h_0 > 0$:
\begin{equation}
\mathbb{E}_{h_0}[e^{-q \tau_0} \mathbf{1}_{\{\tau_0 < \infty\}}] = e^{-h_0 \Phi(q)}
\end{equation}
\end{theorem}
This enables the computation of liquidation probabilities through the following: 
\begin{equation}
\mathbb{P}(\tau_0 \leq T) = \mathcal{L}^{-1}\left[\frac {e^{-h_0 \Phi(q)}}{q}\right](T)    
\end{equation} where $\mathcal{L}^{-1}$ denotes numerical Laplace inversion. For defective hitting cases, the conditional distribution uses:
\begin{equation}
\mathcal{L}\{F_{\mathrm{cond}}\}(q) = \frac{e^{-h_0 (\Phi(q+R) - R)}}{q}
\end{equation}

The computational procedures for numerical Laplace inversion are detailed in Appendix \ref{sec:numerical_methods}.


% ====================================================================
\section{Portfolio Sizing}
% ====================================================================

\subsection{Optimization Framework}

We formulate position sizing as maximizing an objective balancing risk and return:
\begin{equation}
\max_{r} \quad \mathcal{U}(r) = \mathbb{E}[\text{return}] - \rho_1 \mathbb{V}[\text{return}] - \rho_2 \mathbb{P}(\tau_0 \leq T^*) \label{eq:objective_portfolio}
\end{equation}

where $r = w_X/(w_X + w_Y)$ is the allocation ratio, $T^*$ is the investment horizon, and $\rho_1, \rho_2 > 0$ are risk aversion parameters.

The initial log-health is:
\begin{equation}
h_0 = \log b_X + \log r + \log(X_0/Y_0) \label{eq:initial_health_portfolio}
\end{equation}

\subsection{Constraints}

The optimization is subject to:
\begin{align}
h_0 &\geq h_{\min} > 0 \quad \text{(minimum health factor)} \\
0 < r &< 1 \quad \text{(position limits)}
\end{align}

% ====================================================================
\section{Optimization Solution Method}
% ====================================================================

We employ gradient-based optimization with:
\begin{enumerate}
    \item Numerical gradients computed via finite differences
    \item Projected gradient ascent ensuring constraint satisfaction  
    \item Line search for optimal step sizes
    \item Convergence based on gradient norm and objective improvement
\end{enumerate}

% ====================================================================
\section{Calibration Methodology}
% ====================================================================

\subsection{Parameter Estimation}

For the constant-intensity model, we estimate:
\begin{itemize}
    \item \textbf{Drift parameters}: $\mu_X, \mu_Y$ from sample means of log-returns
    \item \textbf{Diffusion parameters}: $\sigma_X, \sigma_Y, \rho$ from sample covariance
    \item \textbf{Jump intensities}: $\hat\lambda_X, \hat\lambda_Y$ from jump counting
    \item \textbf{Jump size parameters}: $\delta_X, \eta_X, \delta_Y, \eta_Y$ via maximum likelihood
\end{itemize}

\subsection{Jump Detection}

We implement peak-over-threshold (POT) methods:
\begin{enumerate}
    \item Select thresholds to minimize skewness and excess kurtosis of residual returns
    \item Count exceedances to estimate jump intensities
    \item Fit shifted exponential distributions to jump sizes
\end{enumerate}

\subsection{Model Validation}

Validation includes:
\begin{itemize}
    \item Residual analysis of filtered diffusion components
    \item Out-of-sample performance of liquidation probability forecasts
    \item Comparison with simpler geometric Brownian motion models
\end{itemize}

% ====================================================================
\section{Extensions and Limitations}
% ====================================================================

\subsection{Hawkes Process Comparison}

While our constant-intensity approach sacrifices the ability to model jump clustering, it provides several advantages:
\begin{itemize}
    \item \textbf{Analytical tractability}: Explicit Laplace transforms vs. complex Riccati systems
    \item \textbf{Numerical stability}: Robust inversion algorithms vs. 12-dimensional ODE solving
    \item \textbf{Computational efficiency}: Fast evaluation suitable for real-time applications
    \item \textbf{Parameter parsimony}: Fewer parameters reduce overfitting risk
\end{itemize}

The trade-off is loss of dynamic clustering effects where jump events increase the probability of subsequent jumps.

\subsection{Future Extensions}

Potential extensions include:
\begin{enumerate}
    \item \textbf{Regime-switching intensities}: Markov-modulated jump processes
    \item \textbf{Stochastic volatility}: Time-varying diffusion parameters  
    \item \textbf{Multi-asset portfolios}: Extension beyond pairwise long-short positions
    \item \textbf{Dynamic hedging}: Optimal rebalancing under transaction costs
\end{enumerate}

% ====================================================================
\section{Use Cases-Applied Strategies}
% ====================================================================
In order to show the practical implications of our findings, we present three examples of long-short strategies and the empirical results.

\begin{table}[h!]
\centering
\small
\caption{Comparison of Four Crypto Strategies}
\begin{tabular}{|l|p{2.5cm}|p{2.5cm}|p{2.5cm}|p{2.5cm}|}
\hline
\textbf{Feature / Parameter} & \textbf{Stablecoin LS Volatile} & \textbf{Momentum} & \textbf{Beta-Neutral LS} & \textbf{Leveraged Yield USDC} \\ \hline

Collateral & Stablecoin & Crypto assets & Long crypto & USDC \\ \hline

Leverage & None & Alpha-scaled & Depends on allocation $r$ & Up to $\frac{1}{1-\phi}$ (looped borrowing) \\ \hline

Target Return & Yield spread of long/short & Trending returns & $R_P = w_L \alpha_L - w_S \alpha_S$ & $r_{\mathrm{net}}^{(\infty)} = \frac{r_A - \varphi r_b}{1-\varphi}$ \\ \hline

Main Risks & Volatility, short squeezes & Trend reversals, shocks & Volatility, correlation breakdown, liquidation & BTC price moves, liquidation, fees/slippage \\ \hline

Log-Health Process & $\log \left( \frac{b_X \, w_X \, X(t)}{w_Y \, Y(t)} \right)$ & $log \left( \frac{b_L \, w_L\, L(t)}{w_S \, S(t)} \right)$ & $\log \left( \frac{b_L \, w_L\, L(t)}{w_S \, S(t)} \right)$ & $\log \frac{(b_X w_X X(t))}{D_t}$ \\ \hline

Drift & $\kappa_h=\mu_X -\mu_Y$ & Can be simplified to $\mu_Y$ for risk-specific purposes & $\kappa_h=w_L \mu_L - w_S \mu_S$  & $r_A/(1-\varphi) - r_b \varphi/(1-\varphi) - \mu_Y$ \\ \hline

Volatility / Jump & $\sigma_X,\sigma_Y, \lambda_Y$ & $\sigma_L,\sigma_S,\lambda_L,\lambda_S$ & $\sigma_L,\sigma_S,\lambda_L,\lambda_S$ & $\sigma_Y,\lambda_Y,\delta_Y$ \\ \hline

Terminal Wealth & $W_T=W_0 + R_{\text{long}} - R_{\text{short}}$ & $W_T=W_0 + w_L R_L - w_S R_S$ & $W_T=W_0 + w_L R_L - w_S R_S$ & $W_T=H_t - D_t$ \\ \hline

Optimization & Max expected return / utility & Max Sharpe / alpha & Max utility / Kelly $f^*(r)$ & Max $r_{\mathrm{net}}(\varphi)$ subject to drift>0 \\ \hline

Calibration & Stablecoin spreads & Momentum factor history & Rolling $\mu_i,\sigma_i,\lambda_i,\delta_i$ & BTC returns, $r_A,r_b$ on-chain \\ \hline
\end{tabular}
\label{tab:crypto_strategy_comparison}
\end{table}
Sidenote: $D_t$ represents outstanding debt.


\subsection{Long Stablecoin–Short Volatile}

For the first use case, we consider a long–short strategy in which a stablecoin is chosen for the long position while a volatile cryptocurrency is borrowed to establish a short position. The economic incentive is based on the following ideas: 
\begin{itemize}
    \item the stablecoin is a relatively safe investment that earns a deterministic supply rate $r_X$, and
    \item the short side is expected to appreciate in value, should the borrowed asset depreciate.
\end{itemize}

The net annualized carry is equal to the spread between supply and borrow rates, scaled by position weights:
\begin{equation}
 R(w_X, w_Y) =  w_X r_X - w_Y r_Y,  
\end{equation}
where $r_X$ represents the supply APY corresponding to the long asset and $r_Y$ represents the borrow APY of the borrowed token.

As mentioned earlier, the liquidation risk is determined by the log-health process. We assume $\log X(t)$ and $\log Y(t)$ to be L\'evy processes with Laplace exponents $\psi_X(\theta)$ and $\psi_Y(\theta)$. If $\log X$ and $\log Y$ are independent:
\begin{equation}
h(t) = \log\!\left( \frac{b_X w_X X_0}{w_Y Y_0} \right) + \big(\log X(t) - \log Y(t)\big),
\end{equation}
the Laplace exponent of $h(t)$ is:
\begin{equation}
\psi_h(\theta) = \psi_X(\theta) + \psi_Y(-\theta).
\end{equation}
In the case of correlated L\'evy drivers with a joint characteristic exponent ${\bf{\Psi}}_{X,Y}(\theta_1,\theta_2)$, one obtains:
\begin{equation}
\psi_h(\theta) = {\bf{\Psi}}_{X,Y} (\theta, - \theta). 
\end{equation}

\paragraph{Terminal Wealth.}
Under the spectrally negative assumption, the terminal wealth at time $T$ is:
\begin{equation}
 W_T = W_0 + w_XR_X(T) - w_YR_Y(T) + w_X r_X T - w_Y r_Y T,
\end{equation}
where $R_X(T) = \frac{X_T}{X_0} - 1$ and $R_Y(T) = \frac{Y_T}{Y_0} - 1$.

\paragraph{Expected Wealth.}
\begin{equation}
\mathbb{E}[W_T] = W_0 + w_X\!\big(e^{T \psi_X(1)}-1\big) - w_Y\!\big(e^{T \psi_Y(1)}-1\big) 
+ w_X r_X T - w_Y r_Y T.
\end{equation}

\paragraph{Variance of Wealth.}
Let $A = w_X R_X(T) - w_Y R_Y(T)$. Then:
\begin{equation}
\text{Var}(W_T) = \text{Var}(A)
= w_X^2 \text{Var}[R_X(T)] + w_Y^2 \text{Var}[R_Y(T)] - 
2w_X w_Y  \text{Cov}[R_X(T), R_Y(T)].
\end{equation}
If the L\'evy log-processes are independent:
\begin{equation}
\text{Var}[R_X(T)] = e^{T \psi_X(2)} - e^{2T \psi_X(1)}, 
\quad 
\text{Var}[R_Y(T)] = e^{T \psi_Y(2)} - e^{2T \psi_Y(1)}.
\end{equation}
For correlated L\'evy drivers:
\begin{equation}
\text{Cov}[R_X(T),R_Y(T)] = e^{T {\bf \Psi}_{X,Y}(1,1)} - e^{T \psi_X(1)} e^{T \psi_Y(1)}.
\end{equation}

A stablecoin's price movements can be assumed to be negligeble: ($\psi_X(1) \approx \psi_X(2) \approx 0$):
\begin{align}
\mathbb{E}[W_T] &\approx W_0 - w_Y\big(e^{T \psi_Y(1)}-1\big) 
+ w_X r_X T - w_Y r_Y T, \\
\text{Var}(W_T) &\approx w_Y^2 \big(e^{T \psi_Y(2)} - e^{2T \psi_Y(1)}\big).
\end{align}

\paragraph{Mean–Variance Optimization.}
The utility function of the investor, parameterized by the allocation ratio $r$, is:
\begin{equation}
\max_{r} \quad \mathcal{U}(r) 
= \mathbb{E}[W_T(r)] 
- \rho_1 \, \mathbb{V}[W_T(r)] 
- \rho_2 \, \mathbb{P}(\tau_0 \leq T^*),
\end{equation}
where $W_T(r)$ follows the above wealth process and the initial log-health is
\begin{equation}
h_0(r) = \log \left( \frac{b_X w_X X_0}{w_Y Y_0} \right) 
= \log b_X + \log r + \log\!\left(\frac{X_0}{Y_0}\right).
\end{equation}
The trade-off between expected carry and liquidation probability defines the feasible region of optimal allocations.

\paragraph{Kelly Criterion Implementation.}
To integrate the Kelly criterion, we maximize the long-term growth rate of wealth. The core components ----win probability ($p$), loss probability ($q$), and net odds ($b$) ---- are derived from the jump-diffusion and liquidation framework.

\subparagraph{Win/Loss Probabilities.}
Define a binary outcome over time $T^*$:
\begin{itemize}
    \item \textbf{Loss (Liquidation):} Probability $q = \pi(T^*) = \mathbb{P}(\tau_0 \le T^*)$. Net loss $R_{\text{loss}} = -L w_X$.
    \item \textbf{Win (No Liquidation):} Probability $p = 1-q$. Expected gain is the net return from the carry and the short leg.
\end{itemize}

\subparagraph{Net Odds.}
\begin{align}
q &= \pi(T^*) = \mathcal{L}^{-1}\!\left[\frac{e^{-h_0 \Phi(q)}}{q}\right](T^*), & p &= 1-q, \\
b &= \frac{\mathbb{E}[\text{Gain} \mid \tau_0 > T^*]}{\mathbb{E}[\text{Loss} \mid \tau_0 \le T^*]} 
= \frac{w_X r_X T^* - w_Y r_Y T^* - w_Y(e^{T^* \psi_Y(1)}-1)}{L w_X}.
\end{align}

\paragraph{Kelly Optimization.}
The optimal Kelly fraction is:
\begin{equation}
f^*(r) = \frac{b(r)p(r) - q(r)}{b(r)},
\end{equation}
and the expected log-growth rate:
\begin{equation}
G(r) = q(r) \log\!\big(1 - f^*(r) L w_X\big) + p(r) \log\!\big(1 + f^*(r) b(r) L w_X\big).
\end{equation}
A fractional Kelly approach mitigates estimation risk:
\begin{equation}
f_{\text{target}}(r) = k \times f^*(r), \quad k \in (0,1).
\end{equation}

\paragraph{Solution Procedure.}
Numerical steps for both optimization frameworks:
\begin{enumerate}
    \item Compute $w_X(r)$, $w_Y(r)$, and initial log-health $h_0(r)$.
    \item Evaluate the probability of liquidation $q(r) = \pi(T^*)$ by numerical Laplace inversion; set $p(r) = 1 - q(r)$.
    \item Compute expected wealth and variance for mean–variance optimization.
    \item Calculate the net odds $b(r)$ and the Kelly fraction $f^*(r)$ for the Kelly problem.
    \item Optimize over $r$, ensuring feasibility: $h_0(r) \ge h_{\min}$ and $0 < r < 1$.
\end{enumerate}

\paragraph{Calibration and DeFi Market Data.}
We calibrate the model using data from decentralized lending protocols and cryptocurrency markets. Stablecoin supply rates $r_X$ typically range from 2--6\% per annum, while borrowing rates for volatile assets (e.g. ETH) range from 5--10\%.  
L\'evy jump–diffusion parameters $(\mu_i, \sigma_i, \lambda_i, \delta_i)$ are estimated using daily log returns. Stablecoins exhibit negligible diffusion and jump components ($\sigma_X \approx 0$, $\lambda_X \approx 0$), while volatile assets show significant jump intensities and fat tails.

\paragraph{Health Process Parameters.}
The log-health process exponent is given by $\psi_h(\theta) = \psi_X(\theta) + \psi_Y(-\theta)$. The drift $\kappa_h = \mathbb{E}[h(t)]/t$ determines whether the regime is defective ($\kappa_h>0$) or almost-sure liquidation ($\kappa_h<0$).

\paragraph{Numerical Evaluation.}
For each parameter set, we compute:
\begin{itemize}
    \item $\mathbb{E}[W_T(r)]$ and $\mathbb{V}[W_T(r)]$ for mean–variance optimization,
    \item $\mathbb{P}(\tau_0 \leq T^*)$ by Laplace inversion,
    \item $f^*(r)$ and $G(r)$ for Kelly optimization.
\end{itemize}
Laplace inversions use the Gaver–Stehfest algorithm with $N=16$ terms.

\paragraph{Parameter Values.}
\begin{align*}
r_X &= 0.04, \quad r_Y = 0.08,\quad\sigma_Y = 0.85,\\
\lambda_Y &= 0.12,\quad \delta_Y = 0.25, \quad h_{\min} = 0.1,\\
T^* &= 30 \text{ days.}
\end{align*}

\paragraph{Results.}
Simulation results yield optimal allocation ratios $r^*$ under both optimization frameworks:
\begin{enumerate}
    \item Under mean–variance optimization, $r^*$ balances carry and volatility risk, with higher $\rho_1$ reducing leverage.
    \item Under the Kelly criterion, $r^*$ maximizes geometric growth, explicitly accounting for liquidation probability and expected gain.
    \item For defective-hitting regimes ($\kappa_h > 0$), feasible allocations exist with moderate leverage and low liquidation probability.
    \item In almost-sure liquidation regimes ($\kappa_h < 0$), both frameworks converge to $r \to 1$, implying that leveraged shorting is unsustainable.
\end{enumerate}

\paragraph{Comparative Analysis.}
The Kelly framework provides a direct growth-optimal benchmark, while the mean–variance approach offers a risk-adjusted perspective more suitable for institutional investors. Fractional Kelly scaling aligns the two, yielding practical allocations that manage risk without sacrificing growth potential.

\paragraph{Implementation.}
All computations are performed in \texttt{Python}. Laplace inversions use \texttt{mpmath}, and projected gradient ascent ensures $h_0(r) \ge h_{\min}$ at each iteration. The code simulates jump–diffusion dynamics, computes $\pi(T^*)$, and outputs optimal $r^*$, expected wealth, variance, and liquidation probabilities for varying $(\rho_1, \rho_2)$ and Kelly fractions $k$.

\subsection{Beta-Neutral Long--Short Strategy}

The beta-neutral long-short strategy isolates idiosyncratic returns by neutralizing exposure to the aggregate cryptocurrency market. The portfolio takes a long position in a high-alpha asset $\alpha_L$ and a short position in an asset with similar beta but lower alpha $\alpha_S$, capturing the alpha spread while mitigating systematic risk.  

\paragraph{Portfolio Construction.}  
Let $\beta_i$ denote the market beta of asset $i$. Construct long leg $\mathcal{L}$ and short leg $\mathcal{S}$ with weights $w_{\mathcal{L}}, w_{\mathcal{S}}$. Beta-neutrality requires
\[
\beta_P = \sum_{i \in \mathcal{L}} w_i \beta_i - \sum_{j \in \mathcal{S}} w_j \beta_j \approx 0,
\]
enforced via scaling between legs while maintaining positive collateralization.

\paragraph{Expected Return and Variance.}  
Ignoring liquidation events, the expected terminal wealth is
\begin{equation}
\mathbb{E}[W_T] = W_0 + w_L(e^{T\psi_L(1)}-1) - w_S(e^{T\psi_S(1)}-1),
\end{equation}
and the variance is
\begin{equation}
\text{Var}[W_T] = w_L^2 \text{Var}[R_L(T)] + w_S^2 \text{Var}[R_S(T)] - 2 w_L w_S \text{Cov}[R_L(T),R_S(T)],
\end{equation}
with $R_i(T) = \frac{P_i(T)}{P_i(0)} - 1$ and $\psi_i(\theta)$ the Laplace exponent of the Lévy process for asset $i$.

\paragraph{Log-Health Process.}  
Define
\begin{equation}
h(t) = \log\left(\frac{b_L w_L L_0}{w_S S_0}\right) + (\log L(t) - \log S(t)),
\end{equation}
with $L(t)$, $S(t)$ the long and short price processes and $b_L$ a collateral factor. The Laplace exponent of $h(t)$ is
\begin{equation}
\psi_h(\theta) = \psi_L(\theta) + \psi_S(-\theta),
\end{equation}
or, in the presence of correlated drivers,
\begin{equation}
\psi_h(\theta) = {\bf{\Psi}}_{L,S}(\theta,-\theta).
\end{equation}

\paragraph{Kelly Criterion Implementation.}  
We define a binary outcome over horizon $T^*$:
\begin{itemize}
    \item \textbf{Loss Event:} liquidation occurs with probability $q = \pi(T^*) = \mathbb{P}(\tau_0 \le T^*)$. The net loss is $R_{\text{loss}} = -L w_L$.
    \item \textbf{Win Event:} no liquidation, probability $p = 1-q$. The gain is the expected idiosyncratic return $R_{\text{gain}} = w_L \alpha_L T^* - w_S \alpha_S T^*$.
\end{itemize}

The net odds are
\begin{equation}
b = \frac{R_{\text{gain}}}{|R_{\text{loss}}|} = \frac{w_L \alpha_L T^* - w_S \alpha_S T^*}{L w_L}.
\end{equation}

The optimal Kelly fraction is
\begin{equation}
f^*(r) = \frac{b(r)p(r)-q(r)}{b(r)},
\end{equation}
and the expected logarithmic growth to maximize is
\begin{equation}
G(r) = q(r) \log(1 - f^*(r) L w_L) + p(r) \log(1 + f^*(r) b(r) L w_L).
\end{equation}

For practical implementation, a fractional Kelly $f_{\rm target}(r) = k f^*(r)$ (e.g., $k=0.5$) is recommended to reduce tail risk.  

\paragraph{Solution.}  
\begin{enumerate}
    \item For a given allocation ratio $r$, compute $w_L(r), w_S(r)$ and initial log-health $h_0(r)$.
    \item Estimate $q(r) = \pi(T^*)$ via Laplace inversion.
    \item Compute $p(r) = 1-q(r)$ and net odds $b(r)$.
    \item Evaluate $f^*(r)$ and $G(r)$.
    \item Maximize $G(r)$ numerically over feasible $r$ with $h_0(r) \ge h_{\min}$.
\end{enumerate}

\paragraph{Calibration and Data.}  
Historical crypto returns are used to calibrate $(\mu_i, \sigma_i, \lambda_i, \delta_i)$ for Lévy jump-diffusion processes. Long and short assets are chosen to enforce $\beta_P \approx 0$. Strong idiosyncratic alpha differences yield positive drift in $h(t)$ and reduced liquidation probability.

\paragraph{Baseline Parameters.}
\begin{align*}
\mu_L &= 0.08, &\sigma_L &= 0.75, &\lambda_L &= 0.08, &\delta_L &= 0.20,\\
\mu_S &= 0.04, &\sigma_S &= 0.70, &\lambda_S &= 0.12, &\delta_S &= 0.22,\\
T^* &= 30 \text{ days}, &h_{\min} &= 0.1.
\end{align*}

\paragraph{Key Insights.}  
\begin{itemize}
    \item Beta-neutral portfolios have lower variance than directional positions for similar expected returns.
    \item Positive drift in $h(t)$ ensures feasibility of larger allocations.
    \item Heavy-tailed jumps or strong correlations restrict the feasible allocation ratio $r$ for maintaining beta neutrality and healthy collateral.
\end{itemize}



\subsection{Beta-Neutral Long--Short Strategy}

The third case we consider is a beta-neutral long-short strategy: this strategy aims to isolate idiosyncratic returns by neutralizing exposure to the aggregate cryptocurrency market. The portfolio takes a long position in a high-alpha asset $\alpha_L$ and a short position in an asset with similar beta but lower alpha $\alpha_S$, capturing the alpha spread while mitigating systematic risk.  

\paragraph{Portfolio Construction.}  
Let $\beta_i$ denote the market beta of asset $i$. Construct long leg $\mathcal{L}$ and short leg $\mathcal{S}$ with weights $w_{\mathcal{L}}, w_{\mathcal{S}}$. Beta-neutrality requires
\[
\beta_P = \sum_{i \in \mathcal{L}} w_i \beta_i - \sum_{j \in \mathcal{S}} w_j \beta_j \approx 0,
\]
enforced via scaling between legs while maintaining positive collateralization.

\paragraph{Expected Return and Variance.}  
Ignoring liquidation events, the expected terminal wealth is
\begin{equation}
\mathbb{E}[W_T] = W_0 + w_L(e^{T\psi_L(1)}-1) - w_S(e^{T\psi_S(1)}-1),
\end{equation}
and the variance is
\begin{equation}
\text{Var}[W_T] = w_L^2 \text{Var}[R_L(T)] + w_S^2 \text{Var}[R_S(T)] - 2 w_L w_S \text{Cov}[R_L(T),R_S(T)],
\end{equation}
with $R_i(T) = \frac{P_i(T)}{P_i(0)} - 1$ and $\psi_i(\theta)$ the Laplace exponent of the Lévy process for asset $i$.

\paragraph{Log-Health Process.}  
Define
\begin{equation}
h(t) = \log\left(\frac{b_L w_L L_0}{w_S S_0}\right) + (\log L(t) - \log S(t)),
\end{equation}
with $L(t)$, $S(t)$ the long and short price processes and $b_L$ a collateral factor. The Laplace exponent of $h(t)$ is
\begin{equation}
\psi_h(\theta) = \psi_L(\theta) + \psi_S(-\theta),
\end{equation}
or, in the presence of correlated drivers,
\begin{equation}
\psi_h(\theta) = {\bf{\Psi}}_{L,S}(\theta,-\theta).
\end{equation}

\paragraph{Kelly Criterion Implementation.}  
We define a binary outcome over horizon $T^*$:
\begin{itemize}
    \item \textbf{Loss Event:} liquidation occurs with probability $q = \pi(T^*) = \mathbb{P}(\tau_0 \le T^*)$. The net loss is $R_{\text{loss}} = -L w_L$.
    \item \textbf{Win Event:} no liquidation, probability $p = 1-q$. The gain is the expected idiosyncratic return $R_{\text{gain}} = w_L \alpha_L T^* - w_S \alpha_S T^*$.
\end{itemize}

The net odds are
\begin{equation}
b = \frac{R_{\text{gain}}}{|R_{\text{loss}}|} = \frac{w_L \alpha_L T^* - w_S \alpha_S T^*}{L w_L}.
\end{equation}

The optimal Kelly fraction is
\begin{equation}
f^*(r) = \frac{b(r)p(r)-q(r)}{b(r)},
\end{equation}
and the expected logarithmic growth to maximize is
\begin{equation}
G(r) = q(r) \log(1 - f^*(r) L w_L) + p(r) \log(1 + f^*(r) b(r) L w_L).
\end{equation}

For practical implementation, a fractional Kelly $f_{\rm target}(r) = k f^*(r)$ (e.g., $k=0.5$) is recommended to reduce tail risk.  

\paragraph{Solution.}  
\begin{enumerate}
    \item For a given allocation ratio $r$, compute $w_L(r), w_S(r)$ and initial log-health $h_0(r)$.
    \item Estimate $q(r) = \pi(T^*)$ via Laplace inversion.
    \item Compute $p(r) = 1-q(r)$ and net odds $b(r)$.
    \item Evaluate $f^*(r)$ and $G(r)$.
    \item Maximize $G(r)$ numerically over feasible $r$ with $h_0(r) \ge h_{\min}$.
\end{enumerate}

\paragraph{Calibration and Data.}  
Historical crypto returns are used to calibrate $(\mu_i, \sigma_i, \lambda_i, \delta_i)$ for Lévy jump-diffusion processes. Long and short assets are chosen to enforce $\beta_P \approx 0$. Strong idiosyncratic alpha differences yield positive drift in $h(t)$ and reduced liquidation probability.

\paragraph{Baseline Parameters.}
\begin{align*}
\mu_L &= 0.08, &\sigma_L &= 0.75, &\lambda_L &= 0.08, &\delta_L &= 0.20,\\
\mu_S &= 0.04, &\sigma_S &= 0.70, &\lambda_S &= 0.12, &\delta_S &= 0.22,\\
T^* &= 30 \text{ days}, &h_{\min} &= 0.1.
\end{align*}

\paragraph{Key Insights.}  
\begin{itemize}
    \item Beta-neutral portfolios have lower variance than directional positions for similar expected returns.
    \item Positive drift in $h(t)$ ensures feasibility of larger allocations.
    \item Heavy-tailed jumps or strong correlations restrict the feasible allocation ratio $r$ for maintaining beta neutrality and healthy collateral.
\end{itemize}

\subsection{Leveraged Yield Strategy: USDC Collateral, Borrow BTC, Swap to USDC, Deposit to AAVE}

Consider an initial USDC collateral \(C_0>0\). Using the protocol's collateral factor \(\varphi \in (0,1)\), BTC is borrowed up to a USD-valued amount \(\varphi C_0\). The borrowed BTC is immediately swapped into USDC and deposited back into AAVE at a supply rate \(r_A\). Repeating the borrow–swap–deposit loop \(k\) times generates a geometric series of leveraged deposits. The primary risks are the BTC borrow interest \(r_b\) and mark-to-market exposure to BTC price movements, which can trigger liquidation.

\paragraph{Notation.}
\begin{itemize}
  \item \(C_0\) — initial USDC collateral.
  \item \(S_t\) — spot price of BTC in USDC.
  \item \(\varphi\) — protocol collateral factor.
  \item \(r_A\) — AAVE supply APY for USDC.
  \item \(r_b\) — borrow interest rate for BTC (USD-equivalent).
  \item \(k\) — number of leverage loops.
  \item \(D_t\) — USD value of outstanding debt.
  \item \(H_t\) — deposited USDC value.
  \item \(h_t = \log H_t - \log D_t\) — log-health process.
  \item \(L\) — liquidation threshold in log-health: liquidation occurs when \(h_t \le -\ell\).
\end{itemize}

\paragraph{Leverage Accounting.}
After \(k\) loops, the total deposited USDC is
\[
H^{(k)} = C_0 \sum_{i=0}^{k} \varphi^{\,i} = C_0 \frac{1-\varphi^{\,k+1}}{1-\varphi},
\]
and the USD-valued debt is
\[
D^{(k)} = C_0 \sum_{i=1}^{k} \varphi^{\,i} = C_0\left(\frac{1-\varphi^{\,k+1}}{1-\varphi}-1\right).
\]
For \(k \to \infty\), these converge to
\[
H^{(\infty)} = \frac{C_0}{1-\varphi}, \qquad D^{(\infty)} = \frac{\varphi}{1-\varphi}\,C_0.
\]

\paragraph{Instantaneous Net Yield.}
Ignoring BTC mark-to-market, fees, and slippage, the infinite-loop net yield is
\[
r_{\mathrm{net}}^{(\infty)}(\varphi) = \frac{r_A - \varphi\, r_b}{1-\varphi}.
\]

\paragraph{Price Risk and Log-Health Process.}
Let \(X_t\) denote the log-value process of USDC deposits and \(Y_t = \log S_t\) the BTC log-price, with Lévy exponents \(\psi_X(\theta)\) and \(\psi_Y(\theta)\). The log-health process exponent is
\[
\psi_h(\theta) = \psi_X(\theta) + \psi_Y(-\theta),
\]
reflecting that upward BTC moves reduce the USD-equivalent debt burden, and downward moves increase it.

Liquidation occurs at the first hitting time
\[
\tau_{\mathrm{liq}} = \inf \{ t \ge 0 : h_t \le -\ell \}.
\]
The drift and jump intensity of \(Y_t\) determine survival probabilities to a given horizon and influence the feasibility of leverage \(\varphi\).

\paragraph{First-Order Stability.}
A positive drift of log-health implies
\[
\frac{r_A}{1-\varphi} - \frac{r_b \varphi}{1-\varphi} - \mu_Y > 0,
\]
with \(\mu_Y = -\mathbb{E}[\dot Y_t]\) representing the effect of BTC price drift. Ignoring \(\mu_Y\) yields the heuristic constraint \(\varphi < r_A / r_b\).

\paragraph{Fees, Slippage, and Gas.}
Realized performance is affected by swap fees, gas costs, variable borrow rates, and liquidation penalties. High leverage and repeated loops amplify these effects.

\paragraph{Health Process Parameters.}
The Lévy exponents \(\psi_X(\theta)\) and \(\psi_Y(\theta)\) define
\[
\psi_h(\theta) = \psi_X(\theta) + \psi_Y(-\theta),
\]
and the drift and tail behavior of \(\psi_h\) determine whether the system is almost-surely safe or defective. Positive drift and light left-tail jumps in BTC favor larger feasible \(\varphi\); negative drift and heavy left-tail jumps require smaller \(\varphi\) to avoid liquidation.

\paragraph{Worked Example.}
With \(r_A = 0.046\) and \(r_b = 0.02\), the infinite-loop net yield is
\[
r_{\mathrm{net}}^{(\infty)}(\varphi) = \frac{0.046 - 0.02\varphi}{1-\varphi}.
\]
For \(\varphi = 0.5\),
\[
r_{\mathrm{net}}^{(\infty)}(0.5) = \frac{0.046-0.01}{0.5} = 0.072 = 7.2\%,
\]
ignoring BTC volatility, liquidation mechanics, slippage, and fees.

% ====================================================================
\section{Conclusion}
% ====================================================================

We have developed a practical framework for first-hitting time analysis in DeFi long-short positioning using constant-intensity jump-diffusion models. Although simpler than Hawkes processes, this approach provides robust semi-analytical solutions suitable for real-time risk management.

Key contributions include:
\begin{enumerate}
    \item Spectrally negative Lévy process formulation for log-health dynamics
    \item Semi-analytical computation via Laplace transforms and Gaver-Stehfest inversion
    \item Practical optimization framework balancing returns and liquidation risk
    \item Straightforward calibration methodology for constant-intensity parameters
\end{enumerate}

The framework demonstrates that sophisticated risk management in DeFi applications need not require computationally intensive models. The constant-intensity approach provides sufficient complexity to capture wrong-way risk while maintaining the analytical tractability necessary for practical implementation.

% ====================================================================
% APPENDIX
% ====================================================================

\appendix

\section{Mathematical Proofs}
\label{sec:proofs}

\begin{delayedproof}{prop:log_health_dynamics}
We prove this by applying Itô's lemma to the log-health function and carefully tracking jump contributions from both assets.

\textbf{Step 1: Setup and Itô decomposition}

Starting from $h(t) = \log(b_X w_X) + \log X(t) - \log(w_Y) - \log Y(t)$, define $f(x,y) = \log x - \log y$ so that $h(t) = \text{const} + f(X(t), Y(t))$.

Applying Itô's lemma to $f(X(t), Y(t))$ with jump-diffusion processes:
\begin{align}
df(X(t), Y(t)) &= \frac{\partial f}{\partial x} dX(t) + \frac{\partial f}{\partial y} dY(t) + \frac{1}{2}\frac{\partial^2 f}{\partial x^2} d[X]_t^c + \frac{1}{2}\frac{\partial^2 f}{\partial y^2} d[Y]_t^c \nonumber\\
&\quad + \frac{\partial^2 f}{\partial x \partial y} d[X,Y]_t^c + \sum_{\text{jumps}} \Delta f
\end{align}

\textbf{Step 2: Compute partial derivatives}

For $f(x,y) = \log x - \log y$:
\begin{align}
\frac{\partial f}{\partial x} &= \frac{1}{x}, \quad \frac{\partial f}{\partial y} = -\frac{1}{y}\\
\frac{\partial^2 f}{\partial x^2} &= -\frac{1}{x^2}, \quad \frac{\partial^2 f}{\partial y^2} = \frac{1}{y^2}, \quad \frac{\partial^2 f}{\partial x \partial y} = 0
\end{align}

\textbf{Step 3: Continuous part analysis}

The continuous parts of $X(t)$ and $Y(t)$ satisfy:
\begin{align}
dX(t) &= X(t^-)[\mu_X dt + \sigma_X dB_X(t)] + X(t^-)\Delta J_X(t)\\
dY(t) &= Y(t^-)[\mu_Y dt + \sigma_Y dB_Y(t)] + Y(t^-)\Delta J_Y(t)
\end{align}

For the continuous part (ignoring jumps temporarily):
\begin{align}
df^c(X(t), Y(t)) &= \frac{1}{X(t)}[\mu_X X(t) dt + \sigma_X X(t) dB_X(t)] - \frac{1}{Y(t)}[\mu_Y Y(t) dt + \sigma_Y Y(t) dB_Y(t)] \nonumber\\
&\quad - \frac{1}{2X(t)^2} \sigma_X^2 X(t)^2 dt + \frac{1}{2Y(t)^2} \sigma_Y^2 Y(t)^2 dt
\end{align}

The correct continuous part, accounting for correlation $\rho$, is:
\begin{align}
df^c(X(t), Y(t)) &= [\mu_X - \mu_Y - \frac{1}{2}(\sigma_X^2 + \sigma_Y^2 - 2\rho\sigma_X\sigma_Y)] dt \nonumber\\
&\quad + \sigma_X dB_X(t) - \sigma_Y dB_Y(t)
\end{align}

\textbf{Step 4: Jump part analysis}

When $X(t)$ jumps at time $T_i$ with size $\Delta J_X(T_i) = U_X^i = -(\delta_X + E_X^i)$:
\begin{align}
\Delta f|_{X\text{-jump}} &= \log(X(T_i^-)(1 + U_X^i)) - \log(X(T_i^-)) = \log(1 + U_X^i) = U_X^i
\end{align}

When $Y(t)$ jumps at time $T_j$ with $\Delta J_Y(T_j) = U_Y^j = +(\delta_Y + E_Y^j)$:
\begin{align}
\Delta f|_{Y\text{-jump}} &= -\log(1 + U_Y^j) = -U_Y^j
\end{align}

\textbf{Step 5: Combine results}

The total jump contribution to $h(t)$ is:
\begin{align}
J_h^X(t) &= \sum_{i=1}^{N_X(t)} U_X^i = \sum_{i=1}^{N_X(t)} [-(\delta_X + E_X^i)]\\
J_h^Y(t) &= \sum_{j=1}^{N_Y(t)} (-U_Y^j) = \sum_{j=1}^{N_Y(t)} [+(\delta_Y + E_Y^j)]
\end{align}

The diffusion coefficient becomes:
\[
\sigma_h^2 = \mathbb{V}[\sigma_X dB_X(t) - \sigma_Y dB_Y(t)] = \sigma_X^2 + \sigma_Y^2 - 2\rho\sigma_X\sigma_Y
\]

Therefore, $h(t) = h_0 + \mu_h t + \sigma_h B_h(t) + J_h^X(t) + J_h^Y(t)$ where all parameters are as stated.
\end{delayedproof}

\begin{delayedproof}{prop:laplace_exponent}
We derive the Laplace exponent by computing the cumulant generating function for each component of $h(t)$.

\textbf{Step 1: Decompose the log-health process}

From Proposition \ref{prop:log_health_dynamics}:
\[
h(t) = h_0 + \mu_h t + \sigma_h B_h(t) + J_h^X(t) + J_h^Y(t)
\]

\textbf{Step 2: Brownian motion contribution}

For $\sigma_h B_h(t)$, the contribution is:
\[
\psi_B(\theta) = \frac{1}{2}\sigma_h^2 \theta^2
\]

\textbf{Step 3: Drift contribution}

The drift contributes $-\mu_h \theta$ (negative sign for spectrally negative processes).

\textbf{Step 4: X-jump contribution}

For $J_h^X(t)$ with intensity $\hat\lambda_X$ and jump sizes $U_X^i = -(\delta_X + E_X^i)$:
\[
\psi_X(\theta) = \hat\lambda_X (\mathbb{E}[e^{\theta U_X}] - 1)
\]

Computing $\mathbb{E}[e^{\theta U_X}]$ where $U_X = -(\delta_X + E_X)$:
\begin{align}
\mathbb{E}[e^{\theta U_X}] &= e^{-\theta \delta_X} \mathbb{E}[e^{-\theta E_X}] = e^{-\theta \delta_X} \frac{\eta_X}{\eta_X + \theta}
\end{align}

\textbf{Step 5: Y-jump contribution}

For $J_h^Y(t)$, Y-jumps contribute negatively to health (they worsen it), so:
\[
\mathbb{E}[e^{\theta(-U_Y)}] = e^{-\theta \delta_Y} \frac{\eta_Y}{\eta_Y + \theta}
\]

\textbf{Step 6: Combine all contributions}

\[
\psi(\theta) = \frac{1}{2}\sigma_h^2 \theta^2 - \mu_h \theta + \hat\lambda_X \left( e^{-\theta \delta_X} \frac{\eta_X}{\eta_X+\theta} - 1 \right) + \hat\lambda_Y \left( e^{-\theta \delta_Y} \frac{\eta_Y}{\eta_Y+\theta} - 1 \right)
\]

The properties follow from direct computation: $\psi(0) = 0$, and 
\[
\psi'(0) = -\mu_h - \hat\lambda_X(\delta_X + 1/\eta_X) - \hat\lambda_Y(\delta_Y + 1/\eta_Y)
\]
\end{delayedproof}

\section{Numerical Methods and Implementation}
\label{sec:numerical_methods}

This appendix details the computational procedures for evaluating first-hitting time distributions through Laplace transform inversion.

\subsection{Gaver-Stehfest Inversion Algorithm}

The Gaver-Stehfest algorithm provides a robust method for numerically inverting Laplace transforms. For the first-hitting time distribution:

\begin{algorithm}
\caption{First-Hitting Time CDF via Gaver-Stehfest Inversion}
\label{alg:gaver_stehfest_appendix}
\begin{algorithmic}[1]
\STATE \textbf{Input:} Initial log-health $h_0 > 0$, evaluation time $T > 0$, Stehfest order $N$ (typically 10-14)
\STATE Compute adjustment coefficient $R = \Phi(0)$ by solving $\psi(R) = 0$
\STATE Precompute Stehfest weights $V_k$ for $k = 1, \ldots, N$:
\[
V_k = (-1)^{N/2+k} \sum_{j=\lfloor(k+1)/2\rfloor}^{\min(k,N/2)} \frac{j^{N/2} (2j)!}{(N/2-j)! j! (j-1)! (k-j)! (2j-k)!}
\]
\FOR{$k = 1$ to $N$}
    \STATE Compute $q_k = k \ln 2 / T$
    \STATE Solve $\psi(\theta) = q_k$ numerically to obtain $\Phi(q_k)$
    \STATE Evaluate $\mathcal{L}\{F\}(q_k) = e^{-h_0 \Phi(q_k)} / q_k$
\ENDFOR
\STATE Compute the inversion:
\[
F(T,h_0) = \frac{\ln 2}{T} \sum_{k=1}^{N} V_k \mathcal{L}\{F\}(q_k)
\]
\STATE Clamp result to $[0,1]$
\STATE \textbf{Output:} $F(T,h_0) = \mathbb{P}(\tau_0 \leq T)$
\end{algorithmic}
\end{algorithm}

For the conditional distribution in defective cases, replace step 8 with:
\[
\mathcal{L}\{F_{\mathrm{cond}}\}(q_k) = \frac{e^{-h_0 (\Phi(q_k+R) - R)}}{q_k}
\]

\subsection{Root Finding for $\Phi(q)$}

The root function $\Phi(q)$ requires solving $\psi(\theta) = q$ for each $q_k$ in the Stehfest algorithm. We employ Brent's method with search interval $[0, \theta_{\max}]$ where $\theta_{\max}$ is chosen to ensure $\psi(\theta_{\max}) > q_{\max}$.

\textbf{Implementation details:}
\begin{itemize}
    \item Initialize with bracketing: $\theta_{\min} = 0$, $\theta_{\max} = 10 \max(\eta_X, \eta_Y)$
    \item Verify $\psi(0) < q < \psi(\theta_{\max})$ before root finding
    \item Use relative tolerance $10^{-12}$ for convergence
    \item Cache $\psi(\theta)$ evaluations to avoid recomputation
\end{itemize}

\subsection{Numerical Stability Considerations}

\begin{itemize}
    \item \textbf{Stehfest order}: Use moderate orders (10-14) to balance accuracy and stability. Higher orders may amplify numerical errors.
    \item \textbf{High precision}: Employ extended precision arithmetic for large $N$ or small $T$ to avoid catastrophic cancellation.
    \item \textbf{Alternative inversion}: For very small $T$ or large $h_0$, consider Talbot or de Hoog methods which may be more stable.
    \item \textbf{Parameter validation}: Ensure $\eta_X + \theta > 0$ and $\eta_Y + \theta > 0$ to avoid singularities.
    \item \textbf{Clamping}: Final probabilities should be clamped to $[0,1]$ to handle numerical errors.
    \item \textbf{Conditioning detection}: Automatically detect defective hitting by checking if $\psi'(0) < 0$.
\end{itemize}

\subsection{Computational Complexity}

For each distribution evaluation:
\begin{itemize}
    \item $O(N)$ Laplace transform evaluations (typically $N = 10-14$)
    \item Each evaluation requires solving $\psi(\theta) = q_k$ via root finding
    \item Root finding: $O(\log \epsilon^{-1})$ iterations for tolerance $\epsilon$
    \item Total complexity: $O(N \log \epsilon^{-1})$ per distribution evaluation
\end{itemize}

% ====================================================================
% BIBLIOGRAPHY
% ====================================================================

%\bibliographystyle{abbrvnamed}
%\bibliography{finance}

\end{document}


\subsection{Beta-Neutral Long--Short Strategy}

The third case we consider is a beta-neutral long-short strategy: this strategy aims to isolate idiosyncratic returns by neutralizing exposure to the aggregate cryptocurrency market. The portfolio takes a long position in a high-alpha asset $\alpha_L$ and a short position in an asset with similar beta but lower alpha $\alpha_S$, capturing the alpha spread while mitigating systematic risk.  

\paragraph{Portfolio Construction.}  
Let $\beta_i$ denote the market beta of asset $i$. Construct long leg $\mathcal{L}$ and short leg $\mathcal{S}$ with weights $w_{\mathcal{L}}, w_{\mathcal{S}}$. Beta-neutrality requires
\[
\beta_P = \sum_{i \in \mathcal{L}} w_i \beta_i - \sum_{j \in \mathcal{S}} w_j \beta_j \approx 0,
\]
enforced via scaling between legs while maintaining positive collateralization.

\paragraph{Expected Return and Variance.}  
Ignoring liquidation events, the expected terminal wealth is
\begin{equation}
\mathbb{E}[W_T] = W_0 + w_L(e^{T\psi_L(1)}-1) - w_S(e^{T\psi_S(1)}-1),
\end{equation}
and the variance is
\begin{equation}
\text{Var}[W_T] = w_L^2 \text{Var}[R_L(T)] + w_S^2 \text{Var}[R_S(T)] - 2 w_L w_S \text{Cov}[R_L(T),R_S(T)],
\end{equation}
with $R_i(T) = \frac{P_i(T)}{P_i(0)} - 1$ and $\psi_i(\theta)$ the Laplace exponent of the Lévy process for asset $i$.

\paragraph{Log-Health Process.}  
Define
\begin{equation}
h(t) = \log\left(\frac{b_L w_L L_0}{w_S S_0}\right) + (\log L(t) - \log S(t)),
\end{equation}
with $L(t)$, $S(t)$ the long and short price processes and $b_L$ a collateral factor. The Laplace exponent of $h(t)$ is
\begin{equation}
\psi_h(\theta) = \psi_L(\theta) + \psi_S(-\theta),
\end{equation}
or, in the presence of correlated drivers,
\begin{equation}
\psi_h(\theta) = {\bf{\Psi}}_{L,S}(\theta,-\theta).
\end{equation}

\paragraph{Kelly Criterion Implementation.}  
We define a binary outcome over horizon $T^*$:
\begin{itemize}
    \item \textbf{Loss Event:} liquidation occurs with probability $q = \pi(T^*) = \mathbb{P}(\tau_0 \le T^*)$. The net loss is $R_{\text{loss}} = -L w_L$.
    \item \textbf{Win Event:} no liquidation, probability $p = 1-q$. The gain is the expected idiosyncratic return $R_{\text{gain}} = w_L \alpha_L T^* - w_S \alpha_S T^*$.
\end{itemize}

The net odds are
\begin{equation}
b = \frac{R_{\text{gain}}}{|R_{\text{loss}}|} = \frac{w_L \alpha_L T^* - w_S \alpha_S T^*}{L w_L}.
\end{equation}

The optimal Kelly fraction is
\begin{equation}
f^*(r) = \frac{b(r)p(r)-q(r)}{b(r)},
\end{equation}
and the expected logarithmic growth to maximize is
\begin{equation}
G(r) = q(r) \log(1 - f^*(r) L w_L) + p(r) \log(1 + f^*(r) b(r) L w_L).
\end{equation}

For practical implementation, a fractional Kelly $f_{\rm target}(r) = k f^*(r)$ (e.g., $k=0.5$) is recommended to reduce tail risk.  

\paragraph{Solution.}  
\begin{enumerate}
    \item For a given allocation ratio $r$, compute $w_L(r), w_S(r)$ and initial log-health $h_0(r)$.
    \item Estimate $q(r) = \pi(T^*)$ via Laplace inversion.
    \item Compute $p(r) = 1-q(r)$ and net odds $b(r)$.
    \item Evaluate $f^*(r)$ and $G(r)$.
    \item Maximize $G(r)$ numerically over feasible $r$ with $h_0(r) \ge h_{\min}$.
\end{enumerate}

\paragraph{Calibration and Data.}  
Historical crypto returns are used to calibrate $(\mu_i, \sigma_i, \lambda_i, \delta_i)$ for Lévy jump-diffusion processes. Long and short assets are chosen to enforce $\beta_P \approx 0$. Strong idiosyncratic alpha differences yield positive drift in $h(t)$ and reduced liquidation probability.

\paragraph{Baseline Parameters.}
\begin{align*}
\mu_L &= 0.08, &\sigma_L &= 0.75, &\lambda_L &= 0.08, &\delta_L &= 0.20,\\
\mu_S &= 0.04, &\sigma_S &= 0.70, &\lambda_S &= 0.12, &\delta_S &= 0.22,\\
T^* &= 30 \text{ days}, &h_{\min} &= 0.1.
\end{align*}

\paragraph{Key Insights.}  
\begin{itemize}
    \item Beta-neutral portfolios have lower variance than directional positions for similar expected returns.
    \item Positive drift in $h(t)$ ensures feasibility of larger allocations.
    \item Heavy-tailed jumps or strong correlations restrict the feasible allocation ratio $r$ for maintaining beta neutrality and healthy collateral.
\end{itemize}

\subsection{Leveraged Yield Strategy: USDC Collateral, Borrow BTC, Swap to USDC, Deposit to AAVE}

Consider an initial USDC collateral \(C_0>0\). Using the protocol's collateral factor \(\varphi \in (0,1)\), BTC is borrowed up to a USD-valued amount \(\varphi C_0\). The borrowed BTC is immediately swapped into USDC and deposited back into AAVE at a supply rate \(r_A\). Repeating the borrow–swap–deposit loop \(k\) times generates a geometric series of leveraged deposits. The primary risks are the BTC borrow interest \(r_b\) and mark-to-market exposure to BTC price movements, which can trigger liquidation.

\paragraph{Notation.}
\begin{itemize}
  \item \(C_0\) — initial USDC collateral.
  \item \(S_t\) — spot price of BTC in USDC.
  \item \(\varphi\) — protocol collateral factor.
  \item \(r_A\) — AAVE supply APY for USDC.
  \item \(r_b\) — borrow interest rate for BTC (USD-equivalent).
  \item \(k\) — number of leverage loops.
  \item \(D_t\) — USD value of outstanding debt.
  \item \(H_t\) — deposited USDC value.
  \item \(h_t = \log H_t - \log D_t\) — log-health process.
  \item \(L\) — liquidation threshold in log-health: liquidation occurs when \(h_t \le -\ell\).
\end{itemize}

\paragraph{Leverage Accounting.}
After \(k\) loops, the total deposited USDC is
\[
H^{(k)} = C_0 \sum_{i=0}^{k} \varphi^{\,i} = C_0 \frac{1-\varphi^{\,k+1}}{1-\varphi},
\]
and the USD-valued debt is
\[
D^{(k)} = C_0 \sum_{i=1}^{k} \varphi^{\,i} = C_0\left(\frac{1-\varphi^{\,k+1}}{1-\varphi}-1\right).
\]
For \(k \to \infty\), these converge to
\[
H^{(\infty)} = \frac{C_0}{1-\varphi}, \qquad D^{(\infty)} = \frac{\varphi}{1-\varphi}\,C_0.
\]

\paragraph{Instantaneous Net Yield.}
Ignoring BTC mark-to-market, fees, and slippage, the infinite-loop net yield is
\[
r_{\mathrm{net}}^{(\infty)}(\varphi) = \frac{r_A - \varphi\, r_b}{1-\varphi}.
\]

\paragraph{Price Risk and Log-Health Process.}
Let \(X_t\) denote the log-value process of USDC deposits and \(Y_t = \log S_t\) the BTC log-price, with Lévy exponents \(\psi_X(\theta)\) and \(\psi_Y(\theta)\). The log-health process exponent is
\[
\psi_h(\theta) = \psi_X(\theta) + \psi_Y(-\theta),
\]
reflecting that upward BTC moves reduce the USD-equivalent debt burden, and downward moves increase it.

Liquidation occurs at the first hitting time
\[
\tau_{\mathrm{liq}} = \inf \{ t \ge 0 : h_t \le -\ell \}.
\]
The drift and jump intensity of \(Y_t\) determine survival probabilities to a given horizon and influence the feasibility of leverage \(\varphi\).

\paragraph{First-Order Stability.}
A positive drift of log-health implies
\[
\frac{r_A}{1-\varphi} - \frac{r_b \varphi}{1-\varphi} - \mu_Y > 0,
\]
with \(\mu_Y = -\mathbb{E}[\dot Y_t]\) representing the effect of BTC price drift. Ignoring \(\mu_Y\) yields the heuristic constraint \(\varphi < r_A / r_b\).

\paragraph{Fees, Slippage, and Gas.}
Realized performance is affected by swap fees, gas costs, variable borrow rates, and liquidation penalties. High leverage and repeated loops amplify these effects.

\paragraph{Health Process Parameters.}
The Lévy exponents \(\psi_X(\theta)\) and \(\psi_Y(\theta)\) define
\[
\psi_h(\theta) = \psi_X(\theta) + \psi_Y(-\theta),
\]
and the drift and tail behavior of \(\psi_h\) determine whether the system is almost-surely safe or defective. Positive drift and light left-tail jumps in BTC favor larger feasible \(\varphi\); negative drift and heavy left-tail jumps require smaller \(\varphi\) to avoid liquidation.

\paragraph{Worked Example.}
With \(r_A = 0.046\) and \(r_b = 0.02\), the infinite-loop net yield is
\[
r_{\mathrm{net}}^{(\infty)}(\varphi) = \frac{0.046 - 0.02\varphi}{1-\varphi}.
\]
For \(\varphi = 0.5\),
\[
r_{\mathrm{net}}^{(\infty)}(0.5) = \frac{0.046-0.01}{0.5} = 0.072 = 7.2\%,
\]
ignoring BTC volatility, liquidation mechanics, slippage, and fees.

% ====================================================================
\section{Conclusion}
% ====================================================================

We have developed a practical framework for first-hitting time analysis in DeFi long-short positioning using constant-intensity jump-diffusion models. Although simpler than Hawkes processes, this approach provides robust semi-analytical solutions suitable for real-time risk management.

Key contributions include:
\begin{enumerate}
    \item Spectrally negative Lévy process formulation for log-health dynamics
    \item Semi-analytical computation via Laplace transforms and Gaver-Stehfest inversion
    \item Practical optimization framework balancing returns and liquidation risk
    \item Straightforward calibration methodology for constant-intensity parameters
\end{enumerate}

The framework demonstrates that sophisticated risk management in DeFi applications need not require computationally intensive models. The constant-intensity approach provides sufficient complexity to capture wrong-way risk while maintaining the analytical tractability necessary for practical implementation.

% ====================================================================
% APPENDIX
% ====================================================================

\appendix

\section{Mathematical Proofs}
\label{sec:proofs}

\begin{delayedproof}{prop:log_health_dynamics}
We prove this by applying Itô's lemma to the log-health function and carefully tracking jump contributions from both assets.

\textbf{Step 1: Setup and Itô decomposition}

Starting from $h(t) = \log(b_X w_X) + \log X(t) - \log(w_Y) - \log Y(t)$, define $f(x,y) = \log x - \log y$ so that $h(t) = \text{const} + f(X(t), Y(t))$.

Applying Itô's lemma to $f(X(t), Y(t))$ with jump-diffusion processes:
\begin{align}
df(X(t), Y(t)) &= \frac{\partial f}{\partial x} dX(t) + \frac{\partial f}{\partial y} dY(t) + \frac{1}{2}\frac{\partial^2 f}{\partial x^2} d[X]_t^c + \frac{1}{2}\frac{\partial^2 f}{\partial y^2} d[Y]_t^c \nonumber\\
&\quad + \frac{\partial^2 f}{\partial x \partial y} d[X,Y]_t^c + \sum_{\text{jumps}} \Delta f
\end{align}

\textbf{Step 2: Compute partial derivatives}

For $f(x,y) = \log x - \log y$:
\begin{align}
\frac{\partial f}{\partial x} &= \frac{1}{x}, \quad \frac{\partial f}{\partial y} = -\frac{1}{y}\\
\frac{\partial^2 f}{\partial x^2} &= -\frac{1}{x^2}, \quad \frac{\partial^2 f}{\partial y^2} = \frac{1}{y^2}, \quad \frac{\partial^2 f}{\partial x \partial y} = 0
\end{align}

\textbf{Step 3: Continuous part analysis}

The continuous parts of $X(t)$ and $Y(t)$ satisfy:
\begin{align}
dX(t) &= X(t^-)[\mu_X dt + \sigma_X dB_X(t)] + X(t^-)\Delta J_X(t)\\
dY(t) &= Y(t^-)[\mu_Y dt + \sigma_Y dB_Y(t)] + Y(t^-)\Delta J_Y(t)
\end{align}

For the continuous part (ignoring jumps temporarily):
\begin{align}
df^c(X(t), Y(t)) &= \frac{1}{X(t)}[\mu_X X(t) dt + \sigma_X X(t) dB_X(t)] - \frac{1}{Y(t)}[\mu_Y Y(t) dt + \sigma_Y Y(t) dB_Y(t)] \nonumber\\
&\quad - \frac{1}{2X(t)^2} \sigma_X^2 X(t)^2 dt + \frac{1}{2Y(t)^2} \sigma_Y^2 Y(t)^2 dt
\end{align}

The correct continuous part, accounting for correlation $\rho$, is:
\begin{align}
df^c(X(t), Y(t)) &= [\mu_X - \mu_Y - \frac{1}{2}(\sigma_X^2 + \sigma_Y^2 - 2\rho\sigma_X\sigma_Y)] dt \nonumber\\
&\quad + \sigma_X dB_X(t) - \sigma_Y dB_Y(t)
\end{align}

\textbf{Step 4: Jump part analysis}

When $X(t)$ jumps at time $T_i$ with size $\Delta J_X(T_i) = U_X^i = -(\delta_X + E_X^i)$:
\begin{align}
\Delta f|_{X\text{-jump}} &= \log(X(T_i^-)(1 + U_X^i)) - \log(X(T_i^-)) = \log(1 + U_X^i) = U_X^i
\end{align}

When $Y(t)$ jumps at time $T_j$ with $\Delta J_Y(T_j) = U_Y^j = +(\delta_Y + E_Y^j)$:
\begin{align}
\Delta f|_{Y\text{-jump}} &= -\log(1 + U_Y^j) = -U_Y^j
\end{align}

\textbf{Step 5: Combine results}

The total jump contribution to $h(t)$ is:
\begin{align}
J_h^X(t) &= \sum_{i=1}^{N_X(t)} U_X^i = \sum_{i=1}^{N_X(t)} [-(\delta_X + E_X^i)]\\
J_h^Y(t) &= \sum_{j=1}^{N_Y(t)} (-U_Y^j) = \sum_{j=1}^{N_Y(t)} [+(\delta_Y + E_Y^j)]
\end{align}

The diffusion coefficient becomes:
\[
\sigma_h^2 = \mathbb{V}[\sigma_X dB_X(t) - \sigma_Y dB_Y(t)] = \sigma_X^2 + \sigma_Y^2 - 2\rho\sigma_X\sigma_Y
\]

Therefore, $h(t) = h_0 + \mu_h t + \sigma_h B_h(t) + J_h^X(t) + J_h^Y(t)$ where all parameters are as stated.
\end{delayedproof}

\begin{delayedproof}{prop:laplace_exponent}
We derive the Laplace exponent by computing the cumulant generating function for each component of $h(t)$.

\textbf{Step 1: Decompose the log-health process}

From Proposition \ref{prop:log_health_dynamics}:
\[
h(t) = h_0 + \mu_h t + \sigma_h B_h(t) + J_h^X(t) + J_h^Y(t)
\]

\textbf{Step 2: Brownian motion contribution}

For $\sigma_h B_h(t)$, the contribution is:
\[
\psi_B(\theta) = \frac{1}{2}\sigma_h^2 \theta^2
\]

\textbf{Step 3: Drift contribution}

The drift contributes $-\mu_h \theta$ (negative sign for spectrally negative processes).

\textbf{Step 4: X-jump contribution}

For $J_h^X(t)$ with intensity $\hat\lambda_X$ and jump sizes $U_X^i = -(\delta_X + E_X^i)$:
\[
\psi_X(\theta) = \hat\lambda_X (\mathbb{E}[e^{\theta U_X}] - 1)
\]

Computing $\mathbb{E}[e^{\theta U_X}]$ where $U_X = -(\delta_X + E_X)$:
\begin{align}
\mathbb{E}[e^{\theta U_X}] &= e^{-\theta \delta_X} \mathbb{E}[e^{-\theta E_X}] = e^{-\theta \delta_X} \frac{\eta_X}{\eta_X + \theta}
\end{align}

\textbf{Step 5: Y-jump contribution}

For $J_h^Y(t)$, Y-jumps contribute negatively to health (they worsen it), so:
\[
\mathbb{E}[e^{\theta(-U_Y)}] = e^{-\theta \delta_Y} \frac{\eta_Y}{\eta_Y + \theta}
\]

\textbf{Step 6: Combine all contributions}

\[
\psi(\theta) = \frac{1}{2}\sigma_h^2 \theta^2 - \mu_h \theta + \hat\lambda_X \left( e^{-\theta \delta_X} \frac{\eta_X}{\eta_X+\theta} - 1 \right) + \hat\lambda_Y \left( e^{-\theta \delta_Y} \frac{\eta_Y}{\eta_Y+\theta} - 1 \right)
\]

The properties follow from direct computation: $\psi(0) = 0$, and 
\[
\psi'(0) = -\mu_h - \hat\lambda_X(\delta_X + 1/\eta_X) - \hat\lambda_Y(\delta_Y + 1/\eta_Y)
\]
\end{delayedproof}

\section{Numerical Methods and Implementation}
\label{sec:numerical_methods}

This appendix details the computational procedures for evaluating first-hitting time distributions through Laplace transform inversion.

\subsection{Gaver-Stehfest Inversion Algorithm}

The Gaver-Stehfest algorithm provides a robust method for numerically inverting Laplace transforms. For the first-hitting time distribution:

\begin{algorithm}
\caption{First-Hitting Time CDF via Gaver-Stehfest Inversion}
\label{alg:gaver_stehfest_appendix}
\begin{algorithmic}[1]
\STATE \textbf{Input:} Initial log-health $h_0 > 0$, evaluation time $T > 0$, Stehfest order $N$ (typically 10-14)
\STATE Compute adjustment coefficient $R = \Phi(0)$ by solving $\psi(R) = 0$
\STATE Precompute Stehfest weights $V_k$ for $k = 1, \ldots, N$:
\[
V_k = (-1)^{N/2+k} \sum_{j=\lfloor(k+1)/2\rfloor}^{\min(k,N/2)} \frac{j^{N/2} (2j)!}{(N/2-j)! j! (j-1)! (k-j)! (2j-k)!}
\]
\FOR{$k = 1$ to $N$}
    \STATE Compute $q_k = k \ln 2 / T$
    \STATE Solve $\psi(\theta) = q_k$ numerically to obtain $\Phi(q_k)$
    \STATE Evaluate $\mathcal{L}\{F\}(q_k) = e^{-h_0 \Phi(q_k)} / q_k$
\ENDFOR
\STATE Compute the inversion:
\[
F(T,h_0) = \frac{\ln 2}{T} \sum_{k=1}^{N} V_k \mathcal{L}\{F\}(q_k)
\]
\STATE Clamp result to $[0,1]$
\STATE \textbf{Output:} $F(T,h_0) = \mathbb{P}(\tau_0 \leq T)$
\end{algorithmic}
\end{algorithm}

For the conditional distribution in defective cases, replace step 8 with:
\[
\mathcal{L}\{F_{\mathrm{cond}}\}(q_k) = \frac{e^{-h_0 (\Phi(q_k+R) - R)}}{q_k}
\]

\subsection{Root Finding for $\Phi(q)$}

The root function $\Phi(q)$ requires solving $\psi(\theta) = q$ for each $q_k$ in the Stehfest algorithm. We employ Brent's method with search interval $[0, \theta_{\max}]$ where $\theta_{\max}$ is chosen to ensure $\psi(\theta_{\max}) > q_{\max}$.

\textbf{Implementation details:}
\begin{itemize}
    \item Initialize with bracketing: $\theta_{\min} = 0$, $\theta_{\max} = 10 \max(\eta_X, \eta_Y)$
    \item Verify $\psi(0) < q < \psi(\theta_{\max})$ before root finding
    \item Use relative tolerance $10^{-12}$ for convergence
    \item Cache $\psi(\theta)$ evaluations to avoid recomputation
\end{itemize}

\subsection{Numerical Stability Considerations}

\begin{itemize}
    \item \textbf{Stehfest order}: Use moderate orders (10-14) to balance accuracy and stability. Higher orders may amplify numerical errors.
    \item \textbf{High precision}: Employ extended precision arithmetic for large $N$ or small $T$ to avoid catastrophic cancellation.
    \item \textbf{Alternative inversion}: For very small $T$ or large $h_0$, consider Talbot or de Hoog methods which may be more stable.
    \item \textbf{Parameter validation}: Ensure $\eta_X + \theta > 0$ and $\eta_Y + \theta > 0$ to avoid singularities.
    \item \textbf{Clamping}: Final probabilities should be clamped to $[0,1]$ to handle numerical errors.
    \item \textbf{Conditioning detection}: Automatically detect defective hitting by checking if $\psi'(0) < 0$.
\end{itemize}

\subsection{Computational Complexity}

For each distribution evaluation:
\begin{itemize}
    \item $O(N)$ Laplace transform evaluations (typically $N = 10-14$)
    \item Each evaluation requires solving $\psi(\theta) = q_k$ via root finding
    \item Root finding: $O(\log \epsilon^{-1})$ iterations for tolerance $\epsilon$
    \item Total complexity: $O(N \log \epsilon^{-1})$ per distribution evaluation
\end{itemize}

% ====================================================================
% BIBLIOGRAPHY
% ====================================================================

%\bibliographystyle{abbrvnamed}
%\bibliography{finance}

\end{document}