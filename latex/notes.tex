\documentclass{article}
\usepackage[margin=1in]{geometry}

% ==== PACKAGES ====
\usepackage{amsmath, amssymb, amsthm}
\usepackage{mathtools}
\usepackage[longnamesfirst,sort&compress]{natbib}
\usepackage{graphicx}
\usepackage{algorithm, algorithmic}
\usepackage{hyperref}
\usepackage{cleveref}
\usepackage{booktabs}
\usepackage{enumitem}

% Citation punctuation style matching original
\bibpunct{(}{)}{;}{a}{,}{,}

% ==== CUSTOM COMMANDS ====
\newcommand{\E}{\mathbb{E}}

\renewcommand{\P}{\mathbb{P}}
\newcommand{\R}{\mathbb{R}}
\newcommand{\diff}{\mathrm{d}}
\newcommand{\pard}[2]{\frac{\partial #1}{\partial #2}}

% Theorem environments
\theoremstyle{definition}
\newtheorem{definition}{Definition}[section]
\newtheorem{theorem}{Theorem}[section]
\newtheorem{proposition}{Proposition}[section]
\newtheorem{lemma}{Lemma}[section]
\newtheorem{remark}{Remark}[section]

% Proof environment for appendix
\newenvironment{delayedproof}[1]
 {\begin{proof}[\raisedtarget{#1}Proof of \Cref{#1}]}
 {\end{proof}}
\newcommand{\raisedtarget}[1]{%
  \raisebox{\fontcharht\font`P}[0pt][0pt]{\hypertarget{#1}{}}%
}
\newcommand{\proofref}[1]{\hyperlink{#1}{proof}}

\title{First-Hitting Time Analysis for Log-Health Process with Constant Jump Intensities}
\author{[Author Names]}
\date{\today}

\begin{document}
\maketitle

\begin{abstract}
We develop an optimal allocation framework for long-short cryptocurrency positions on decentralized finance (DeFi) lending platforms by analyzing first-hitting time distributions for log-health processes under constant-intensity jump-diffusion dynamics. Moving beyond the complexity of Hawkes processes, we employ a spectrally negative Lévy process with constant-intensity compound Poisson jumps to model wrong-way risk in cryptocurrency markets. The approach provides semi-analytical solutions via Laplace transform methods and Gaver-Stehfest inversion, enabling practical computation of margin call probabilities. Our framework balances expected returns against liquidation risk through optimization of position weights subject to collateral constraints.

\textbf{Keywords:} DeFi, long-short strategies, liquidation risk, Lévy processes, first passage times, cryptocurrency
\end{abstract}

% ====================================================================
\section{Introduction}
% ====================================================================

Decentralized finance (DeFi) lending platforms such as AAVE and Compound have emerged as powerful tools for implementing sophisticated long-short cryptocurrency strategies. With combined total value locked exceeding \$10 billion, these protocols enable traders to simultaneously hold long positions in promising assets while borrowing and shorting others, all while earning interest on collateralized deposits. This creates unique opportunities for alpha generation that are unavailable through traditional centralized exchanges.

The core challenge in DeFi long-short positioning lies in managing liquidation risk under extreme market volatility. Unlike traditional margin trading, DeFi platforms require significant overcollateralization (typically 120-150\%) and impose automatic liquidation when health factors fall below unity. Given the notorious volatility of cryptocurrency markets, where daily moves of 10-20\% are common and flash crashes can exceed 50\%, the probability and timing of liquidation events become critical factors in position sizing and risk management.

Wrong-way risk represents the most significant threat to DeFi portfolios: adverse price movements in either the collateral asset (downward) or borrowed asset (upward) directly deteriorate the health factor, with large jumps potentially triggering immediate liquidation. Traditional portfolio theory, which assumes smooth price evolution and continuous rebalancing opportunities, fails to capture the discrete, path-dependent nature of liquidation risk in DeFi environments.

The academic literature on first-hitting times for jump-diffusion processes suggests that Hawkes processes, with their ability to model clustered jumps and cross-excitation between assets, would be ideal for capturing wrong-way risk dynamics. However, our initial investigations revealed that Hawkes-based models, while theoretically appealing, present severe computational challenges. The resulting characteristic function analysis requires solving high-dimensional Riccati systems that are numerically unstable and computationally intensive, making them impractical for real-time risk management applications.

This paper advocates for a more pragmatic approach: modeling the log-health process as a spectrally negative Lévy process with constant jump intensities. While this approach cannot capture the dynamic clustering effects of Hawkes processes, it offers substantial practical advantages:

\begin{enumerate}
    \item \textbf{Analytical tractability}: Explicit Laplace transform representations for first-hitting time distributions
    \item \textbf{Numerical stability}: Robust inversion algorithms that converge reliably across parameter ranges
    \item \textbf{Computational efficiency}: Fast evaluation suitable for real-time portfolio optimization
    \item \textbf{Parameter parsimony}: Fewer parameters reduce calibration complexity and overfitting risk
    \item \textbf{Practical implementation}: Semi-analytical solutions enable deployment in production trading systems
\end{enumerate}

Our framework demonstrates that effective liquidation risk management in DeFi applications need not require the most sophisticated stochastic models. The constant-intensity jump-diffusion approach captures the essential wrong-way risk characteristics while maintaining the computational tractability necessary for practical portfolio management. This represents a conscious trade-off between theoretical completeness and practical utility, prioritizing robust, implementable solutions over academic elegance.

% ====================================================================
\section{Portfolio Setup and Log-Health Process}
% ====================================================================

\subsection{Long-Short Position Construction}

Consider a long-short position on a DeFi lending platform where a user deposits collateral asset $X$ and borrows asset $Y$. The position consists of:
\begin{itemize}
    \item \textbf{Long position}: $w_X$ units of asset $X(t)$ deposited as collateral
    \item \textbf{Short position}: $w_Y$ units of asset $Y(t)$ borrowed for short selling
\end{itemize}

We implement a \textbf{net exposure constraint}:
\begin{equation}
w_X - w_Y = 1, \quad w_X > 0, \quad w_Y > 0 \label{eq:net_exposure}
\end{equation}

This ensures unit net exposure while allowing flexible allocation between long and short positions based on relative expected returns and risk characteristics.

\subsection{Health Factor and Liquidation Mechanics}

The \textbf{log-health process} serves as the primary risk metric:
\begin{equation}
h(t) = \log\left( \frac{b_X w_X X(t)}{w_Y Y(t)} \right) \label{eq:log_health}
\end{equation}

where $b_X \in (0,1]$ is the collateral factor (haircut) applied to the long position. A margin call is triggered when $h(t)$ first hits zero, corresponding to the health factor $H(t) = e^{h(t)} \leq 1$.

The liquidation time is formulated as the first hitting time:
\begin{equation}
\tau_0 = \inf\{t \geq 0 : h(t) \leq 0\} \label{eq:liquidation_time}
\end{equation}

% ====================================================================
\section{Log-Health Process and Laplace Exponent Analysis}
% ====================================================================

\subsection{Asset Price Dynamics}

The asset prices follow jump-diffusion dynamics with constant-intensity compound Poisson jumps:
\begin{align}
\frac{dX(t)}{X(t^-)} &= \mu_X dt + \sigma_X dB_X(t) + dJ_X(t) \label{eq:X_dynamics}\\
\frac{dY(t)}{Y(t^-)} &= \mu_Y dt + \sigma_Y dB_Y(t) + dJ_Y(t) \label{eq:Y_dynamics}
\end{align}

where:
\begin{itemize}
    \item $B_X(t), B_Y(t)$ are Brownian motions with correlation $\rho$
    \item $J_X(t)$ has constant intensity $\hat\lambda_X$ and i.i.d. jumps $U_X = -(\delta_X + E_X)$, $E_X \sim \mathrm{Exp}(\eta_X)$ representing negative jumps in the long asset
    \item $J_Y(t)$ has constant intensity $\hat\lambda_Y$ and i.i.d. jumps $U_Y = +(\delta_Y + E_Y)$, $E_Y \sim \mathrm{Exp}(\eta_Y)$ representing positive jumps in the short asset
\end{itemize}

This jump structure captures wrong-way risk: downward jumps in collateral and upward jumps in borrowed assets are both adverse to the portfolio.

\subsection{Wealth Process Dynamics}

[Placeholder for wealth process dynamics discussion]

\subsection{Log-Health Process Dynamics}

\begin{proposition}[Log-Health Process with Inherited Jumps]
\label{prop:log_health_dynamics}
Given asset price processes $X(t)$ and $Y(t)$ following the dynamics in equations \eqref{eq:X_dynamics} and \eqref{eq:Y_dynamics}, and position weights satisfying the net exposure constraint $w_X - w_Y = 1$, the log-health process
\begin{equation}
h(t) = \log\left( \frac{b_X w_X X(t)}{w_Y Y(t)} \right)
\end{equation}
evolves as a spectrally negative Lévy process:
\begin{equation}
h(t) = h_0 + \mu_h t + \sigma_h B_h(t) + J_h^X(t) + J_h^Y(t) \label{eq:log_health_full}
\end{equation}
where $J_h^X(t)$ and $J_h^Y(t)$ are compound Poisson processes inherited from the asset jump processes, with:
\begin{align}
J_h^X(t) &= \sum_{i=1}^{N_X(t)} U_X^i, \quad U_X^i = -(\delta_X + E_X^i) \label{eq:jump_X_health}\\
J_h^Y(t) &= \sum_{j=1}^{N_Y(t)} U_Y^j, \quad U_Y^j = +(\delta_Y + E_Y^j) \label{eq:jump_Y_health}
\end{align}
and Brownian motion and drift parameters:
\begin{align}
\mu_h &= \mu_X - \mu_Y - \frac{1}{2}(\sigma_X^2 + \sigma_Y^2 - 2\rho\sigma_X\sigma_Y) \label{eq:drift_h}\\
\sigma_h^2 &= \sigma_X^2 + \sigma_Y^2 - 2\rho\sigma_X\sigma_Y \label{eq:vol_h}\\
dB_h(t) &= \frac{\sigma_X dB_X(t) - \sigma_Y dB_Y(t)}{\sigma_h} \label{eq:brownian_h}
\end{align}
\end{proposition}

\begin{proof}
See \proofref{prop:log_health_dynamics} in Appendix \ref{sec:proofs}.
\end{proof}

\subsection{Laplace Exponent}

\begin{proposition}[Laplace Exponent of Log-Health Process]
\label{prop:laplace_exponent}
For the log-health process $h(t) = h_0 + \mu_h t + \sigma_h B_h(t) + J_h^X(t) + J_h^Y(t)$ from Proposition \ref{prop:log_health_dynamics}, the Laplace exponent is:
\begin{equation}
\psi(\theta) = \frac{1}{2}\sigma_h^2 \theta^2 - \mu_h \theta + \hat\lambda_X \left( e^{-\theta \delta_X} \frac{\eta_X}{\eta_X+\theta} - 1 \right) + \hat\lambda_Y \left( e^{-\theta \delta_Y} \frac{\eta_Y}{\eta_Y+\theta} - 1 \right) \label{eq:laplace_exponent_sec3}
\end{equation}
where the jump terms arise from the shifted exponential distributions:
\begin{align}
U_X^i &= -(\delta_X + E_X^i), \quad E_X^i \sim \mathrm{Exp}(\eta_X)\\
U_Y^j &= +(\delta_Y + E_Y^j), \quad E_Y^j \sim \mathrm{Exp}(\eta_Y)
\end{align}

Furthermore, $\psi(\theta)$ satisfies:
\begin{enumerate}
    \item $\psi(0) = 0$ and $\psi(\theta)$ is convex in $\theta$
    \item $\psi'(0) = -\mu_h - \hat\lambda_X(\delta_X + 1/\eta_X) - \hat\lambda_Y(\delta_Y + 1/\eta_Y)$
    \item The sign of $\psi'(0)$ determines hitting behavior:
    \begin{itemize}
        \item If $\psi'(0) \geq 0$: almost sure hitting ($R = 0$)
        \item If $\psi'(0) < 0$: defective hitting with eventual probability $e^{-Rh_0}$ where $R > 0$ solves $\psi(R) = 0$
    \end{itemize}
\end{enumerate}
\end{proposition}

\begin{proof}
See \proofref{prop:laplace_exponent} in Appendix \ref{sec:proofs}.
\end{proof}

\subsection{Root Function}

The root function $\Phi(q)$ serves as the cornerstone of first-hitting time analysis for spectrally negative Lévy processes. For $q \geq 0$, it is defined as the largest root of $\psi(\theta) = q$:
\begin{equation}
\Phi(q) = \sup\{\theta \geq 0 : \psi(\theta) = q\} \label{eq:root_function_sec3}
\end{equation}

The function satisfies several key properties: $\Phi(0) = R$ corresponds to the adjustment coefficient that determines whether hitting behavior is defective or almost sure, $\Phi(q)$ is strictly increasing and smooth for $q \geq 0$, and $\Phi(q) \to \infty$ as $q \to \infty$ reflecting finite hitting times.

The root function enables the computation of liquidation probabilities $\mathbb{P}(\tau_0 \leq T)$ through Laplace inversion of $e^{-h_0 \Phi(q)}/q$ and facilitates risk-adjusted position sizing through optimal $h_0$ selection that balances expected returns against liquidation costs driven by $\Phi(q)$.

% ====================================================================
\section{First-Hitting Time Analysis}
% ====================================================================

\subsection{First-Hitting Time Problem}

The liquidation event occurs when the log-health process $h(t)$ first reaches zero, corresponding to a health factor $H(t) = e^{h(t)} \leq 1$. This transforms the liquidation risk problem into analyzing the first-hitting time:
\begin{equation}
\tau_0 = \inf\{t \geq 0 : h(t) \leq 0\}
\end{equation}

The distribution of $\tau_0$ determines the timing of liquidation events, which is fundamental for position sizing, risk management, and portfolio optimization in DeFi environments. Unlike traditional finance where positions can be continuously monitored and adjusted, DeFi liquidations occur automatically through smart contracts, making the precise timing distribution critical for risk assessment.

\subsection{Almost Sure vs. Defective Hitting}

The nature of first-hitting behavior fundamentally depends on whether liquidation is inevitable or merely possible, determined by the long-run behavior of the log-health process. When the log-health process has a negative drift that dominates stabilizing effects, liquidation becomes almost sure with $\mathbb{P}(\tau_0 < \infty) = 1$ for all initial health levels $h_0 > 0$. In this regime, the question is not whether liquidation will occur, but when.

Conversely, when favorable drift components can dominate adverse jump effects, hitting becomes defective with $\mathbb{P}(\tau_0 < \infty) = e^{-R h_0} < 1$ for some adjustment coefficient $R > 0$. This regime allows for genuine long-term positions where liquidation is not inevitable. Here, risk management requires conditioning on eventual liquidation:
\begin{equation}
F_{\mathrm{cond}}(T,h_0) = \mathbb{P}(\tau_0 \leq T \mid \tau_0 < \infty) = e^{R h_0} F(T,h_0)
\end{equation}

The distinction between these regimes is crucial for DeFi strategy selection. Almost sure hitting regimes are unsuitable for long-term positions but may support high-frequency strategies. Defective hitting regimes enable traditional investment approaches where positions can potentially be held indefinitely.

\subsection{Laplace Transform of First-Hitting Time}

The mathematical analysis relies on Laplace transform methods. The fundamental result links the root function to liquidation timing:

\begin{theorem}[First-Hitting Time Laplace Transform]
\label{thm:fht_laplace_section3}
For the first hitting time $\tau_0 = \inf\{t \geq 0 : h(t) \leq 0\}$ with $h_0 > 0$:
\begin{equation}
\mathbb{E}_{h_0}[e^{-q \tau_0} \mathbf{1}_{\{\tau_0 < \infty\}}] = e^{-h_0 \Phi(q)}
\end{equation}
\end{theorem}

This enables computation of liquidation probabilities through $\mathbb{P}(\tau_0 \leq T) = \mathcal{L}^{-1}[e^{-h_0 \Phi(q)}/q](T)$, where $\mathcal{L}^{-1}$ denotes numerical Laplace inversion. For defective hitting cases, the conditional distribution uses:
\begin{equation}
\mathcal{L}\{F_{\mathrm{cond}}\}(q) = \frac{e^{-h_0 (\Phi(q+R) - R)}}{q}
\end{equation}

The computational procedures for numerical Laplace inversion are detailed in Appendix \ref{sec:numerical_methods}.


% ====================================================================
\section{Portfolio Sizing}
% ====================================================================

\subsection{Optimization Framework}

We formulate position sizing as maximizing an objective balancing risk and return:
\begin{equation}
\max_{r} \quad \mathcal{U}(r) = \mathbb{E}[\text{return}] - \rho_1 \mathbb{V}[\text{return}] - \rho_2 \mathbb{P}(\tau_0 \leq T^*) \label{eq:objective_portfolio}
\end{equation}

where $r = w_X/(w_X + w_Y)$ is the allocation ratio, $T^*$ is the investment horizon, and $\rho_1, \rho_2 > 0$ are risk aversion parameters.

The initial log-health is:
\begin{equation}
h_0 = \log b_X + \log r + \log(X_0/Y_0) \label{eq:initial_health_portfolio}
\end{equation}

\subsection{Constraints}

The optimization is subject to:
\begin{align}
h_0 &\geq h_{\min} > 0 \quad \text{(minimum health factor)} \\
0 < r &< 1 \quad \text{(position limits)}
\end{align}

% ====================================================================
\section{Optimization Solution Method}
% ====================================================================

We employ gradient-based optimization with:
\begin{enumerate}
    \item Numerical gradients computed via finite differences
    \item Projected gradient ascent ensuring constraint satisfaction  
    \item Line search for optimal step sizes
    \item Convergence based on gradient norm and objective improvement
\end{enumerate}

% ====================================================================
\section{Calibration Methodology}
% ====================================================================

\subsection{Parameter Estimation}

For the constant-intensity model, we estimate:
\begin{itemize}
    \item \textbf{Drift parameters}: $\mu_X, \mu_Y$ from sample means of log-returns
    \item \textbf{Diffusion parameters}: $\sigma_X, \sigma_Y, \rho$ from sample covariance
    \item \textbf{Jump intensities}: $\hat\lambda_X, \hat\lambda_Y$ from jump counting
    \item \textbf{Jump size parameters}: $\delta_X, \eta_X, \delta_Y, \eta_Y$ via maximum likelihood
\end{itemize}

\subsection{Jump Detection}

We implement peak-over-threshold (POT) methods:
\begin{enumerate}
    \item Select thresholds to minimize skewness and excess kurtosis of residual returns
    \item Count exceedances to estimate jump intensities
    \item Fit shifted exponential distributions to jump sizes
\end{enumerate}

\subsection{Model Validation}

Validation includes:
\begin{itemize}
    \item Residual analysis of filtered diffusion components
    \item Out-of-sample performance of liquidation probability forecasts
    \item Comparison with simpler geometric Brownian motion models
\end{itemize}

% ====================================================================
\section{Extensions and Limitations}
% ====================================================================

\subsection{Hawkes Process Comparison}

While our constant-intensity approach sacrifices the ability to model jump clustering, it provides several advantages:
\begin{itemize}
    \item \textbf{Analytical tractability}: Explicit Laplace transforms vs. complex Riccati systems
    \item \textbf{Numerical stability}: Robust inversion algorithms vs. 12-dimensional ODE solving
    \item \textbf{Computational efficiency}: Fast evaluation suitable for real-time applications
    \item \textbf{Parameter parsimony}: Fewer parameters reduce overfitting risk
\end{itemize}

The trade-off is loss of dynamic clustering effects where jump events increase the probability of subsequent jumps.

\subsection{Future Extensions}

Potential extensions include:
\begin{enumerate}
    \item \textbf{Regime-switching intensities}: Markov-modulated jump processes
    \item \textbf{Stochastic volatility}: Time-varying diffusion parameters  
    \item \textbf{Multi-asset portfolios}: Extension beyond pairwise long-short positions
    \item \textbf{Dynamic hedging}: Optimal rebalancing under transaction costs
\end{enumerate}

% ====================================================================
\section{Conclusion}
% ====================================================================

We have developed a practical framework for first-hitting time analysis in DeFi long-short positioning using constant-intensity jump-diffusion models. While simpler than Hawkes processes, this approach provides robust semi-analytical solutions suitable for real-time risk management.

Key contributions include:
\begin{enumerate}
    \item Spectrally negative Lévy process formulation for log-health dynamics
    \item Semi-analytical computation via Laplace transforms and Gaver-Stehfest inversion
    \item Practical optimization framework balancing returns and liquidation risk
    \item Straightforward calibration methodology for constant-intensity parameters
\end{enumerate}

The framework demonstrates that sophisticated risk management in DeFi applications need not require computationally intensive models. The constant-intensity approach provides sufficient complexity to capture wrong-way risk while maintaining the analytical tractability necessary for practical implementation.

% ====================================================================
% APPENDIX
% ====================================================================

\appendix

\section{Mathematical Proofs}
\label{sec:proofs}

\begin{delayedproof}{prop:log_health_dynamics}
We prove this by applying Itô's lemma to the log-health function and carefully tracking jump contributions from both assets.

\textbf{Step 1: Setup and Itô decomposition}

Starting from $h(t) = \log(b_X w_X) + \log X(t) - \log(w_Y) - \log Y(t)$, define $f(x,y) = \log x - \log y$ so that $h(t) = \text{const} + f(X(t), Y(t))$.

Applying Itô's lemma to $f(X(t), Y(t))$ with jump-diffusion processes:
\begin{align}
df(X(t), Y(t)) &= \frac{\partial f}{\partial x} dX(t) + \frac{\partial f}{\partial y} dY(t) + \frac{1}{2}\frac{\partial^2 f}{\partial x^2} d[X]_t^c + \frac{1}{2}\frac{\partial^2 f}{\partial y^2} d[Y]_t^c \nonumber\\
&\quad + \frac{\partial^2 f}{\partial x \partial y} d[X,Y]_t^c + \sum_{\text{jumps}} \Delta f
\end{align}

\textbf{Step 2: Compute partial derivatives}

For $f(x,y) = \log x - \log y$:
\begin{align}
\frac{\partial f}{\partial x} &= \frac{1}{x}, \quad \frac{\partial f}{\partial y} = -\frac{1}{y}\\
\frac{\partial^2 f}{\partial x^2} &= -\frac{1}{x^2}, \quad \frac{\partial^2 f}{\partial y^2} = \frac{1}{y^2}, \quad \frac{\partial^2 f}{\partial x \partial y} = 0
\end{align}

\textbf{Step 3: Continuous part analysis}

The continuous parts of $X(t)$ and $Y(t)$ satisfy:
\begin{align}
dX(t) &= X(t^-)[\mu_X dt + \sigma_X dB_X(t)] + X(t^-)\Delta J_X(t)\\
dY(t) &= Y(t^-)[\mu_Y dt + \sigma_Y dB_Y(t)] + Y(t^-)\Delta J_Y(t)
\end{align}

For the continuous part (ignoring jumps temporarily):
\begin{align}
df^c(X(t), Y(t)) &= \frac{1}{X(t)}[\mu_X X(t) dt + \sigma_X X(t) dB_X(t)] - \frac{1}{Y(t)}[\mu_Y Y(t) dt + \sigma_Y Y(t) dB_Y(t)] \nonumber\\
&\quad - \frac{1}{2X(t)^2} \sigma_X^2 X(t)^2 dt + \frac{1}{2Y(t)^2} \sigma_Y^2 Y(t)^2 dt
\end{align}

The correct continuous part, accounting for correlation $\rho$, is:
\begin{align}
df^c(X(t), Y(t)) &= [\mu_X - \mu_Y - \frac{1}{2}(\sigma_X^2 + \sigma_Y^2 - 2\rho\sigma_X\sigma_Y)] dt \nonumber\\
&\quad + \sigma_X dB_X(t) - \sigma_Y dB_Y(t)
\end{align}

\textbf{Step 4: Jump part analysis}

When $X(t)$ jumps at time $T_i$ with size $\Delta J_X(T_i) = U_X^i = -(\delta_X + E_X^i)$:
\begin{align}
\Delta f|_{X\text{-jump}} &= \log(X(T_i^-)(1 + U_X^i)) - \log(X(T_i^-)) = \log(1 + U_X^i) = U_X^i
\end{align}

When $Y(t)$ jumps at time $T_j$ with $\Delta J_Y(T_j) = U_Y^j = +(\delta_Y + E_Y^j)$:
\begin{align}
\Delta f|_{Y\text{-jump}} &= -\log(1 + U_Y^j) = -U_Y^j
\end{align}

\textbf{Step 5: Combine results}

The total jump contribution to $h(t)$ is:
\begin{align}
J_h^X(t) &= \sum_{i=1}^{N_X(t)} U_X^i = \sum_{i=1}^{N_X(t)} [-(\delta_X + E_X^i)]\\
J_h^Y(t) &= \sum_{j=1}^{N_Y(t)} (-U_Y^j) = \sum_{j=1}^{N_Y(t)} [+(\delta_Y + E_Y^j)]
\end{align}

The diffusion coefficient becomes:
\[
\sigma_h^2 = \mathbb{V}[\sigma_X dB_X(t) - \sigma_Y dB_Y(t)] = \sigma_X^2 + \sigma_Y^2 - 2\rho\sigma_X\sigma_Y
\]

Therefore, $h(t) = h_0 + \mu_h t + \sigma_h B_h(t) + J_h^X(t) + J_h^Y(t)$ where all parameters are as stated.
\end{delayedproof}

\begin{delayedproof}{prop:laplace_exponent}
We derive the Laplace exponent by computing the cumulant generating function for each component of $h(t)$.

\textbf{Step 1: Decompose the log-health process}

From Proposition \ref{prop:log_health_dynamics}:
\[
h(t) = h_0 + \mu_h t + \sigma_h B_h(t) + J_h^X(t) + J_h^Y(t)
\]

\textbf{Step 2: Brownian motion contribution}

For $\sigma_h B_h(t)$, the contribution is:
\[
\psi_B(\theta) = \frac{1}{2}\sigma_h^2 \theta^2
\]

\textbf{Step 3: Drift contribution}

The drift contributes $-\mu_h \theta$ (negative sign for spectrally negative processes).

\textbf{Step 4: X-jump contribution}

For $J_h^X(t)$ with intensity $\hat\lambda_X$ and jump sizes $U_X^i = -(\delta_X + E_X^i)$:
\[
\psi_X(\theta) = \hat\lambda_X (\mathbb{E}[e^{\theta U_X}] - 1)
\]

Computing $\mathbb{E}[e^{\theta U_X}]$ where $U_X = -(\delta_X + E_X)$:
\begin{align}
\mathbb{E}[e^{\theta U_X}] &= e^{-\theta \delta_X} \mathbb{E}[e^{-\theta E_X}] = e^{-\theta \delta_X} \frac{\eta_X}{\eta_X + \theta}
\end{align}

\textbf{Step 5: Y-jump contribution}

For $J_h^Y(t)$, Y-jumps contribute negatively to health (they worsen it), so:
\[
\mathbb{E}[e^{\theta(-U_Y)}] = e^{-\theta \delta_Y} \frac{\eta_Y}{\eta_Y + \theta}
\]

\textbf{Step 6: Combine all contributions}

\[
\psi(\theta) = \frac{1}{2}\sigma_h^2 \theta^2 - \mu_h \theta + \hat\lambda_X \left( e^{-\theta \delta_X} \frac{\eta_X}{\eta_X+\theta} - 1 \right) + \hat\lambda_Y \left( e^{-\theta \delta_Y} \frac{\eta_Y}{\eta_Y+\theta} - 1 \right)
\]

The properties follow from direct computation: $\psi(0) = 0$, and 
\[
\psi'(0) = -\mu_h - \hat\lambda_X(\delta_X + 1/\eta_X) - \hat\lambda_Y(\delta_Y + 1/\eta_Y)
\]
\end{delayedproof}

\section{Numerical Methods and Implementation}
\label{sec:numerical_methods}

This appendix details the computational procedures for evaluating first-hitting time distributions through Laplace transform inversion.

\subsection{Gaver-Stehfest Inversion Algorithm}

The Gaver-Stehfest algorithm provides a robust method for numerically inverting Laplace transforms. For the first-hitting time distribution:

\begin{algorithm}
\caption{First-Hitting Time CDF via Gaver-Stehfest Inversion}
\label{alg:gaver_stehfest_appendix}
\begin{algorithmic}[1]
\STATE \textbf{Input:} Initial log-health $h_0 > 0$, evaluation time $T > 0$, Stehfest order $N$ (typically 10-14)
\STATE Compute adjustment coefficient $R = \Phi(0)$ by solving $\psi(R) = 0$
\STATE Precompute Stehfest weights $V_k$ for $k = 1, \ldots, N$:
\[
V_k = (-1)^{N/2+k} \sum_{j=\lfloor(k+1)/2\rfloor}^{\min(k,N/2)} \frac{j^{N/2} (2j)!}{(N/2-j)! j! (j-1)! (k-j)! (2j-k)!}
\]
\FOR{$k = 1$ to $N$}
    \STATE Compute $q_k = k \ln 2 / T$
    \STATE Solve $\psi(\theta) = q_k$ numerically to obtain $\Phi(q_k)$
    \STATE Evaluate $\mathcal{L}\{F\}(q_k) = e^{-h_0 \Phi(q_k)} / q_k$
\ENDFOR
\STATE Compute the inversion:
\[
F(T,h_0) = \frac{\ln 2}{T} \sum_{k=1}^{N} V_k \mathcal{L}\{F\}(q_k)
\]
\STATE Clamp result to $[0,1]$
\STATE \textbf{Output:} $F(T,h_0) = \mathbb{P}(\tau_0 \leq T)$
\end{algorithmic}
\end{algorithm}

For the conditional distribution in defective cases, replace step 8 with:
\[
\mathcal{L}\{F_{\mathrm{cond}}\}(q_k) = \frac{e^{-h_0 (\Phi(q_k+R) - R)}}{q_k}
\]

\subsection{Root Finding for $\Phi(q)$}

The root function $\Phi(q)$ requires solving $\psi(\theta) = q$ for each $q_k$ in the Stehfest algorithm. We employ Brent's method with search interval $[0, \theta_{\max}]$ where $\theta_{\max}$ is chosen to ensure $\psi(\theta_{\max}) > q_{\max}$.

\textbf{Implementation details:}
\begin{itemize}
    \item Initialize with bracketing: $\theta_{\min} = 0$, $\theta_{\max} = 10 \max(\eta_X, \eta_Y)$
    \item Verify $\psi(0) < q < \psi(\theta_{\max})$ before root finding
    \item Use relative tolerance $10^{-12}$ for convergence
    \item Cache $\psi(\theta)$ evaluations to avoid recomputation
\end{itemize}

\subsection{Numerical Stability Considerations}

\begin{itemize}
    \item \textbf{Stehfest order}: Use moderate orders (10-14) to balance accuracy and stability. Higher orders may amplify numerical errors.
    \item \textbf{High precision}: Employ extended precision arithmetic for large $N$ or small $T$ to avoid catastrophic cancellation.
    \item \textbf{Alternative inversion}: For very small $T$ or large $h_0$, consider Talbot or de Hoog methods which may be more stable.
    \item \textbf{Parameter validation}: Ensure $\eta_X + \theta > 0$ and $\eta_Y + \theta > 0$ to avoid singularities.
    \item \textbf{Clamping}: Final probabilities should be clamped to $[0,1]$ to handle numerical errors.
    \item \textbf{Conditioning detection}: Automatically detect defective hitting by checking if $\psi'(0) < 0$.
\end{itemize}

\subsection{Computational Complexity}

For each distribution evaluation:
\begin{itemize}
    \item $O(N)$ Laplace transform evaluations (typically $N = 10-14$)
    \item Each evaluation requires solving $\psi(\theta) = q_k$ via root finding
    \item Root finding: $O(\log \epsilon^{-1})$ iterations for tolerance $\epsilon$
    \item Total complexity: $O(N \log \epsilon^{-1})$ per distribution evaluation
\end{itemize}

% ====================================================================
% BIBLIOGRAPHY
% ====================================================================

\bibliographystyle{abbrvnamed}
\bibliography{finance}

\end{document}