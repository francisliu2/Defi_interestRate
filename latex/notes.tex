\documentclass{article}
\usepackage[margin=1in]{geometry}

% ==== PACKAGES ====
\usepackage{amsmath, amssymb, amsthm}
\usepackage{mathtools}
\usepackage[longnamesfirst,sort&compress]{natbib}
\usepackage{graphicx}
\usepackage{algorithm, algorithmic}
\usepackage{hyperref}
\usepackage{cleveref}
\usepackage{booktabs}
\usepackage{enumitem}

% Citation punctuation style matching original
\bibpunct{(}{)}{;}{a}{,}{,}

% ==== CUSTOM COMMANDS ====
\newcommand{\E}{\mathbb{E}}

\renewcommand{\P}{\mathbb{P}}
\newcommand{\R}{\mathbb{R}}
\newcommand{\diff}{\mathrm{d}}
\newcommand{\pard}[2]{\frac{\partial #1}{\partial #2}}

% Theorem environments
\theoremstyle{definition}
\newtheorem{definition}{Definition}[section]
\newtheorem{theorem}{Theorem}[section]
\newtheorem{proposition}{Proposition}[section]
\newtheorem{lemma}{Lemma}[section]
\newtheorem{remark}{Remark}[section]

% Proof environment for appendix
\newenvironment{delayedproof}[1]
 {\begin{proof}[\raisedtarget{#1}Proof of \Cref{#1}]}
 {\end{proof}}
\newcommand{\raisedtarget}[1]{%
  \raisebox{\fontcharht\font`P}[0pt][0pt]{\hypertarget{#1}{}}%
}
\newcommand{\proofref}[1]{\hyperlink{#1}{proof}}

\title{First-Hitting Time Analysis for Log-Health Process with Constant Jump Intensities}
\author{[Author Names]}
\date{\today}

\begin{document}
\maketitle

\begin{abstract}
We develop an optimal allocation framework for long-short cryptocurrency positions on decentralized finance (DeFi) lending platforms by analyzing first-hitting time distributions for log-health processes under constant-intensity jump-diffusion dynamics. Moving beyond the complexity of Hawkes processes, we employ a spectrally negative Lévy process with constant-intensity compound Poisson jumps to model wrong-way risk in cryptocurrency markets. The approach provides semi-analytical solutions via Laplace transform methods and Gaver-Stehfest inversion, enabling practical computation of margin call probabilities. Our framework balances expected returns against liquidation risk through optimization of position weights subject to collateral constraints.

\textbf{Keywords:} DeFi, long-short strategies, liquidation risk, Lévy processes, first passage times, cryptocurrency
\end{abstract}

% ====================================================================
\section{Introduction}
% ====================================================================

Decentralized finance (DeFi) lending platforms such as AAVE and Compound have emerged as powerful tools for implementing sophisticated long-short cryptocurrency strategies. With combined total value locked exceeding \$10 billion, these protocols enable traders to simultaneously hold long positions in promising assets while borrowing and shorting others, all while earning interest on collateralized deposits. This creates unique opportunities for alpha generation that are unavailable through traditional centralized exchanges.

The core challenge in DeFi long-short positioning lies in managing liquidation risk under extreme market volatility. Unlike traditional margin trading, DeFi platforms require significant overcollateralization (typically 120-150\%) and impose automatic liquidation when health factors fall below unity. Given the notorious volatility of cryptocurrency markets, where daily moves of 10-20\% are common and flash crashes can exceed 50\%, the probability and timing of liquidation events become critical factors in position sizing and risk management.

Wrong-way risk represents the most significant threat to DeFi portfolios: adverse price movements in either the collateral asset (downward) or borrowed asset (upward) directly deteriorate the health factor, with large jumps potentially triggering immediate liquidation. Traditional portfolio theory, which assumes smooth price evolution and continuous rebalancing opportunities, fails to capture the discrete, path-dependent nature of liquidation risk in DeFi environments.

The academic literature on first-hitting times for jump-diffusion processes suggests that Hawkes processes, with their ability to model clustered jumps and cross-excitation between assets, would be ideal for capturing wrong-way risk dynamics. However, our initial investigations revealed that Hawkes-based models, while theoretically appealing, present severe computational challenges. The resulting characteristic function analysis requires solving high-dimensional Riccati systems that are numerically unstable and computationally intensive, making them impractical for real-time risk management applications.

This paper advocates for a more pragmatic approach: modeling the log-health process as a spectrally negative Lévy process with constant jump intensities. While this approach cannot capture the dynamic clustering effects of Hawkes processes, it offers substantial practical advantages:

\begin{enumerate}
    \item \textbf{Analytical tractability}: Explicit Laplace transform representations for first-hitting time distributions
    \item \textbf{Numerical stability}: Robust inversion algorithms that converge reliably across parameter ranges
    \item \textbf{Computational efficiency}: Fast evaluation suitable for real-time portfolio optimization
    \item \textbf{Parameter parsimony}: Fewer parameters reduce calibration complexity and overfitting risk
    \item \textbf{Practical implementation}: Semi-analytical solutions enable deployment in production trading systems
\end{enumerate}

Our framework demonstrates that effective liquidation risk management in DeFi applications need not require the most sophisticated stochastic models. The constant-intensity jump-diffusion approach captures the essential wrong-way risk characteristics while maintaining the computational tractability necessary for practical portfolio management. This represents a conscious trade-off between theoretical completeness and practical utility, prioritizing robust, implementable solutions over academic elegance.

% ====================================================================
\section{Portfolio Setup and Log-Health Process}
% ====================================================================

\subsection{Long-Short Position Construction}

Consider a long-short position on a DeFi lending platform where a user deposits collateral asset $X$ and borrows asset $Y$. The position consists of:
\begin{itemize}
    \item \textbf{Long position}: $w_X$ units of asset $X(t)$ deposited as collateral
    \item \textbf{Short position}: $w_Y$ units of asset $Y(t)$ borrowed for short selling
\end{itemize}

We implement a \textbf{net exposure constraint}:
\begin{equation}
w_X - w_Y = 1, \quad w_X > 0, \quad w_Y > 0 \label{eq:net_exposure}
\end{equation}

This ensures unit net exposure while allowing flexible allocation between long and short positions based on relative expected returns and risk characteristics.

\subsection{Health Factor and Liquidation Mechanics}

The \textbf{log-health process} serves as the primary risk metric:
\begin{equation}
h(t) = \log\left( \frac{b_X w_X X(t)}{w_Y Y(t)} \right) \label{eq:log_health}
\end{equation}

where $b_X \in (0,1]$ is the collateral factor (haircut) applied to the long position. A margin call is triggered when $h(t)$ first hits zero, corresponding to the health factor $H(t) = e^{h(t)} \leq 1$.

The liquidation time is formulated as the first hitting time:
\begin{equation}
\tau_0 = \inf\{t \geq 0 : h(t) \leq 0\} \label{eq:liquidation_time}
\end{equation}

% ====================================================================
\section{Log-Health Process and Laplace Exponent Analysis}
% ====================================================================

\subsection{Asset Price Dynamics}

The asset prices follow jump-diffusion dynamics with constant-intensity compound Poisson jumps:
\begin{align}
\frac{dX(t)}{X(t^-)} &= \mu_X dt + \sigma_X dB_X(t) + dJ_X(t) \label{eq:X_dynamics}\\
\frac{dY(t)}{Y(t^-)} &= \mu_Y dt + \sigma_Y dB_Y(t) + dJ_Y(t) \label{eq:Y_dynamics}
\end{align}

where:
\begin{itemize}
    \item $B_X(t), B_Y(t)$ are Brownian motions with correlation $\rho$
    \item $J_X(t)$ has constant intensity $\hat\lambda_X$ and i.i.d. jumps $U_X = -(\delta_X + E_X)$, $E_X \sim \mathrm{Exp}(\eta_X)$ representing negative jumps in the long asset
    \item $J_Y(t)$ has constant intensity $\hat\lambda_Y$ and i.i.d. jumps $U_Y = +(\delta_Y + E_Y)$, $E_Y \sim \mathrm{Exp}(\eta_Y)$ representing positive jumps in the short asset
\end{itemize}

This jump structure captures wrong-way risk: downward jumps in collateral and upward jumps in borrowed assets are both adverse to the portfolio.

\subsection{Wealth Process Dynamics}

The wealth process dynamics are:
\begin{equation}
\frac{dV(t)}{V(t^{-})} = w_X \frac{dX(t)}{X(t^-)}  - w_Y \frac{dY(t)}{Y(t^-)} 
\end{equation}
Using equations 4 and 5 and grouping the terms based on their types:
\begin{equation}
\frac{dV(t)}{V(t^{-})}= (w_X \mu_X - w_Y \mu_y) dt + w_X \sigma_X dB_X(t) - w_Y \sigma_Y dB_Y(t) + w_X dJ_X(t) - w_Y dJ_Y(t)
\end{equation}
The solution to the SDE will then be:
\begin{equation}
V(T) = V(0) e^{(\psi T + \sigma_W W(T))} \prod_{i=1}^{N_X (T)} (1 + w_X U_X^i) \prod_{j=1}^{N_Y (T)} (1 - w_Y U_Y^j)
\end{equation}
Where the drift adjustment term $\psi$ is equal to:
\begin{equation}
\psi = w_X \mu_X - w_Y \mu_Y - \frac{1}{2} ( {w^2_X} {\sigma_X}^2 + {w^2_Y} {\sigma_Y}^2 - 2w_X w_Y \rho \sigma_X \sigma_Y ) 
+ \lambda_X w_X \mathbb{E} [U_X] - \lambda_Y w_Y \mathbb{E} [U_Y]
\end{equation}
The first moment of the wealth process is:
\begin{equation}
\mathbb{E}[V(T)] = V(0) e^{\Theta T}
\end{equation}
With the growth rate $\Theta$:
\begin{equation}
\Theta = w_X \mu_X - w_Y \mu_Y + \lambda_X w_X \mathbb{E} [U_X] - \lambda_Y w_Y \mathbb{E} [U_Y]
\end{equation}
Now, the second moment of the wealth process is:
\begin{equation}
\text{Var}(V(T)) = \mathbb{E}[V(T)^2] - (\mathbb{E}[V(T)])^2 = V(0)^2 e^{2 \Theta T} (e^{\xi T} - 1)
\end{equation}
Where the excess variance rate $\xi$ represents:
\begin{equation}
\xi = w^2_X \sigma^2_X  + w^2_Y \sigma^2_Y - 2w_Xw_Y \rho \sigma_X \rho \sigma_Y +
\lambda_X w^2_X \mathbb{E}[U^2_X] + \lambda_Y w^2_Y \mathbb{E}[U^2_Y]
\end{equation}
The second moment of the jumps is given by:
\begin{equation}
\mathbb{E}[U^2_X] = \frac{1}{\eta^2_X} + \left( \delta_X + \frac{1}{\eta_X} \right)
\end{equation}
\begin{equation}
\mathbb{E}[U^2_Y] = \frac{1}{\eta^2_Y} + \left( \delta_Y + \frac{1}{\eta_Y} \right)
\end{equation}

\subsection{Log-Health Process Dynamics}

\begin{proposition}[Log-Health Process with Inherited Jumps]
\label{prop:log_health_dynamics}
Given the asset price processes $X(t)$ and $Y(t)$ follow the dynamics in equations \eqref{eq:X_dynamics} and \eqref{eq:Y_dynamics}, and position weights satisfying the net exposure constraint $w_X - w_Y = 1$, the log-health process
\begin{equation}
h(t) = \log\left( \frac{b_X w_X X(t)}{w_Y Y(t)} \right)
\end{equation}
evolves as a spectrally negative Lévy process:
\begin{equation}
h(t) = h_0 + \mu_h t + \sigma_h B_h(t) + J_h^X(t) + J_h^Y(t) \label{eq:log_health_full}
\end{equation}
where $J_h^X(t)$ and $J_h^Y(t)$ are compound Poisson processes inherited from the asset jump processes, with:
\begin{align}
J_h^X(t) &= \sum_{i=1}^{N_X(t)} U_X^i, \quad U_X^i = -(\delta_X + E_X^i) \label{eq:jump_X_health}\\
J_h^Y(t) &= \sum_{j=1}^{N_Y(t)} U_Y^j, \quad U_Y^j = +(\delta_Y + E_Y^j) \label{eq:jump_Y_health}
\end{align}
and Brownian motion and drift parameters:
\begin{align}
\mu_h &= \mu_X - \mu_Y - \frac{1}{2}(\sigma_X^2 + \sigma_Y^2 - 2\rho\sigma_X\sigma_Y) \label{eq:drift_h}\\
\sigma_h^2 &= \sigma_X^2 + \sigma_Y^2 - 2\rho\sigma_X\sigma_Y \label{eq:vol_h}\\
dB_h(t) &= \frac{\sigma_X dB_X(t) - \sigma_Y dB_Y(t)}{\sigma_h} \label{eq:brownian_h}
\end{align}
\end{proposition}

\begin{proof}
See \proofref{prop:log_health_dynamics} in Appendix \ref{sec:proofs}.
\end{proof}

\subsection{Laplace Exponent}

\begin{proposition}[Laplace Exponent of Log-Health Process]
\label{prop:laplace_exponent}
For the log-health process $h(t) = h_0 + \mu_h t + \sigma_h B_h(t) + J_h^X(t) + J_h^Y(t)$ from Proposition \ref{prop:log_health_dynamics}, the Laplace exponent is:
\begin{equation}
\psi(\theta) = \frac{1}{2}\sigma_h^2 \theta^2 - \mu_h \theta + \hat\lambda_X \left( e^{-\theta \delta_X} \frac{\eta_X}{\eta_X+\theta} - 1 \right) + \hat\lambda_Y \left( e^{-\theta \delta_Y} \frac{\eta_Y}{\eta_Y+\theta} - 1 \right) \label{eq:laplace_exponent_sec3}
\end{equation}
where the jump terms arise from the shifted exponential distributions:
\begin{align}
U_X^i &= -(\delta_X + E_X^i), \quad E_X^i \sim \mathrm{Exp}(\eta_X)\\
U_Y^j &= +(\delta_Y + E_Y^j), \quad E_Y^j \sim \mathrm{Exp}(\eta_Y)
\end{align}

Furthermore, $\psi(\theta)$ satisfies:
\begin{enumerate}
    \item $\psi(0) = 0$ and $\psi(\theta)$ is convex in $\theta$
    \item $\psi'(0) = -\mu_h - \hat\lambda_X(\delta_X + 1/\eta_X) - \hat\lambda_Y(\delta_Y + 1/\eta_Y)$
    \item The sign of $\psi'(0)$ determines hitting behavior:
    \begin{itemize}
        \item If $\psi'(0) \geq 0$: almost sure hitting ($R = 0$)
        \item If $\psi'(0) < 0$: defective hitting with eventual probability $e^{-Rh_0}$ where $R > 0$ solves $\psi(R) = 0$
    \end{itemize}
\end{enumerate}
\end{proposition}

\begin{proof}
See \proofref{prop:laplace_exponent} in Appendix \ref{sec:proofs}.
\end{proof}

\subsection{Root Function}

The root function $\Phi(q)$ serves as the cornerstone of first-hitting time analysis for spectrally negative Lévy processes. For $q \geq 0$, it is defined as the largest root of $\psi(\theta) = q$:
\begin{equation}
\Phi(q) = \sup\{\theta \geq 0 : \psi(\theta) = q\} \label{eq:root_function_sec3}
\end{equation}

The function satisfies several key properties: $\Phi(0) = R$ corresponds to the adjustment coefficient that determines whether hitting behavior is defective or almost sure, $\Phi(q)$ is strictly increasing and smooth for $q \geq 0$, and $\Phi(q) \to \infty$ as $q \to \infty$ reflecting finite hitting times.

The root function enables the computation of liquidation probabilities $\mathbb{P}(\tau_0 \leq T)$ through Laplace inversion of $e^{-h_0 \Phi(q)}/q$ and facilitates risk-adjusted position sizing through optimal $h_0$ selection that balances expected returns against liquidation costs driven by $\Phi(q)$.

% ====================================================================
\section{First-Hitting Time Analysis}
% ====================================================================

\subsection{First-Hitting Time Problem}

The liquidation event occurs when the log-health process $h(t)$ first reaches zero, corresponding to a health factor $H(t) = e^{h(t)} \leq 1$. This transforms the liquidation risk problem into analyzing the first-hitting time:
\begin{equation}
\tau_0 = \inf\{t \geq 0 : h(t) \leq 0\}
\end{equation}

The distribution of $\tau_0$ determines the timing of liquidation events, which is fundamental for position sizing, risk management, and portfolio optimization in DeFi environments. Unlike traditional finance where positions can be continuously monitored and adjusted, DeFi liquidations occur automatically through smart contracts, making the precise timing distribution critical for risk assessment.

\subsection{Almost Sure vs. Defective Hitting}

The nature of first-hitting behavior fundamentally depends on whether liquidation is inevitable or merely possible, determined by the long-run behavior of the log-health process. When the log-health process has a negative drift that dominates stabilizing effects, liquidation becomes almost sure with $\mathbb{P}(\tau_0 < \infty) = 1$ for all initial health levels $h_0 > 0$. In this regime, the question is not whether liquidation will occur, but when.

Conversely, when favorable drift components can dominate adverse jump effects, hitting becomes defective with $\mathbb{P}(\tau_0 < \infty) = e^{-R h_0} < 1$ for some adjustment coefficient $R > 0$. This regime allows for genuine long-term positions where liquidation is not inevitable. Here, risk management requires conditioning on eventual liquidation:
\begin{equation}
F_{\mathrm{cond}}(T,h_0) = \mathbb{P}(\tau_0 \leq T \mid \tau_0 < \infty) = e^{R h_0} F(T,h_0)
\end{equation}

The distinction between these regimes is crucial for DeFi strategy selection. Almost sure hitting regimes are unsuitable for long-term positions but may support high-frequency strategies. Defective hitting regimes enable traditional investment approaches where positions can potentially be held indefinitely.

\subsection{Laplace Transform of First-Hitting Time}

The mathematical analysis relies on Laplace transform methods. The fundamental result links the root function to liquidation timing:

\begin{theorem}[First-Hitting Time Laplace Transform]
\label{thm:fht_laplace_section3}
For the first hitting time $\tau_0 = \inf\{t \geq 0 : h(t) \leq 0\}$ with $h_0 > 0$:
\begin{equation}
\mathbb{E}_{h_0}[e^{-q \tau_0} \mathbf{1}_{\{\tau_0 < \infty\}}] = e^{-h_0 \Phi(q)}
\end{equation}
\end{theorem}
This enables the computation of liquidation probabilities through the following: 
\begin{equation}
\mathbb{P}(\tau_0 \leq T) = \mathcal{L}^{-1}\left[\frac {e^{-h_0 \Phi(q)}}{q}\right](T)    
\end{equation} where $\mathcal{L}^{-1}$ denotes numerical Laplace inversion. For defective hitting cases, the conditional distribution uses:
\begin{equation}
\mathcal{L}\{F_{\mathrm{cond}}\}(q) = \frac{e^{-h_0 (\Phi(q+R) - R)}}{q}
\end{equation}

The computational procedures for numerical Laplace inversion are detailed in Appendix \ref{sec:numerical_methods}.


% ====================================================================
\section{Portfolio Sizing}
% ====================================================================

\subsection{Optimization Framework}

We formulate position sizing as maximizing an objective balancing risk and return:
\begin{equation}
\max_{r} \quad \mathcal{U}(r) = \mathbb{E}[\text{return}] - \rho_1 \mathbb{V}[\text{return}] - \rho_2 \mathbb{P}(\tau_0 \leq T^*) \label{eq:objective_portfolio}
\end{equation}

where $r = w_X/(w_X + w_Y)$ is the allocation ratio, $T^*$ is the investment horizon, and $\rho_1, \rho_2 > 0$ are risk aversion parameters.

The initial log-health is:
\begin{equation}
h_0 = \log b_X + \log r + \log(X_0/Y_0) \label{eq:initial_health_portfolio}
\end{equation}

\subsection{Constraints}

The optimization is subject to:
\begin{align}
h_0 &\geq h_{\min} > 0 \quad \text{(minimum health factor)} \\
0 < r &< 1 \quad \text{(position limits)}
\end{align}

% ====================================================================
\section{Optimization Solution Method}
% ====================================================================

We employ gradient-based optimization with:
\begin{enumerate}
    \item Numerical gradients computed via finite differences
    \item Projected gradient ascent ensuring constraint satisfaction  
    \item Line search for optimal step sizes
    \item Convergence based on gradient norm and objective improvement
\end{enumerate}

% ====================================================================
\section{Calibration Methodology}
% ====================================================================

\subsection{Parameter Estimation}

For the constant-intensity model, we estimate:
\begin{itemize}
    \item \textbf{Drift parameters}: $\mu_X, \mu_Y$ from sample means of log-returns
    \item \textbf{Diffusion parameters}: $\sigma_X, \sigma_Y, \rho$ from sample covariance
    \item \textbf{Jump intensities}: $\hat\lambda_X, \hat\lambda_Y$ from jump counting
    \item \textbf{Jump size parameters}: $\delta_X, \eta_X, \delta_Y, \eta_Y$ via maximum likelihood
\end{itemize}

\subsection{Jump Detection}

We implement peak-over-threshold (POT) methods:
\begin{enumerate}
    \item Select thresholds to minimize skewness and excess kurtosis of residual returns
    \item Count exceedances to estimate jump intensities
    \item Fit shifted exponential distributions to jump sizes
\end{enumerate}

\subsection{Model Validation}

Validation includes:
\begin{itemize}
    \item Residual analysis of filtered diffusion components
    \item Out-of-sample performance of liquidation probability forecasts
    \item Comparison with simpler geometric Brownian motion models
\end{itemize}

% ====================================================================
\section{Extensions and Limitations}
% ====================================================================

\subsection{Hawkes Process Comparison}

While our constant-intensity approach sacrifices the ability to model jump clustering, it provides several advantages:
\begin{itemize}
    \item \textbf{Analytical tractability}: Explicit Laplace transforms vs. complex Riccati systems
    \item \textbf{Numerical stability}: Robust inversion algorithms vs. 12-dimensional ODE solving
    \item \textbf{Computational efficiency}: Fast evaluation suitable for real-time applications
    \item \textbf{Parameter parsimony}: Fewer parameters reduce overfitting risk
\end{itemize}

The trade-off is loss of dynamic clustering effects where jump events increase the probability of subsequent jumps.

\subsection{Future Extensions}

Potential extensions include:
\begin{enumerate}
    \item \textbf{Regime-switching intensities}: Markov-modulated jump processes
    \item \textbf{Stochastic volatility}: Time-varying diffusion parameters  
    \item \textbf{Multi-asset portfolios}: Extension beyond pairwise long-short positions
    \item \textbf{Dynamic hedging}: Optimal rebalancing under transaction costs
\end{enumerate}

% ====================================================================
\section{Use Cases-Applied Strategies}
% ====================================================================
In order to show the practical implications of our findings, we present three examples of long-short strategies and the empirical results.

\subsection{Long Stablecoin-Short Volatile }
For the first use case, we consider a long-short strategy in which a stablecoin is chosen for the long position while a volatile cryptocurrency is borrowed to establish a short position. The economic incentive is based on the following ideas: 
\begin{itemize}
    \item the stablecoin is a relatively safe investment that earns a deterministic supply rate $r_X$, and
    \item the short side is expected to appreciate in value, should the borrowed coin depreciate.
\end{itemize}
The net annualized carry is equal to the spread between supply and borrow rates, scaled by position weights:
\begin{equation}
 R(w_X, w_Y) =  w_X r_X - w_Y r_Y  
\end{equation}
Where $r_X$ represents the supply APY corresponding to the long asset and $r_Y$ represents the borrow APY of the borrowed token.
As mentioned earlier, the liquidation risk is determined by the log-health process. We assume $\log X(t)$ and $\log Y(t)$ to be Lévy processes with Laplace exponents $\psi_X(\theta)$ and $\psi_Y(\theta)$. If $\log X$ and $\log Y$ are independent:
\begin{equation}
h(t) = \log\left( \frac{b_X w_X X_0}{w_Y Y_0} \right) + \left(\log X(t) - \log Y(t) \right) 
\end{equation}
The Laplace exponent of $h(t)$ is:
\begin{equation}
\psi_h(\theta) = \psi_X(\theta) + \psi_Y(-\theta)
\end{equation}
In the case of correlated Lévy drivers with a joint characteristic exponent ${\bf{\Psi}}_{X,Y}(\theta_1,\theta_2)$, one obtains the following:
\begin{equation}
\psi_h(\theta) = {\bf{\Psi}}_{X,Y} (\theta, - \theta) 
\end{equation}
\paragraph{Terminal Wealth.}  
Under the spectrally negative assumption, the classical one-sided Lévy first-passage results apply. The terminal wealth at time $T$ is described by the following equation:
\begin{equation}
 W_T = W_0 + w_XR_X (T) - w_YR_Y (T) + w_X r_X T - w_Y r_Y T - Lw_X 1_{ \{\tau_0 \leq T\}}   
\end{equation}
where $R_X(T) = \frac{X_T}{X_0} - 1$ and $R_Y(T) = \frac{Y_T}{Y_0} - 1$. The last term reflects the liquidation loss: upon liquidation being triggered $\tau_0 \leq T $, a certain amount $L$ of collateral is lost.  
The first moment of the wealth process in this setting is then:
\begin{equation}
\mathbb{E}[W_T]=W_0 + w_x(e^{T \psi_X(1)}-1) - w_x(e^{T \psi_X(1)}-1) 
+ w_X r_X T - w_Y r_Y T - Lw_X \pi (T)
\end{equation}
where $\pi(T) = \mathbb{P} (\tau_0 \leq T)$. 
So, the variance of the wealth process will be:
\begin{equation}
\text{Var}(W_T) = \text{Var}(A) + L^2w_X^2 \pi(T)(1-\pi(T)) - 
2Lw_X(\mathbb{E}[AI_T] - (\mathbb{E}[A]\pi (T)),
\end{equation}
where $A$ represents the difference between the returns of the two positions, and $I_T = {\bf{1}}_\{\tau_0 \leq T\}$.    
The term $\mathbb{E}[AI_T]$, which couples terminal returns to the liquidation event, must be estimated via simulation or conditional methods. Neglecting this dependency leads to the independence approximation:
\begin{equation}
\text{Var}(W_T) \approx \text{Var}(A) +  L^2w_X^2 \pi(T)(1-\pi(T))
\end{equation}
The variance of $A$ is given by the following equation:
\begin{equation}
\text{Var}(A) = w_X^2 \text{Var}[R_X(T)] + w_Y^2  \text{Var}[R_Y(T)] - 
2w_X w_Y  \text{Cov} [R_X(T), R_Y(T)]
\end{equation}
If the Lévy log-processes are independent:
\begin{equation}
\text{Var}[R_X(T)] = e^{T \psi X(2)} - e^{T \psi X(1)}, 
\ \ 
\text{Var}[R_Y(T)] = e^{T \psi Y(2)} - e^{T \psi Y(1)}
\end{equation}
For correlated Lévy drivers, the covariance is given by $\mathbb{E}[X_TY_T] = X_0Y_0 e^{T\Psi_{X,Y}(1,1)}$.
When longing a stablecoin, an investor could assume that the returns generated by price movements are virtually null: $\psi_X(1) \approx \psi_X(2) \approx 0$ and $R_X(T) \approx 0$. Based on these assumptions, the expected value of the wealth process simplifies to:
\begin{equation}
\mathbb{E}[W_T]=W_0 - w_x(e^{T \psi_X(1)}-1) 
+ w_X r_X T - w_Y r_Y T - Lw_X \pi (T)    
\end{equation}
And the variance, under the independence assumption, is:
\begin{equation}
\text{Var}(W_T) = w_Y^2(e^{T \psi_Y (2)} - e^{T \psi_Y (1)} + L^2w_X^2 \pi(T)(1-\pi(T))
\end{equation}
In this configuration, the expected return is a function of the carry and the anticipated depreciation of the borrowed token. Conversely, the return variance may be attributed to a combination of volatility on the short side and the risk of liquidation.
\paragraph{Optimization Problem.}  

We now adapt the optimization problem to the stablecoin-collateral strategy. Parameterizing the problem in terms of the allocation ratio described in Section 5 reduces optimization to a one-dimensional decision problem in $r$.
The investor’s utility is defined as:
\begin{equation}
\max_{r} \quad \mathcal{U}(r) 
= \mathbb{E}[W_T(r)] 
- \rho_1 \, \mathbb{V}[W_T(r)] 
- \rho_2 \, \mathbb{P}(\tau_0 \leq T^*),
\label{eq:objective_stablecoin}
\end{equation}
where $W_T(r)$ denotes terminal wealth at time $T^*$. The first two terms are determined by the wealth process moments derived earlier, while the probability of liquidation is given by the first-hitting time distribution of the log-health process.  
The initial log-health is:
\begin{equation}
h_0(r) = \log \left( \frac{b_X w_X X_0}{w_Y Y_0} \right) 
= \log b_X + \log r + \log\!\left(\frac{X_0}{Y_0}\right),
\label{eq:initial_health_stablecoin}
\end{equation}
so that the risk of liquidation is explicitly linked to the allocation ratio.  

\paragraph{Solution.} Provided that $W_T(r)$ is characterized solely by its first two moments and the probability of liquidation requires a Laplace inversion, closed-form solutions are generally not available. As a result, we adopt a numerical solution approach:
\begin{enumerate}
    \item Compute $\mathbb{E}[W_T(r)]$ and $\mathbb{V}[W_T(r)]$ using the expressions described above.
    \item Evaluate $\mathbb{P}(\tau_0 \leq T^*)$ via the Laplace transform inversion of Theorem~\ref{thm:fht_laplace_section3}.
    \item Approximate numerical gradients $\nabla \mathcal{U}(r)$ by finite differences.
    \item Employ projected gradient ascent to update $r$, ensuring feasibility with respect to the constraints.
    \item Employ backtracking line search for step-size selection and terminate when both the gradient norm and objective improvement are below tolerance.
\end{enumerate}

The optimization problem highlights the fundamental trade-off between over-collateralization and return amplification. Increasing $r$ improves initial health $h_0(r)$, thus lowering the probability of liquidation, but concurrently reduces expected carry. Conversely, decreasing $r$ leverages the short exposure, raising expected returns but pushing the process toward the almost sure liquidation regime.  
The optimal collateral ratio $r^*$ represents a trade-off between carry yield, volatility risk, and liquidation probability. A feasible optimum exists only in the defective hitting regime, where liquidation is not certain; in the alternative regime, no sustainable position exists. Thus, this framework rigorously determines feasible and optimal leverage for DeFi strategies.
\\
\paragraph{Calibration and DeFi Market Data.}  
To demonstrate the practical implications of our framework, we calibrate the model using data from decentralized lending protocols and cryptocurrency markets. This calibration evaluates the strategy's feasibility under realistic conditions and quantifies the sensitivity of key outputs—optimal allocations and liquidation probabilities—to market volatility and jump risk.

We use historical data from two major lending platforms, namely \textit{Aave} and \textit{Compound}, focusing on weekly averages of supply and borrow APY. Stablecoin supply rates $r_X$ are typically in the range of 2--6\% per annum, while borrowing rates for the most volatile assets (e.g. ETH, WBTC) range from 5--10\%. These rates determine the deterministic carry component from equation 36.

As mentioned earlier, the asset prices $\log X(t)$ and $\log Y(t)$ are modeled as Lévy jump-diffusions. The parameters $(\mu_i, \sigma_i, \lambda_i, \delta_i)$ are estimated using daily log returns of representative assets, using a rolling window.
We assume stablecoins to exhibit negligible diffusion and jump terms, i.e. $\sigma_X \approx 0$, $\lambda_X \approx 0$, while rather volatile assets (e.g. SOL, BTC) display significant jump intensities and fat-tailed return distributions.

\paragraph{Health Process Parameters.}  
Using the given Lévy exponents $\psi_X(\theta)$ and $\psi_Y(\theta)$, we derive the log-health process exponent as $\psi_h(\theta) = \psi_X(\theta) + \psi_Y(-\theta)$ (see Equation 38). The system's regime—whether it is almost-sure or defective—is determined by the drift $\kappa_h = \mathbb{E}[h(t)]/t$. This calibration allows us to directly set the initial log-health $h_0(r)$ and manage its evolution during market stress.

\paragraph{Numerical Evaluation.}  
For each set of estimated parameters, we compute the objective function in Eq.~\eqref{eq:objective_stablecoin}. The moments of terminal wealth, $\mathbb{E}[W_T(r)]$ and $\mathbb{V}[W_T(r)]$, are derived from their analytical expressions, while the liquidation probabilities $\mathbb{P}(\tau_0 \leq T^*)$ are obtained by numerical Laplace inversion. Specifically, we invert the expression $\mathbb{E}[e^{-q \tau_0}] = e^{-h_0 \Phi(q)}$ using the stable Gaver–Stehfest algorithm with $N=16$ terms.

\paragraph{Parameter Values.}  
Placeholder parameters:
\begin{align*}
r_X &= 0.04, \quad r_Y = 0.08,\quad\sigma_Y = 0.85,\\
 \lambda_Y &= 0.12,\quad \delta_Y = 0.25, \quad h_{\min} = 0.1,\\
 T^* &= 30 \text{ days}.
\end{align*}

\paragraph{Implementation.}  
All computations are performed in \texttt{Python}, with Laplace inversions handled by custom routines built on \texttt{mpmath}. Projected gradient ascent is used to solve the optimization problem, ensuring $h_0(r) \geq h_{\min}$ and $0 < r < 1$ at each iteration.

\paragraph{Results.}  
Simulation results yield optimal allocation ratios $r^*$ as functions of the risk-aversion parameters $(\rho_1, \rho_2)$, collateral thresholds, and volatility levels. The resulting efficient frontier highlights the nonlinear trade-off between expected carry and liquidation probability.  
In particular, we find that(Placeholders):
\begin{enumerate}
    \item For defective-hitting regimes (positive effective drift $\kappa_h > 0$), feasible long-term allocations exist with moderate leverage and low liquidation probability. 
    \item In almost-sure liquidation regimes (negative $\kappa_h$ dominated by jumps), optimal allocations collapse toward $r \to 1$, implying that leveraged shorting is not sustainable.
    \item Sensitivity analysis confirms that the stability of optimal $r^*$ depends strongly on jump intensity $\lambda_Y$ and the initial health $h_0$, underscoring the importance of jump-risk calibration.
\end{enumerate}

The following section presents empirical validation using historical DeFi rates and price data. We evaluate realized liquidation frequencies against model-implied probabilities, test robustness under different collateral factors, and compare the optimal allocation rule to naive leverage strategies.

A reference implementation in Python is provided for replication and sensitivity analysis. The script simulates jump–diffusion dynamics for the shorted asset, computes $\pi(T)$ via Monte Carlo simulation, and evaluates the utility $\mathcal{U}(r)$ on a discrete grid of allocation ratios. The framework outputs the optimal ratio $r^*$ together with the corresponding expected wealth, variance, and liquidation probability, enabling direct numerical comparison across risk-aversion parameters $(\rho_1, \rho_2)$ and market conditions.


\paragraph{Kelly Criterion Implementation.}
To integrate the Kelly Criterion into this strategy, we define the 

optimization problem in terms of maximizing the long-term growth rate of wealth, rather than using a mean-variance utility function. The core components of the Kelly formula—win probability ($p$), loss probability ($q$), and net odds ($b$)—are derived directly from the jump-diffusion and liquidation risk framework.

\subparagraph{Defining Win/Loss Probabilities and Outcomes.}
We define a binary outcome for the Kelly calculation over a fixed time horizon $T^*$: 
\begin{itemize}
    \item \textbf{Loss Event (Liquidation):} The position is liquidated within the time horizon $T^*$. The probability of this event, $q$, is the liquidation probability $\pi(T^*) = \mathbb{P}(\tau_0 \leq T^*)$. The loss incurred is assumed to be a fraction $L$ of the collateral, so the net loss on the wager is $R_{loss} = -L w_X$.
    \item \textbf{Win Event (No Liquidation):} The position is not liquidated within the time horizon $T^*$. The probability of this event is $p = 1 - q = 1 - \pi(T^*)$. The expected gain, conditional on no liquidation, is the net return from the carry and the short position's performance. 
\end{itemize}

\subparagraph{Estimating Probabilities and Odds.}
The probabilities are directly given by the first-hitting time analysis:
\begin{align}
q &= \pi(T^*) = \mathbb{P}(\tau_0 \leq T^*) = \mathcal{L}^{-1}\left[\frac {e^{-h_0 \Phi(q)}}{q}\right](T^*) \\
p &= 1 - q
\end{align}
The net odds, $b$, are the ratio of the expected gain if not liquidated to the loss if liquidated:
\begin{equation}
b = \frac{\mathbb{E}[\text{Gain} | \tau_0 > T^*]}{\mathbb{E}[\text{Loss} | \tau_0 \leq T^*]} = \frac{w_X r_X T^* - w_Y r_Y T^* - w_Y(e^{T^* \psi_Y(1)}-1)}{L w_X}
\end{equation}
Here, we use the simplified expected wealth expression, assuming the stablecoin's price return is negligible and focusing on the carry and the short leg's performance.

\paragraph{Optimization Problem with Kelly Criterion.}
The objective is to find the allocation ratio $r$ that maximizes the expected logarithmic growth of wealth. The Kelly fraction, $f^*(r)$, which represents the optimal fraction of capital to allocate to this strategy, is given by:
\begin{equation}
f^*(r) = \frac{b(r)p(r) - q(r)}{b(r)}
\end{equation}
Since $p, q,$ and $b$ are all functions of the allocation ratio $r$ (through $w_X, w_Y,$ and $h_0$), the optimization problem becomes:
\begin{equation}
\max_{r} \quad G(r) = q(r) \log(1 - f^*(r) L w_X) + p(r) \log(1 + f^*(r) b(r) L w_X)
\end{equation}
where $G(r)$ is the expected logarithmic growth rate. A more direct approach is to maximize the Kelly fraction $f^*(r)$ itself, as this is equivalent to maximizing the growth rate.

For practical implementation, we use a fractional Kelly approach to mitigate the risks of model error and high volatility. The target allocation fraction becomes:
\begin{equation}
f_{target}(r) = k \times f^*(r)
\end{equation}
where $k$ is the Kelly fraction (e.g., $k=0.5$ for Half Kelly, $k=0.25$ for Quarter Kelly).

\paragraph{Solution.}
As with the mean-variance approach, a closed-form solution is not available. We solve this numerically:
\begin{enumerate}
    \item For a given allocation ratio $r$, calculate the corresponding weights $w_X(r), w_Y(r)$ and initial log-health $h_0(r)$.
    \item Compute the liquidation probability $q(r) = \pi(T^*)$ via numerical Laplace inversion.
    \item Calculate the win probability $p(r) = 1 - q(r)$.
    \item Calculate the net odds $b(r)$ based on the expected returns and liquidation loss.
    \item Compute the full Kelly fraction $f^*(r)$.
    \item The optimization then searches for the ratio $r$ that maximizes $f^*(r)$, subject to the constraints $h_0(r) \geq h_{\min}$ and $0 < r < 1$. This can be done using a grid search or a numerical optimization routine like projected gradient ascent on $f^*(r)$.
\end{enumerate}

The optimal allocation ratio $r^*$ under the Kelly framework represents the trade-off that maximizes the long-term geometric growth rate of the portfolio, explicitly balancing the potential for high returns from leverage against the risk of ruin from liquidation. This provides a more direct and theoretically grounded approach to position sizing compared to utility-based methods.

\subsection{Long-Short Momentum Strategy}

For our second use case, we consider a long–short momentum strategy: here momentum is defined as performance over a specific period of time. A long position is taken in an asset with positive recent returns (``momentum winner''), and a short position is taken in an asset with negative recent returns (``momentum loser''). The idea underlying the strategy is fairly simple: the investor must buy tokens that are outperforming and short-sell those that are underperforming..

The net expected return is proportional to the difference in expected performance between the two assets, scaled by position weights:
\begin{equation}
R(w_L, w_S) = w_L r_L - w_S r_S,
\end{equation}
where $r_L$ is the expected return of the long asset, and $r_S$ the expected return of the short asset.  

In this different setting, we update the log-health process:
\begin{equation}
h(t) = \log\left(\frac{b_L w_L L_0}{w_S S_0}\right) + \left(\log L(t) - \log S(t)\right),
\end{equation}
where $L(t)$ and $S(t)$ are the long and short asset price processes, $L_0, S_0$ their initial prices, and $b_L$ is a collateral factor for the long asset.

Assuming $\log L$ and $\log S$ follow Lévy processes with Laplace exponents $\psi_L(\theta)$ and $\psi_S(\theta)$, the Laplace exponent of $h(t)$ is
\begin{equation}
\psi_h(\theta) = \psi_L(\theta) + \psi_S(-\theta).
\end{equation}
In the case of correlated drivers with joint characteristic exponent ${\bf{\Psi}}_{L,S}(\theta_1,\theta_2)$:
\begin{equation}
\psi_h(\theta) = {\bf{\Psi}}_{L,S}(\theta,-\theta).
\end{equation}

\paragraph{Terminal Wealth.}  
The total wealth of the portfolio at time $T$ is:
\begin{equation}
W_T = W_0 + w_L R_L(T) - w_S R_S(T) - L w_L 1_{\{\tau_0 \leq T\}},
\end{equation}
where $R_L(T) = \frac{L_T}{L_0}-1$ and $R_S(T) = \frac{S_T}{S_0}-1$. The last term represents the liquidation loss triggered by $h(t)$ reaching zero.

The first two moments of the wealth process are the following:
\begin{equation}
\mathbb{E}[W_T] = W_0 + w_L(e^{T\psi_L(1)}-1) - w_S(e^{T\psi_S(1)}-1) - L w_L \pi(T),
\end{equation}
\begin{equation}
\text{Var}(W_T) = \text{Var}(A) + L^2 w_L^2 \pi(T)(1-\pi(T)) - 2L w_L (\mathbb{E}[A I_T] - \mathbb{E}[A]\pi(T)),
\end{equation}
where $A = w_L R_L(T) - w_S R_S(T)$ and $I_T = 1_{\{\tau_0 \leq T\}}$. Under the independence approximation:
\begin{equation}
\text{Var}(W_T) \approx \text{Var}(A) + L^2 w_L^2 \pi(T)(1-\pi(T)),
\end{equation}
with
\begin{equation}
\text{Var}(A) = w_L^2 \text{Var}[R_L(T)] + w_S^2 \text{Var}[R_S(T)] - 2 w_L w_S \text{Cov}[R_L(T),R_S(T)].
\end{equation}

In this configuration, the expected return is a function of the collateralized carry and the relative performance of the borrowed asset. The variance of the wealth process arises from both asset volatility and the risk of liquidation.  

\paragraph{Optimization Problem.}  
The investor’s utility is:
\begin{equation}
\max_{r} \quad \mathcal{U}(r) 
= \mathbb{E}[W_T(r)] 
- \rho_1 \, \mathbb{V}[W_T(r)] 
- \rho_2 \, \mathbb{P}(\tau_0 \leq T^*),
\label{eq:objective_momentum_defi}
\end{equation}
with $W_T(r)$ the terminal wealth at $T^*$. The initial log-health is
\begin{equation}
h_0(r) = \log \left( \frac{b_L w_L L_0}{w_S S_0} \right) 
= \log b_L + \log r + \log\!\left(\frac{L_0}{S_0}\right),
\end{equation}
which links the probability of liquidation to the allocation and initial asset prices.

\paragraph{Solution.}  
As in the stablecoin strategy, closed-form solutions are generally unavailable, so we adopt a numerical approach:
\begin{enumerate}
    \item Compute $\mathbb{E}[W_T(r)]$ and $\mathbb{V}[W_T(r)]$ using the moment formulas above.
    \item Evaluate $\mathbb{P}(\tau_0 \leq T^*)$ via Laplace inversion of the first-passage time of $h(t)$.
    \item Approximate $\nabla \mathcal{U}(r)$ using finite differences.
    \item Apply projected gradient ascent to update $r$, enforcing $r>0$.
    \item Use backtracking line search and terminate when the gradient and objective improvements are below tolerance.
\end{enumerate}

\paragraph{Calibration and Market Data.}  
The model is calibrated on daily data for tokens with pronounced cross-sectional momentum. Momentum winners (long leg) and losers (short leg) are selected based on extreme cumulative returns over a $k=30$ day formation period. The Lévy jump-diffusion parameters $(\mu_i, \sigma_i, \lambda_i, \delta_i)$ for all assets are then estimated through rolling windows of historical log-returns. Empirical estimates show that momentum winners are characterized by a higher drift $\mu_L$ and comparable or slightly higher volatility $\sigma_L$ relative to losers. The jump parameters $(\lambda_i, \delta_i)$ effectively capture the pronounced asymmetry and fat tails in crypto returns. This calibration aligns with the stylized fact that positive-return assets often exhibit sustained upward drift alongside elevated idiosyncratic jump risk.

\paragraph{Numerical Evaluation.}  
For each set of calibrated parameters, we compute the utility in \eqref{eq:objective_momentum_defi}. Laplace inversion is used to evaluate $\pi(T) = \mathbb{P}(\tau_0 \le T^*)$. Gradients are approximated numerically, and $r^*$ is located using projected gradient ascent subject to feasibility constraints.

\paragraph{Parameter Values (Baseline).}
\begin{align*}
\mu_L &= 0.10, &\sigma_L &= 0.80, &\lambda_L &= 0.10, &\delta_L &= 0.20,\\
\mu_S &= 0.05, &\sigma_S &= 0.70, &\lambda_S &= 0.20, &\delta_S &= 0.25,\\
T^* &= 30 \text{ days}, &h_{\min} &= 0.1.
\end{align*}

\paragraph{Results.}  
Simulation produces optimal allocation ratios $r^*$ as functions of $(\rho_1, \rho_2)$ and asset asymmetries. Key findings (Placeholder):
\begin{enumerate}
    \item In defective-hitting regimes ($\kappa_h>0$), moderate momentum exposure is feasible with low liquidation probability.
    \item In almost-sure breakdown regimes ($\kappa_h<0$), $r^* \to 0$, effectively neutralizing the momentum position.
    \item Sensitivity analysis shows $r^*$ is strongly affected by jump intensity $\lambda_S$ and volatility $\sigma_S$, as downside risk of the short leg dominates.
\end{enumerate}


\paragraph{Kelly Criterion Implementation.}
To integrate the Kelly Criterion into this strategy, we define the optimization problem in terms of maximizing the long-term growth rate of wealth, rather than using a mean-variance utility function. The core components of the Kelly formula—win probability ($p$), loss probability ($q$), and net odds ($b$)—are derived directly from the jump-diffusion and liquidation risk framework.

\subparagraph{Defining Win/Loss Probabilities and Outcomes.}
We define a binary outcome for the Kelly calculation over a fixed time horizon $T^*$: 
\begin{itemize}
    \item \textbf{Loss Event (Liquidation):} The position is liquidated within the time horizon $T^*$. The probability of this event, $q$, is the liquidation probability $\pi(T^*) = \mathbb{P}(\tau_0 \leq T^*)$. The loss incurred is assumed to be a fraction $L$ of the collateral, so the net loss on the wager is $R_{loss} = -L w_L$.
    \item \textbf{Win Event (No Liquidation):} The position is not liquidated within the time horizon $T^*$. The probability of this event is $p = 1 - q = 1 - \pi(T^*)$. The expected gain, conditional on no liquidation, is the net return from the long and short positions' performance. 
\end{itemize}

\subparagraph{Estimating Probabilities and Odds.}
The probabilities are directly given by the first-hitting time analysis:
\begin{align}
q &= \pi(T^*) = \mathbb{P}(\tau_0 \leq T^*) = \mathcal{L}^{-1}\left[\frac {e^{-h_0 \Phi(q)}}{q}\right](T^*) \\
p &= 1 - q
\end{align}
The net odds, $b$, are the ratio of the expected gain if not liquidated to the loss if liquidated:
\begin{equation}
b = \frac{\mathbb{E}[\text{Gain} | \tau_0 > T^*]}{\mathbb{E}[\text{Loss} | \tau_0 \leq T^*]} = \frac{w_L(e^{T^*\psi_L(1)}-1) - w_S(e^{T^*\psi_S(1)}-1)}{L w_L}
\end{equation}
Here, the expected gain is based on the expected returns of the long and short assets over the period $T^*$, conditional on no liquidation.

\paragraph{Optimization Problem with Kelly Criterion.}
The objective is to find the allocation ratio $r$ that maximizes the expected logarithmic growth of wealth. The Kelly fraction, $f^*(r)$, which represents the optimal fraction of the total capital to allocate to this strategy, is given by:
\begin{equation}
f^*(r) = \frac{b(r)p(r) - q(r)}{b(r)}
\end{equation}
Since $p, q,$ and $b$ are all functions of the allocation ratio $r$ (through $w_L, w_S,$ and $h_0$), the optimization problem becomes:
\begin{equation}
\max_{r} \quad G(r) = q(r) \log(1 - f^*(r) L w_L) + p(r) \log(1 + f^*(r) b(r) L w_L)
\end{equation}
where $G(r)$ is the expected logarithmic growth rate. A more direct approach is to maximize the Kelly fraction $f^*(r)$ itself, as this is equivalent to maximizing the growth rate.

For practical implementation, we use a fractional Kelly approach to mitigate the risks of model error and high volatility. The target allocation fraction becomes:
\begin{equation}
f_{target}(r) = k \times f^*(r)
\end{equation}
where $k$ is the Kelly fraction (e.g., $k=0.5$ for Half Kelly, $k=0.25$ for Quarter Kelly).

\paragraph{Solution.}
As with the mean-variance approach, a closed-form solution is not available. We solve this numerically:
\begin{enumerate}
    \item For a given allocation ratio $r$, calculate the corresponding weights $w_L(r), w_S(r)$ and initial log-health $h_0(r)$.
    \item Compute the liquidation probability $q(r) = \pi(T^*)$ via numerical Laplace inversion.
    \item Calculate the win probability $p(r) = 1 - q(r)$.
    \item Calculate the net odds $b(r)$ based on the expected returns and liquidation loss.
    \item Compute the full Kelly fraction $f^*(r)$.
    \item The optimization then searches for the ratio $r$ that maximizes $f^*(r)$, subject to the constraints $h_0(r) \geq h_{\min}$ and $0 < r < 1$. This can be done using a grid search or a numerical optimization routine like projected gradient ascent on $f^*(r)$.
\end{enumerate}

The optimal allocation ratio $r^*$ under the Kelly framework represents the trade-off that maximizes the long-term geometric growth rate of the portfolio, explicitly balancing the potential for high returns from leverage against the risk of ruin from liquidation. This provides a more direct and theoretically grounded approach to position sizing compared to utility-based methods.

\paragraph{Calibration and Market Data.}  
The model is calibrated on daily data for tokens with pronounced cross-sectional momentum. Momentum winners (long leg) and losers (short leg) are selected based on extreme cumulative returns over a $k=30$ day formation period. The Lévy jump-diffusion parameters $(\mu_i, \sigma_i, \lambda_i, \delta_i)$ for all assets are then estimated through rolling windows of historical log-returns. Empirical estimates show that momentum winners are characterized by a higher drift $\mu_L$ and comparable or slightly higher volatility $\sigma_L$ relative to losers. The jump parameters $(\lambda_i, \delta_i)$ effectively capture the pronounced asymmetry and fat tails in crypto returns. This calibration aligns with the stylized fact that positive-return assets often exhibit sustained upward drift alongside elevated idiosyncratic jump risk.

\paragraph{Numerical Evaluation.}  
For each set of calibrated parameters, we compute the Kelly fraction $f^*(r)$. Laplace inversion is used to evaluate $\pi(T) = \mathbb{P}(\tau_0 \le T^*)$. Gradients are approximated numerically, and $r^*$ is located using projected gradient ascent subject to feasibility constraints.

\paragraph{Parameter Values (Baseline).}
\begin{align*}
\mu_L &= 0.10, &\sigma_L &= 0.80, &\lambda_L &= 0.10, &\delta_L &= 0.20,\\
\mu_S &= 0.05, &\sigma_S &= 0.70, &\lambda_S &= 0.20, &\delta_S &= 0.25,\\
T^* &= 30 \text{ days}, &h_{\min} &= 0.1.
\end{align*}

\paragraph{Results.}  
Simulation produces optimal allocation ratios $r^*$ as functions of the Kelly fraction $k$ and asset asymmetries. Key findings (Placeholder):
\begin{enumerate}
    \item In defective-hitting regimes ($\kappa_h>0$), moderate momentum exposure is feasible with low liquidation probability.
    \item In almost-sure breakdown regimes ($\kappa_h<0$), $r^* \to 0$, effectively neutralizing the momentum position.
    \item Sensitivity analysis shows $r^*$ is strongly affected by jump intensity $\lambda_S$ and volatility $\sigma_S$, as downside risk of the short leg dominates.
\end{enumerate}

\subsection{Beta-Neutral Long–Short Strategy}

The third use case implements a beta-neutral long-short strategy with the goal of isolating idiosyncratic returns. The mechanism is to neutralize the exposure to the aggregate cryptocurrency market. This is achieved by taking a long position in a high-alpha asset $\alpha_L$ and a short position in an asset with a similar beta but lower alpha $\alpha_S$, thus capturing the spread of alpha while mitigating the systematic risk.
The net expected return is given by the weighted difference in the assets' idiosyncratic alphas:
\begin{equation}
R(w_L, w_S) = w_L \alpha_L - w_S \alpha_S,
\end{equation}
where $\alpha_L$ is the expected idiosyncratic return of the long asset, and $\alpha_S$ is the expected idiosyncratic return of the short asset.

Let $\beta_i$ denote the estimated market beta of asset $i$ with respect to the representative crypto market index CRIX. We construct a portfolio composed of a long leg $\mathcal{L}$ and a short leg $\mathcal{S}$, with corresponding portfolio weights $w_{\mathcal{L}}$ and $w_{\mathcal{S}}$. Beta-neutrality means that the total systematic exposure vanishes:
\begin{equation}
    \beta_P = \sum_{i \in \mathcal{L}} w_i \beta_i - \sum_{j \in \mathcal{S}} w_j \beta_j \approx 0.
\end{equation}
To ensure market neutrality, we adjust the scaling between the two legs such that
\begin{equation}
    \frac{\sum_{i \in \mathcal{L}} w_i \beta_i}{\sum_{j \in \mathcal{S}} w_j \beta_j} = 1,
\end{equation}
while maintaining a net positive collateralization ratio.

\paragraph{Implementation.}
\begin{enumerate}
    \item Supply the lending pool with a bluechip cryptocurrency as collateral on a lending protocol.
    \item Borrow a correlated cryptocurrency to short.
    \item Periodically rebalance to maintain beta neutrality.
\end{enumerate}

The economic rationale hinges on the pronounced common-factor behavior observed in cryptocurrency markets, primarily driven by liquidity and sentiment. By neutralizing market beta, the portfolio hedges out this systematic exposure, generating returns from relative performance and transient valuation spreads between fundamentally aligned assets. The target expected return is calculated from the following equation:
\begin{equation}
    \mathbb{E}[r_P] \approx \sum_{i \in \mathcal{L}} w_i \alpha_i - \sum_{j \in \mathcal{S}} w_j \alpha_j,
\end{equation}
where $\alpha_i$ denotes the idiosyncratic excess return of asset $i$. 

In this different setting, we update the log-health process:
\begin{equation}
h(t) = \log\left(\frac{b_L w_L L_0}{w_S S_0}\right) + \left(\log L(t) - \log S(t)\right),
\end{equation}
where $L(t)$ and $S(t)$ are the long and short asset price processes, $L_0, S_0$ their initial prices, and $b_L$ is a collateral factor for the long asset.

Assuming $\log L$ and $\log S$ follow Lévy processes with Laplace exponents $\psi_L(\theta)$ and $\psi_S(\theta)$, the Laplace exponent of $h(t)$ is
\begin{equation}
\psi_h(\theta) = \psi_L(\theta) + \psi_S(-\theta).
\end{equation}
In the case of correlated drivers with joint characteristic exponent ${\bf{\Psi}}_{L,S}(\theta_1,\theta_2)$:
\begin{equation}
\psi_h(\theta) = {\bf{\Psi}}_{L,S}(\theta,-\theta).
\end{equation}

\paragraph{Terminal Wealth.}  
The total wealth of the portfolio at time $T$ is:
\begin{equation}
W_T = W_0 + w_L R_L(T) - w_S R_S(T) - L w_L 1_{\{\tau_0 \leq T\}},
\end{equation}
where $R_L(T) = \frac{L_T}{L_0}-1$ and $R_S(T) = \frac{S_T}{S_0}-1$. The last term represents the liquidation loss triggered by $h(t)$ reaching zero.
The first two moments of the wealth process are:
\begin{equation}
\mathbb{E}[W_T] = W_0 + w_L(e^{T\psi_L(1)}-1) - w_S(e^{T\psi_S(1)}-1) - L w_L \pi(T),
\end{equation}
\begin{equation}
\text{Var}(W_T) = \text{Var}(A) + L^2 w_L^2 \pi(T)(1-\pi(T)) - 2L w_L (\mathbb{E}[A I_T] - \mathbb{E}[A]\pi(T)),
\end{equation}
where $A = w_L R_L(T) - w_S R_S(T)$ and $I_T = 1_{\{\tau_0 \leq T\}}$. Under the independence approximation:
\begin{equation}
\text{Var}(W_T) \approx \text{Var}(A) + L^2 w_L^2 \pi(T)(1-\pi(T)),
\end{equation}
with
\begin{equation}
\text{Var}(A) = w_L^2 \text{Var}[R_L(T)] + w_S^2 \text{Var}[R_S(T)] - 2 w_L w_S \text{Cov}[R_L(T),R_S(T)].
\end{equation}

This configuration models the expected return as a function of the collateralized carry and the relative performance of the borrowed asset. The variance of the terminal wealth, in contrast, is driven by the combined effects of asset price volatility and the risk of a liquidation event.

\paragraph{Optimization Problem.}  
The investor’s utility is:
\begin{equation}
\max_{r} \quad \mathcal{U}(r) 
= \mathbb{E}[W_T(r)] 
- \rho_1 \, \mathbb{V}[W_T(r)] 
- \rho_2 \, \mathbb{P}(\tau_0 \leq T^*),
\label{eq:objective_beta_neutral}
\end{equation}
with $W_T(r)$ the terminal wealth at $T^*$. The initial log-health is
\begin{equation}
h_0(r) = \log \left( \frac{b_L w_L L_0}{w_S S_0} \right) 
= \log b_L + \log r + \log\!\left(\frac{L_0}{S_0}\right),
\end{equation}
which links the probability of liquidation to the allocation, collateral factor, and initial asset prices.

\paragraph{Solution.}  
As in the previous sections, we adopt a numerical approach:
\begin{enumerate}
    \item Compute $\mathbb{E}[W_T(r)]$ and $\mathbb{V}[W_T(r)]$ using the moment formulas above.
    \item Evaluate $\mathbb{P}(\tau_0 \leq T^*)$ via Laplace inversion of the first-passage time of $h(t)$.
    \item Approximate $\nabla \mathcal{U}(r)$ using finite differences.
    \item Apply projected gradient ascent to update $r$, enforcing $r>0$.
    \item Use backtracking line search and terminate when the gradient and objective improvements are below tolerance.
\end{enumerate}

\paragraph{Calibration and Market Data.}  
We calibrate the model using historical data from crypto assets with large idiosyncratic alpha differences. The long and short assets are chosen to enforce $\beta_P \approx 0$. Parameter estimation for the Lévy jump-diffusion model $(\mu_i, \sigma_i, \lambda_i, \delta_i)$ employs rolling windows of log-returns. Empirical results indicate that long assets have higher $\alpha_L$ and comparable or slightly higher $\sigma_L$ than short assets, and the jump parameters account for the tail asymmetry inherent in crypto markets.

\paragraph{Health Process Parameters.}
The log-health process \( h(t) \) quantifies the collateralization health of the leveraged long--short momentum portfolio. Its characteristic exponent is defined as
\[
\psi_h(\theta) = \psi_L(\theta) + \psi_S(-\theta),
\]
synthesizing the Lévy dynamics of the long winner and short loser legs (cf. Equation 38). This captures the joint jump-diffusion risk, where adverse jumps in the winner or favorable jumps in the loser drive the health process downward toward liquidation.

The asymptotic regime is determined by the drift
\[
\kappa_h = \frac{\mathbb{E}[h(t)]}{t}.
\]
A positive \(\kappa_h\) indicates an almost-sure (sustainable) regime, while a negative drift implies a defective regime where liquidation becomes probable. Calibration of \(\psi_h(\theta)\) and \(\kappa_h\) enables us to initialize the log-health \(h_0(r)\) via the leverage ratio \(r\) and simulate its stressed evolution, thereby endogenously linking liquidation risk to the relative performance of the momentum legs.

\paragraph{Numerical Evaluation.}  
For each calibrated parameter set, we compute $\mathcal{U}(r)$ in \eqref{eq:objective_beta_neutral}. Laplace inversion is used to estimate $\pi(T) = \mathbb{P}(\tau_0 \le T^*)$. Gradients are approximated numerically, and $r^*$ is obtained using projected gradient ascent subject to feasibility constraints.

\paragraph{Health Process Parameters.}
The log-health process exponent is derived from the Lévy exponents of the portfolio's long and short legs:
\begin{equation}
\psi_h(\theta) = \psi_L(\theta) + \psi_S(-\theta)
\end{equation}
(see Equation 38). The system's regime—almost-sure or defective—is determined by the drift $\kappa_h = \mathbb{E}[h(t)]/t$.
This calibration directly sets the initial log-health $h_0(r)$ via the collateral ratio and position scaling, allowing us to manage its evolution under market stress while preserving beta neutrality and a positive health factor.

\paragraph{Parameter Values (Baseline).}
\begin{align*}
\mu_L &= 0.08, &\sigma_L &= 0.75, &\lambda_L &= 0.08, &\delta_L &= 0.20,\\
\mu_S &= 0.04, &\sigma_S &= 0.70, &\lambda_S &= 0.12,  &\delta_S &= 0.22,\\
T^* &= 30 \text{ days}, &h_{\min} &= 0.1.
\end{align*}

\paragraph{Results.}  
Simulation produces optimal allocation ratios $r^*$ as functions of $(\rho_1, \rho_2)$ and asset asymmetries. Key findings (Placeholder):
\begin{enumerate}
    \item Beta-neutral pairs exhibit lower variance but comparable expected returns to directional strategies.
    \item Strong idiosyncratic alpha differences yield positive drift in $h(t)$ and reduced liquidation probability.
    \item High correlation or elevated jump intensity reduces the feasible range of $r$ for maintaining $\beta_P \approx 0$ and $H(t)>1$.
\end{enumerate}

\paragraph{Kelly Criterion Implementation.}
To integrate the Kelly Criterion into this strategy, we define the optimization problem in terms of maximizing the long-term growth rate of wealth, rather than using a mean-variance utility function. The core components of the Kelly formula—win probability ($p$), loss probability ($q$), and net odds ($b$)—are derived directly from the jump-diffusion and liquidation risk framework.

\subparagraph{Defining Win/Loss Probabilities and Outcomes.}
We define a binary outcome for the Kelly calculation over a fixed time horizon $T^*$: 
\begin{itemize}
    \item \textbf{Loss Event (Liquidation):} The position is liquidated within the time horizon $T^*$. The probability of this event, $q$, is the liquidation probability $\pi(T^*) = \mathbb{P}(\tau_0 \leq T^*)$. The loss incurred is assumed to be a fraction $L$ of the collateral, so the net loss on the wager is $R_{loss} = -L w_L$.
    \item \textbf{Win Event (No Liquidation):} The position is not liquidated within the time horizon $T^*$. The probability of this event is $p = 1 - q = 1 - \pi(T^*)$. The expected gain, conditional on no liquidation, is the net return from the long and short positions' idiosyncratic alpha. 
\end{itemize}

\subparagraph{Estimating Probabilities and Odds.}
The probabilities are directly given by the first-hitting time analysis:
\begin{align}
q &= \pi(T^*) = \mathbb{P}(\tau_0 \leq T^*) = \mathcal{L}^{-1}\left[\frac {e^{-h_0 \Phi(q)}}{q}\right](T^*) \\
p &= 1 - q
\end{align}
The net odds, $b$, are the ratio of the expected gain if not liquidated to the loss if liquidated:
\begin{equation}
b = \frac{\mathbb{E}[\text{Gain} | \tau_0 > T^*]}{\mathbb{E}[\text{Loss} | \tau_0 \leq T^*]} = \frac{w_L \alpha_L T^* - w_S \alpha_S T^*}{L w_L}
\end{equation}
Here, the expected gain is based on the expected idiosyncratic returns of the long and short assets over the period $T^*$, conditional on no liquidation.

\paragraph{Optimization Problem with Kelly Criterion.}
The objective is to find the allocation ratio $r$ that maximizes the expected logarithmic growth of wealth. The Kelly fraction, $f^*(r)$, which represents the optimal fraction of the total capital to allocate to this strategy, is given by:
\begin{equation}
f^*(r) = \frac{b(r)p(r) - q(r)}{b(r)}
\end{equation}
Since $p, q,$ and $b$ are all functions of the allocation ratio $r$ (through $w_L, w_S,$ and $h_0$), the optimization problem becomes:
\begin{equation}
\max_{r} \quad G(r) = q(r) \log(1 - f^*(r) L w_L) + p(r) \log(1 + f^*(r) b(r) L w_L)
\end{equation}
where $G(r)$ is the expected logarithmic growth rate. A more direct approach is to maximize the Kelly fraction $f^*(r)$ itself, as this is equivalent to maximizing the growth rate.

For practical implementation, we use a fractional Kelly approach to mitigate the risks of model error and high volatility. The target allocation fraction becomes:
\begin{equation}
f_{target}(r) = k \times f^*(r)
\end{equation}
where $k$ is the Kelly fraction (e.g., $k=0.5$ for Half Kelly, $k=0.25$ for Quarter Kelly).

\paragraph{Solution.}
As with the mean-variance approach, a closed-form solution is not available. We solve this numerically:
\begin{enumerate}
    \item For a given allocation ratio $r$, calculate the corresponding weights $w_L(r), w_S(r)$ and initial log-health $h_0(r)$.
    \item Compute the liquidation probability $q(r) = \pi(T^*)$ via numerical Laplace inversion.
    \item Calculate the win probability $p(r) = 1 - q(r)$.
    \item Calculate the net odds $b(r)$ based on the expected idiosyncratic returns and liquidation loss.
    \item Compute the full Kelly fraction $f^*(r)$.
    \item The optimization then searches for the ratio $r$ that maximizes $f^*(r)$, subject to the constraints $h_0(r) \geq h_{\min}$ and $0 < r < 1$. This can be done using a grid search or a numerical optimization routine like projected gradient ascent on $f^*(r)$.
\end{enumerate}

The optimal allocation ratio $r^*$ under the Kelly framework represents the trade-off that maximizes the long-term geometric growth rate of the portfolio, explicitly balancing the potential for high returns from leverage against the risk of ruin from liquidation. This provides a more direct and theoretically grounded approach to position sizing compared to utility-based methods.

\paragraph{Calibration and Market Data.}  
We calibrate the model using historical data from crypto assets with large idiosyncratic alpha differences. The long and short assets are chosen to enforce $\beta_P \approx 0$. Parameter estimation for the Lévy jump-diffusion model $(\mu_i, \sigma_i, \lambda_i, \delta_i)$ employs rolling windows of log-returns. Empirical results indicate that long assets have higher $\alpha_L$ and comparable or slightly higher $\sigma_L$ than short assets, and the jump parameters account for the tail asymmetry inherent in crypto markets.

\paragraph{Health Process Parameters.}
The log-health process \( h(t) \) quantifies the collateralization health of the leveraged long--short momentum portfolio. Its characteristic exponent is defined as
\[
\psi_h(\theta) = \psi_L(\theta) + \psi_S(-\theta),
\]
synthesizing the Lévy dynamics of the long winner and short loser legs (cf. Equation 38). This captures the joint jump-diffusion risk, where adverse jumps in the winner or favorable jumps in the loser drive the health process downward toward liquidation.

The asymptotic regime is determined by the drift
\[
\kappa_h = \frac{\mathbb{E}[h(t)]}{t}.
\]
A positive \(\kappa_h\) indicates an almost-sure (sustainable) regime, while a negative drift implies a defective regime where liquidation becomes probable. Calibration of \(\psi_h(\theta)\) and \(\kappa_h\) enables us to initialize the log-health \(h_0(r)\) via the leverage ratio \(r\) and simulate its stressed evolution, thereby endogenously linking liquidation risk to the relative performance of the momentum legs.

\paragraph{Numerical Evaluation.}  
For each calibrated parameter set, we compute the Kelly fraction $f^*(r)$. Laplace inversion is used to estimate $\pi(T) = \mathbb{P}(\tau_0 \le T^*)$. Gradients are approximated numerically, and $r^*$ is obtained using projected gradient ascent subject to feasibility constraints.

\paragraph{Parameter Values (Baseline).}
\begin{align*}
\mu_L &= 0.08, &\sigma_L &= 0.75, &\lambda_L &= 0.08, &\delta_L &= 0.20,\\
\mu_S &= 0.04, &\sigma_S &= 0.70, &\lambda_S &= 0.12,  &\delta_S &= 0.22,\\
T^* &= 30 \text{ days}, &h_{\min} &= 0.1.
\end{align*}

\paragraph{Results.}  
Simulation produces optimal allocation ratios $r^*$ as functions of the Kelly fraction $k$ and asset asymmetries. Key findings (Placeholder):
\begin{enumerate}
    \item Beta-neutral pairs exhibit lower variance but comparable expected returns to directional strategies.
    \item Strong idiosyncratic alpha differences yield positive drift in $h(t)$ and reduced liquidation probability.
    \item High correlation or elevated jump intensity reduces the feasible range of $r$ for maintaining $\beta_P \approx 0$ and $H(t)>1$.
\end{enumerate}

\paragraph{Kelly Criterion Implementation.}
To integrate the Kelly Criterion into this strategy, we define the optimization problem in terms of maximizing the long-term growth rate of wealth, rather than using a mean-variance utility function. The core components of the Kelly formula—win probability ($p$), loss probability ($q$), and net odds ($b$)—are derived directly from the jump-diffusion and liquidation risk framework.

\subparagraph{Defining Win/Loss Probabilities and Outcomes.}
We define a binary outcome for the Kelly calculation over a fixed time horizon $T^*$: 
\begin{itemize}
    \item \textbf{Loss Event (Liquidation):} The position is liquidated within the time horizon $T^*$. The probability of this event, $q$, is the liquidation probability $\pi(T^*) = \mathbb{P}(\tau_0 \leq T^*)$. The loss incurred is assumed to be a fraction $L$ of the collateral, so the net loss on the wager is $R_{loss} = -L w_L$.
    \item \textbf{Win Event (No Liquidation):} The position is not liquidated within the time horizon $T^*$. The probability of this event is $p = 1 - q = 1 - \pi(T^*)$. The expected gain, conditional on no liquidation, is the net return from the long and short positions' idiosyncratic alpha. 
\end{itemize}

\subparagraph{Estimating Probabilities and Odds.}
The probabilities are directly given by the first-hitting time analysis:
\begin{align}
q &= \pi(T^*) = \mathbb{P}(\tau_0 \leq T^*) = \mathcal{L}^{-1}\left[\frac {e^{-h_0 \Phi(q)}}{q}\right](T^*) \\
p &= 1 - q
\end{align}
The net odds, $b$, are the ratio of the expected gain if not liquidated to the loss if liquidated:
\begin{equation}
b = \frac{\mathbb{E}[\text{Gain} | \tau_0 > T^*]}{\mathbb{E}[\text{Loss} | \tau_0 \leq T^*]} = \frac{w_L \alpha_L T^* - w_S \alpha_S T^*}{L w_L}
\end{equation}
Here, the expected gain is based on the expected idiosyncratic returns of the long and short assets over the period $T^*$, conditional on no liquidation.

\paragraph{Optimization Problem with Kelly Criterion.}
The objective is to find the allocation ratio $r$ that maximizes the expected logarithmic growth of wealth. The Kelly fraction, $f^*(r)$, which represents the optimal fraction of the total capital to allocate to this strategy, is given by:
\begin{equation}
f^*(r) = \frac{b(r)p(r) - q(r)}{b(r)}
\end{equation}
Since $p, q,$ and $b$ are all functions of the allocation ratio $r$ (through $w_L, w_S,$ and $h_0$), the optimization problem becomes:
\begin{equation}
\max_{r} \quad G(r) = q(r) \log(1 - f^*(r) L w_L) + p(r) \log(1 + f^*(r) b(r) L w_L)
\end{equation}
where $G(r)$ is the expected logarithmic growth rate. A more direct approach is to maximize the Kelly fraction $f^*(r)$ itself, as this is equivalent to maximizing the growth rate.

For practical implementation, we use a fractional Kelly approach to mitigate the risks of model error and high volatility. The target allocation fraction becomes:
\begin{equation}
f_{target}(r) = k \times f^*(r)
\end{equation}
where $k$ is the Kelly fraction (e.g., $k=0.5$ for Half Kelly, $k=0.25$ for Quarter Kelly).

\paragraph{Solution.}
As with the mean-variance approach, a closed-form solution is not available. We solve this numerically:
\begin{enumerate}
    \item For a given allocation ratio $r$, calculate the corresponding weights $w_L(r), w_S(r)$ and initial log-health $h_0(r)$.
    \item Compute the liquidation probability $q(r) = \pi(T^*)$ via numerical Laplace inversion.
    \item Calculate the win probability $p(r) = 1 - q(r)$.
    \item Calculate the net odds $b(r)$ based on the expected idiosyncratic returns and liquidation loss.
    \item Compute the full Kelly fraction $f^*(r)$.
    \item The optimization then searches for the ratio $r$ that maximizes $f^*(r)$, subject to the constraints $h_0(r) \geq h_{\min}$ and $0 < r < 1$. This can be done using a grid search or a numerical optimization routine like projected gradient ascent on $f^*(r)$.
\end{enumerate}

The optimal allocation ratio $r^*$ under the Kelly framework represents the trade-off that maximizes the long-term geometric growth rate of the portfolio, explicitly balancing the potential for high returns from leverage against the risk of ruin from liquidation. This provides a more direct and theoretically grounded approach to position sizing compared to utility-based methods.

\paragraph{Calibration and Market Data.}  
We calibrate the model using historical data from crypto assets with large idiosyncratic alpha differences. The long and short assets are chosen to enforce $\beta_P \approx 0$. Parameter estimation for the Lévy jump-diffusion model $(\mu_i, \sigma_i, \lambda_i, \delta_i)$ employs rolling windows of log-returns. Empirical results indicate that long assets have higher $\alpha_L$ and comparable or slightly higher $\sigma_L$ than short assets, and the jump parameters account for the tail asymmetry inherent in crypto markets.

\paragraph{Health Process Parameters.}
The log-health process \( h(t) \) quantifies the collateralization health of the leveraged long--short momentum portfolio. Its characteristic exponent is defined as
\[
\psi_h(\theta) = \psi_L(\theta) + \psi_S(-\theta),
\]
synthesizing the Lévy dynamics of the long winner and short loser legs (cf. Equation 38). This captures the joint jump-diffusion risk, where adverse jumps in the winner or favorable jumps in the loser drive the health process downward toward liquidation.

The asymptotic regime is determined by the drift
\[
\kappa_h = \frac{\mathbb{E}[h(t)]}{t}.
\]
A positive \(\kappa_h\) indicates an almost-sure (sustainable) regime, while a negative drift implies a defective regime where liquidation becomes probable. Calibration of \(\psi_h(\theta)\) and \(\kappa_h\) enables us to initialize the log-health \(h_0(r)\) via the leverage ratio \(r\) and simulate its stressed evolution, thereby endogenously linking liquidation risk to the relative performance of the momentum legs.

\paragraph{Numerical Evaluation.}  
For each calibrated parameter set, we compute the Kelly fraction $f^*(r)$. Laplace inversion is used to estimate $\pi(T) = \mathbb{P}(\tau_0 \le T^*)$. Gradients are approximated numerically, and $r^*$ is obtained using projected gradient ascent subject to feasibility constraints.

\paragraph{Parameter Values (Baseline).}
\begin{align*}
\mu_L &= 0.08, &\sigma_L &= 0.75, &\lambda_L &= 0.08, &\delta_L &= 0.20,\\
\mu_S &= 0.04, &\sigma_S &= 0.70, &\lambda_S &= 0.12,  &\delta_S &= 0.22,\\
T^* &= 30 \text{ days}, &h_{\min} &= 0.1.
\end{align*}

\paragraph{Results.}  
Simulation produces optimal allocation ratios $r^*$ as functions of the Kelly fraction $k$ and asset asymmetries. Key findings (Placeholder):
\begin{enumerate}
    \item Beta-neutral pairs exhibit lower variance but comparable expected returns to directional strategies.
    \item Strong idiosyncratic alpha differences yield positive drift in $h(t)$ and reduced liquidation probability.
    \item High correlation or elevated jump intensity reduces the feasible range of $r$ for maintaining $\beta_P \approx 0$ and $H(t)>1$.
\end{enumerate}

% ====================================================================
\section{Conclusion}
% ====================================================================

We have developed a practical framework for first-hitting time analysis in DeFi long-short positioning using constant-intensity jump-diffusion models. Although simpler than Hawkes processes, this approach provides robust semi-analytical solutions suitable for real-time risk management.

Key contributions include:
\begin{enumerate}
    \item Spectrally negative Lévy process formulation for log-health dynamics
    \item Semi-analytical computation via Laplace transforms and Gaver-Stehfest inversion
    \item Practical optimization framework balancing returns and liquidation risk
    \item Straightforward calibration methodology for constant-intensity parameters
\end{enumerate}

The framework demonstrates that sophisticated risk management in DeFi applications need not require computationally intensive models. The constant-intensity approach provides sufficient complexity to capture wrong-way risk while maintaining the analytical tractability necessary for practical implementation.

% ====================================================================
% APPENDIX
% ====================================================================

\appendix

\section{Mathematical Proofs}
\label{sec:proofs}

\begin{delayedproof}{prop:log_health_dynamics}
We prove this by applying Itô's lemma to the log-health function and carefully tracking jump contributions from both assets.

\textbf{Step 1: Setup and Itô decomposition}

Starting from $h(t) = \log(b_X w_X) + \log X(t) - \log(w_Y) - \log Y(t)$, define $f(x,y) = \log x - \log y$ so that $h(t) = \text{const} + f(X(t), Y(t))$.

Applying Itô's lemma to $f(X(t), Y(t))$ with jump-diffusion processes:
\begin{align}
df(X(t), Y(t)) &= \frac{\partial f}{\partial x} dX(t) + \frac{\partial f}{\partial y} dY(t) + \frac{1}{2}\frac{\partial^2 f}{\partial x^2} d[X]_t^c + \frac{1}{2}\frac{\partial^2 f}{\partial y^2} d[Y]_t^c \nonumber\\
&\quad + \frac{\partial^2 f}{\partial x \partial y} d[X,Y]_t^c + \sum_{\text{jumps}} \Delta f
\end{align}

\textbf{Step 2: Compute partial derivatives}

For $f(x,y) = \log x - \log y$:
\begin{align}
\frac{\partial f}{\partial x} &= \frac{1}{x}, \quad \frac{\partial f}{\partial y} = -\frac{1}{y}\\
\frac{\partial^2 f}{\partial x^2} &= -\frac{1}{x^2}, \quad \frac{\partial^2 f}{\partial y^2} = \frac{1}{y^2}, \quad \frac{\partial^2 f}{\partial x \partial y} = 0
\end{align}

\textbf{Step 3: Continuous part analysis}

The continuous parts of $X(t)$ and $Y(t)$ satisfy:
\begin{align}
dX(t) &= X(t^-)[\mu_X dt + \sigma_X dB_X(t)] + X(t^-)\Delta J_X(t)\\
dY(t) &= Y(t^-)[\mu_Y dt + \sigma_Y dB_Y(t)] + Y(t^-)\Delta J_Y(t)
\end{align}

For the continuous part (ignoring jumps temporarily):
\begin{align}
df^c(X(t), Y(t)) &= \frac{1}{X(t)}[\mu_X X(t) dt + \sigma_X X(t) dB_X(t)] - \frac{1}{Y(t)}[\mu_Y Y(t) dt + \sigma_Y Y(t) dB_Y(t)] \nonumber\\
&\quad - \frac{1}{2X(t)^2} \sigma_X^2 X(t)^2 dt + \frac{1}{2Y(t)^2} \sigma_Y^2 Y(t)^2 dt
\end{align}

The correct continuous part, accounting for correlation $\rho$, is:
\begin{align}
df^c(X(t), Y(t)) &= [\mu_X - \mu_Y - \frac{1}{2}(\sigma_X^2 + \sigma_Y^2 - 2\rho\sigma_X\sigma_Y)] dt \nonumber\\
&\quad + \sigma_X dB_X(t) - \sigma_Y dB_Y(t)
\end{align}

\textbf{Step 4: Jump part analysis}

When $X(t)$ jumps at time $T_i$ with size $\Delta J_X(T_i) = U_X^i = -(\delta_X + E_X^i)$:
\begin{align}
\Delta f|_{X\text{-jump}} &= \log(X(T_i^-)(1 + U_X^i)) - \log(X(T_i^-)) = \log(1 + U_X^i) = U_X^i
\end{align}

When $Y(t)$ jumps at time $T_j$ with $\Delta J_Y(T_j) = U_Y^j = +(\delta_Y + E_Y^j)$:
\begin{align}
\Delta f|_{Y\text{-jump}} &= -\log(1 + U_Y^j) = -U_Y^j
\end{align}

\textbf{Step 5: Combine results}

The total jump contribution to $h(t)$ is:
\begin{align}
J_h^X(t) &= \sum_{i=1}^{N_X(t)} U_X^i = \sum_{i=1}^{N_X(t)} [-(\delta_X + E_X^i)]\\
J_h^Y(t) &= \sum_{j=1}^{N_Y(t)} (-U_Y^j) = \sum_{j=1}^{N_Y(t)} [+(\delta_Y + E_Y^j)]
\end{align}

The diffusion coefficient becomes:
\[
\sigma_h^2 = \mathbb{V}[\sigma_X dB_X(t) - \sigma_Y dB_Y(t)] = \sigma_X^2 + \sigma_Y^2 - 2\rho\sigma_X\sigma_Y
\]

Therefore, $h(t) = h_0 + \mu_h t + \sigma_h B_h(t) + J_h^X(t) + J_h^Y(t)$ where all parameters are as stated.
\end{delayedproof}

\begin{delayedproof}{prop:laplace_exponent}
We derive the Laplace exponent by computing the cumulant generating function for each component of $h(t)$.

\textbf{Step 1: Decompose the log-health process}

From Proposition \ref{prop:log_health_dynamics}:
\[
h(t) = h_0 + \mu_h t + \sigma_h B_h(t) + J_h^X(t) + J_h^Y(t)
\]

\textbf{Step 2: Brownian motion contribution}

For $\sigma_h B_h(t)$, the contribution is:
\[
\psi_B(\theta) = \frac{1}{2}\sigma_h^2 \theta^2
\]

\textbf{Step 3: Drift contribution}

The drift contributes $-\mu_h \theta$ (negative sign for spectrally negative processes).

\textbf{Step 4: X-jump contribution}

For $J_h^X(t)$ with intensity $\hat\lambda_X$ and jump sizes $U_X^i = -(\delta_X + E_X^i)$:
\[
\psi_X(\theta) = \hat\lambda_X (\mathbb{E}[e^{\theta U_X}] - 1)
\]

Computing $\mathbb{E}[e^{\theta U_X}]$ where $U_X = -(\delta_X + E_X)$:
\begin{align}
\mathbb{E}[e^{\theta U_X}] &= e^{-\theta \delta_X} \mathbb{E}[e^{-\theta E_X}] = e^{-\theta \delta_X} \frac{\eta_X}{\eta_X + \theta}
\end{align}

\textbf{Step 5: Y-jump contribution}

For $J_h^Y(t)$, Y-jumps contribute negatively to health (they worsen it), so:
\[
\mathbb{E}[e^{\theta(-U_Y)}] = e^{-\theta \delta_Y} \frac{\eta_Y}{\eta_Y + \theta}
\]

\textbf{Step 6: Combine all contributions}

\[
\psi(\theta) = \frac{1}{2}\sigma_h^2 \theta^2 - \mu_h \theta + \hat\lambda_X \left( e^{-\theta \delta_X} \frac{\eta_X}{\eta_X+\theta} - 1 \right) + \hat\lambda_Y \left( e^{-\theta \delta_Y} \frac{\eta_Y}{\eta_Y+\theta} - 1 \right)
\]

The properties follow from direct computation: $\psi(0) = 0$, and 
\[
\psi'(0) = -\mu_h - \hat\lambda_X(\delta_X + 1/\eta_X) - \hat\lambda_Y(\delta_Y + 1/\eta_Y)
\]
\end{delayedproof}

\section{Numerical Methods and Implementation}
\label{sec:numerical_methods}

This appendix details the computational procedures for evaluating first-hitting time distributions through Laplace transform inversion.

\subsection{Gaver-Stehfest Inversion Algorithm}

The Gaver-Stehfest algorithm provides a robust method for numerically inverting Laplace transforms. For the first-hitting time distribution:

\begin{algorithm}
\caption{First-Hitting Time CDF via Gaver-Stehfest Inversion}
\label{alg:gaver_stehfest_appendix}
\begin{algorithmic}[1]
\STATE \textbf{Input:} Initial log-health $h_0 > 0$, evaluation time $T > 0$, Stehfest order $N$ (typically 10-14)
\STATE Compute adjustment coefficient $R = \Phi(0)$ by solving $\psi(R) = 0$
\STATE Precompute Stehfest weights $V_k$ for $k = 1, \ldots, N$:
\[
V_k = (-1)^{N/2+k} \sum_{j=\lfloor(k+1)/2\rfloor}^{\min(k,N/2)} \frac{j^{N/2} (2j)!}{(N/2-j)! j! (j-1)! (k-j)! (2j-k)!}
\]
\FOR{$k = 1$ to $N$}
    \STATE Compute $q_k = k \ln 2 / T$
    \STATE Solve $\psi(\theta) = q_k$ numerically to obtain $\Phi(q_k)$
    \STATE Evaluate $\mathcal{L}\{F\}(q_k) = e^{-h_0 \Phi(q_k)} / q_k$
\ENDFOR
\STATE Compute the inversion:
\[
F(T,h_0) = \frac{\ln 2}{T} \sum_{k=1}^{N} V_k \mathcal{L}\{F\}(q_k)
\]
\STATE Clamp result to $[0,1]$
\STATE \textbf{Output:} $F(T,h_0) = \mathbb{P}(\tau_0 \leq T)$
\end{algorithmic}
\end{algorithm}

For the conditional distribution in defective cases, replace step 8 with:
\[
\mathcal{L}\{F_{\mathrm{cond}}\}(q_k) = \frac{e^{-h_0 (\Phi(q_k+R) - R)}}{q_k}
\]

\subsection{Root Finding for $\Phi(q)$}

The root function $\Phi(q)$ requires solving $\psi(\theta) = q$ for each $q_k$ in the Stehfest algorithm. We employ Brent's method with search interval $[0, \theta_{\max}]$ where $\theta_{\max}$ is chosen to ensure $\psi(\theta_{\max}) > q_{\max}$.

\textbf{Implementation details:}
\begin{itemize}
    \item Initialize with bracketing: $\theta_{\min} = 0$, $\theta_{\max} = 10 \max(\eta_X, \eta_Y)$
    \item Verify $\psi(0) < q < \psi(\theta_{\max})$ before root finding
    \item Use relative tolerance $10^{-12}$ for convergence
    \item Cache $\psi(\theta)$ evaluations to avoid recomputation
\end{itemize}

\subsection{Numerical Stability Considerations}

\begin{itemize}
    \item \textbf{Stehfest order}: Use moderate orders (10-14) to balance accuracy and stability. Higher orders may amplify numerical errors.
    \item \textbf{High precision}: Employ extended precision arithmetic for large $N$ or small $T$ to avoid catastrophic cancellation.
    \item \textbf{Alternative inversion}: For very small $T$ or large $h_0$, consider Talbot or de Hoog methods which may be more stable.
    \item \textbf{Parameter validation}: Ensure $\eta_X + \theta > 0$ and $\eta_Y + \theta > 0$ to avoid singularities.
    \item \textbf{Clamping}: Final probabilities should be clamped to $[0,1]$ to handle numerical errors.
    \item \textbf{Conditioning detection}: Automatically detect defective hitting by checking if $\psi'(0) < 0$.
\end{itemize}

\subsection{Computational Complexity}

For each distribution evaluation:
\begin{itemize}
    \item $O(N)$ Laplace transform evaluations (typically $N = 10-14$)
    \item Each evaluation requires solving $\psi(\theta) = q_k$ via root finding
    \item Root finding: $O(\log \epsilon^{-1})$ iterations for tolerance $\epsilon$
    \item Total complexity: $O(N \log \epsilon^{-1})$ per distribution evaluation
\end{itemize}

% ====================================================================
% BIBLIOGRAPHY
% ====================================================================

%\bibliographystyle{abbrvnamed}
%\bibliography{finance}

\end{document}