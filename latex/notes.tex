%%
%% $Id: article.tex,v 1.1 2008/09/20 10:19:28 natalie Exp $
%% $Source: /Users/natalie/cvs/tex/templates/article.tex,v $
%% $Date: 2008/09/20 10:19:28 $
%% $Revision: 1.1 $
%%

%\documentclass[a4paper,11pt,BCOR1cm,DIV11,headinclude]{scrbook}
% bei 12pt ist DIV 12 default, bei 11pt ist es DIV 10
% Textbereiche 
% DIV 10: 147*207.9mm, DIV 11: 152.73*216mm, DIV 12:157.50*222.75
% DIV 13: 161.54*228.46mm, DIV 14: 165*233.36mm

\def\deftitle{Notes on Dynamics for cryptocurrency derivatives}
% \def\defauthor{N.\ Packham}
% \def\defauthor{nat}
\def\defauthor{}

%% option: largefont
\documentclass[square]{article} %
%% options: vscreen, garamond, wnotes, savespace
\usepackage[vscreen]{nat}
\usepackage[longnamesfirst,sort&compress]{natbib}
\usepackage{booktabs}

\bibpunct{(}{)}{;}{a}{,}{,}
\usepackage{amsfonts,amssymb,amsthm} %
\usepackage{mathrsfs}
\usepackage[tbtags]{amsmath} %
\usepackage{bm}
\usepackage{bbm}
\usepackage{tabularx,ragged2e}
\usepackage[table,xcdraw]{xcolor}
\usepackage{subfig}
\usepackage{enumitem}
\usepackage{csquotes}


\newcolumntype{C}{>{\Centering\arraybackslash}X}
\newcolumntype{s}{>{\hsize=.2\hsize \Centering\arraybackslash}X}
% \usepackage{fullpage}
\usepackage{footnote}
\makesavenoteenv{tabular}

\usepackage{cleveref}
\newenvironment{delayedproof}[1]
 {\begin{proof}[\raisedtarget{#1}Proof of \Cref{#1}]}
 {\end{proof}}
\newcommand{\raisedtarget}[1]{%
  \raisebox{\fontcharht\font`P}[0pt][0pt]{\hypertarget{#1}{}}%
}
\newcommand{\proofref}[1]{\hyperlink{#1}{proof}}


\usepackage{graphicx,color}
\graphicspath{{./pics/}}
\definecolor{BrickRed}{rgb}{.625,.25,.25}
\providecommand{\red}[1]{\textcolor{BrickRed}{#1}}
\definecolor{markergreen}{rgb}{0.6, 1.0, 0}
\definecolor{darkgreen}{rgb}{0, .5, 0}
\definecolor{darkred}{rgb}{.7,0,0}
\definecolor{darkorange}{rgb}{1,0.3,0}
\definecolor{darkblue}{rgb}{0,29,245}
%\definecolor{orange}{rgb}{239, 133, 54}
%\definecolor{lightblue}{rgb}{59, 188, 175}

\providecommand{\marker}[1]{\fcolorbox{markergreen}{markergreen}{{#1}}}
\providecommand{\natp}[1]{\textcolor{darkred}{#1}}
\providecommand{\mj}[1]{\textcolor{darkred}{#1}}
\providecommand{\francis}[1]{\textcolor{darkgreen}{#1}}

\theoremstyle{plain}
\newtheorem{theorem}{Theorem}%[section]
\newtheorem{proposition}[theorem]{Proposition}
\newtheorem{corollary}[theorem]{Corollary} %%
\newtheorem{lemma}[theorem]{Lemma} %%
\theoremstyle{definition} %%
\newtheorem{definition}{Definition}
\newtheorem{remark}[theorem]{Remark}
\newtheorem{remarks}{Remarks}
\newtheorem{condition}[theorem]{Condition}
\newtheorem{example}[theorem]{Example}
\newtheorem{assumption}{Assumption}
\setlength{\parindent}{0pt}

\usepackage{amsmath}
%%
%% Mathematical Definitions for Interest Rate Theory
%%

%% GENERAL SETTINGS
\unitlength1cm

%% =============================================================================
%% BASIC MATHEMATICAL SETS AND SPACES
%% =============================================================================
%% =============================================================================
%% PROBABILITY MEASURES AND EXPECTATIONS
%% =============================================================================
\newcommand{\E}{{\mathbb{\sf E}}}
\newcommand{\P}{{\mathbb{P}}}
\newcommand{\p}{{\bf P}}
\newcommand{\q}{{\bf Q}}

\providecommand{\R}{{\mathbb{R}}}
\providecommand{\N}{{\mathbb N}}
\def\Z{{\mathbb Z}}
\def\Q{{\mathbb Q}}
\def\I{{\mathbb I}}
\def\M{{\mathbb M}}
\newcommand{\T}{{\mathbb{T}}}
\newcommand{\Fb}{{\mathbb{F}}}
%% Conditional expectations under various measures
\newcommand{\Eq}{{\mathbb{E}}_{{\bf Q}}}
\newcommand{\Eqn}{{\mathbb{E}}_{{\bf Q}_N}}
\newcommand{\Eqm}{{\mathbb{E}}_{{\bf Q}_M}}
\newcommand{\EqT}{{\mathbb{E}}_{{\bf Q}_T}}
\newcommand{\EqTe}{{\mathbb{E}}_{{\bf Q}_{T_1}}}
\newcommand{\EqTz}{{\mathbb{E}}_{{\bf Q}_{T_2}}}
\newcommand{\EqSe}{{\mathbb{E}}_{{\bf Q}_{S^1}}}
\newcommand{\EqSz}{{\mathbb{E}}_{{\bf Q}_{S^2}}}
%% Almost sure convergence notation
\newcommand{\pas}{\text{{\bf P}--a.s.}}
\newcommand{\paa}{\text{{\bf P}--a.a.}}
\newcommand{\qas}{\text{{\bf Q}--a.s.}}

%% Specific probability measures
\newcommand{\qn}{{\bf Q}_N}
\newcommand{\qm}{{\bf Q}_M}
\newcommand{\qT}{{\bf Q}_T}
\newcommand{\qTe}{{\bf Q}_{T_1}}
\newcommand{\qTz}{{\bf Q}_{T_2}}
\newcommand{\qS}{{\bf Q}_S}
\newcommand{\qSe}{{\bf Q}_{S^1}}
\newcommand{\qSz}{{\bf Q}_{S^2}}
\newcommand{\qs}{{\q_{\rm Swap}}}

%% Miscellaneous
\newcommand{\e}{{\bf e}}
%% =============================================================================
%% FILTRATIONS AND SIGMA-ALGEBRAS
%% =============================================================================
\newcommand{\F}{{\cal F}}
\newcommand{\G}{{\cal G}}
\newcommand{\A}{{\cal A}}
\newcommand{\Hc}{{\cal H}}
\renewcommand{\H}{\ensuremath{\mathcal H}}
\def\filtration#1{{\ensuremath\mathcal{#1}}}
\providecommand{\Fsigma}{\ensuremath{\mathcal \F_\infty^\sigma}}
%% =============================================================================
%% DIFFERENTIAL NOTATION
%% =============================================================================
\newcommand{\dP}{{\rm d}{\bf P}}
\newcommand{\du}{{\rm d}u}
\newcommand{\dd}{{\rm d}}
\newcommand{\df}{{\rm \bf DF}}
\providecommand{\dx}{\ensuremath{\frac{\partial}{\partial x}}}
\providecommand{\dt}{\ensuremath{\frac{\partial}{\partial t}}}
\providecommand{\dy}{\ensuremath{\frac{\partial}{\partial y}}}
%% =============================================================================
%% STATISTICAL AND FINANCIAL NOTATION
%% =============================================================================
\providecommand{\Ncdf}{{\rm N}}
\newcommand{\n}{{\rm n}}
\newcommand{\emb}{\bf \em}
\newcommand{\1}{\textbf{1}}
\newcommand{\fx}{{\rm fx}}
\newcommand{\V}{{\rm Var}}
\providecommand{\var}{\ensuremath{\text{Var}}}
\providecommand{\cov}{\ensuremath{\text{Cov}}}
\newcommand{\Om}{{\Omega}}
%% =============================================================================
%% MATHEMATICAL OPERATORS AND FUNCTIONS
%% =============================================================================
\providecommand{\limn}{\ensuremath{\lim_{n\rightarrow\infty}}}
\providecommand{\qv}[2]{\ensuremath{\langle #1,#1\rangle_{#2}}}
\newcommand{\argmax}{\operatornamewithlimits{argmax}}
\newcommand{\argmin}{\operatornamewithlimits{argmin}}
\providecommand{\vec}[1]{\ensuremath{\bm #1}}
\providecommand{\vecb}[1]{\ensuremath{\bm #1}}
\providecommand{\abs}[1]{\ensuremath{\lvert#1\rvert}}
\providecommand{\norm}[1]{\ensuremath{\lVert#1\rVert}}

%% =============================================================================
%% THEOREM ENVIRONMENTS
%% =============================================================================
\ifx\prop\undefined
\newtheorem{prop}{Proposition}[section]
\fi
\newtheorem{theo}[prop]{Theorem}
\newtheorem{lem}[prop]{Lemma}
\newtheorem{cor}[prop]{Corollary}
\newtheorem{defi}[prop]{Definition}

%% =============================================================================
%% LIST FORMATTING
%% =============================================================================
\providecommand{\labelenumi}{{\rm (\roman{enumi})}}
\setlength{\labelsep}{0.3cm}
\setlength{\leftmargin}{10cm}
\setlength{\labelwidth}{5cm}

%% =============================================================================
%% STOCHASTIC PROCESS TERMINOLOGY
%% =============================================================================
\providecommand{\cadlag}{c\`adl\`ag }
\providecommand{\cadlagns}{c\`adl\`ag}
\providecommand{\caglad}{c\`agl\`ad }
\providecommand{\cad}{c\`ad}
\providecommand{\cag}{c\`ag}
\providecommand{\levy}{L\'evy\ }
\providecommand{\levyns}{L\'evy}
\providecommand{\levyito}{L\'evy-It\^o\ }
\providecommand{\levykhinchin}{L\'evy-Khinchin\ }
\providecommand{\ito}{It\^o }
\providecommand{\itos}{It\^o's\, }
%% =============================================================================
%% FUNCTION SPACES
%% =============================================================================
\providecommand{\D}{\ensuremath{D(\R_+,\R)}}
\providecommand{\Dsig}{\ensuremath{D(\R_+, \R_+\setminus\{0\}})}
\providecommand{\Dd}{\ensuremath{D(\R_+,\R^d)}}
\providecommand{\C}{\ensuremath{C(\R_+,\R)}}
\providecommand{\Cd}{\ensuremath{C(\R_+,\R^d)}}
\providecommand{\rpos}{\ensuremath{{[0,\infty)}}}}

\def\Mc{{\mathcal M}}
\def\tp{\tilde{\p}}
%% =============================================================================
%% MEASURE THEORY AND INTEGRATION
%% =============================================================================
\providecommand{\borel}[0]{\ensuremath{\mathcal{B}}}
\providecommand{\intinf}[0]{\ensuremath{\int_{-\infty}^\infty}}
\providecommand{\intpos}[0]{\ensuremath{\int_0^\infty}}
\providecommand{\intneg}[0]{\ensuremath{\int_{-\infty}^0}}
\providecommand{\dynkin}[0]{\ensuremath{\mathcal D}}

%% =============================================================================
%% CONDITIONAL EXPECTATION AND PROCESSES
%% =============================================================================
\providecommand{\ce}[2]{\ensuremath{\E(#1|\filtration{#2})}}
\providecommand{\inv}[1]{\ensuremath{#1^{(-1)}}}
\providecommand{\os}[2]{\ensuremath{#1^{(#2)}}}
\providecommand{\pos}[2]{\ensuremath{h_{#1}(#2)}}
\providecommand{\poslong}[3]{\ensuremath{h_{#1, #2}(#3)}}
\providecommand{\variation}[2]{\ensuremath{\rm V_{#1}(#2)}}

%% =============================================================================
%% UTILITY COMMANDS
%% =============================================================================
\providecommand{\todo}[1]{\footnote{#1}}

%% =============================================================================
%% PROCESS CLASSES AND STOCHASTIC INTEGRATION
%% =============================================================================
%% Classes of stochastic processes
\providecommand{\classfv}{\ensuremath{\mathscr V}}
\providecommand{\classv}{\ensuremath{\mathscr V}}
\providecommand{\classh}{\ensuremath{\mathscr H^2}}
\providecommand{\classhloc}{\ensuremath{\mathscr H^2_{\rm loc}}}
\providecommand{\classm}{\ensuremath{\mathscr M}}
\providecommand{\classmloc}{\ensuremath{\mathscr M_{\rm loc}}}
\providecommand{\classl}{\ensuremath{L^2}}
\providecommand{\classlloc}{\ensuremath{L^2_{\rm loc}}}
\providecommand{\classa}{\ensuremath{\mathscr A}}
\providecommand{\classaloc}{\ensuremath{\mathscr A_{\rm loc}}}
\providecommand{\classalocpos}{\ensuremath{\mathscr A_{\rm loc}^+}}
\providecommand{\classp}{\ensuremath{\mathscr P}}
\providecommand{\classo}{\ensuremath{\mathscr O}}
\providecommand{\classs}{\ensuremath{\mathscr S}}
\providecommand{\classsp}{\ensuremath{\mathscr S_p}}
\providecommand{\classu}{\ensuremath{\mathscr U}}
\providecommand{\nullset}{\ensuremath{\mathscr N}}

%% Stochastic integration
\providecommand{\stint}{\ensuremath{\cdotp}}

%% =============================================================================
%% FINANCE-SPECIFIC NOTATION
%% =============================================================================
%% CPO distribution
\providecommand{\cpo}{\ensuremath{{\rm CPO}}}
\providecommand{\sigd}{\ensuremath{\mathscr D}}

%% Credit spreads
\providecommand{\s}{{\bf s}}

%% State spaces
\providecommand{\sX}{\ensuremath{\mathcal X}}
\providecommand{\sY}{\ensuremath{\mathcal Y}}

\sloppy
\begin{document}
\setlength{\boxlength}{0.95\textwidth} %
\title{\large{\bf\deftitle}} %
\author{{\normalsize\bf\defauthor}}%
\thispagestyle{empty}
\addtocounter{page}{1}
\maketitle
\begin{abstract}
 To be filled. 
\end{abstract}
% \keywords{keywords here} %%
% \jel{jel here} %%
\vspace{.5cm}
\def\contentsname{Contents}
\tableofcontents
%%
\vspace{.5cm}
\section{Ideas}
\subsection{Uncovered interest parity in the absence of bond}
This section is mostly taken/copied from section 6.1 of \cite{gudgeon2020defi} for idea generation. 
Uncovered interest parity (UIP) normally appear in the context of foreign exchange between two countries: domestic and foreign. 
An investor has the choice of whether to hold domestic or foreign assets. 
If the condition of UIP holds, a risk-neutral investor should be indifferent between holding the domestic or foreign assets because the exchange rate is expected to adjust such that returns are equivalent. \\

Example
An investor starting with 1m GBP at $t=0$ could either:
\begin{itemize}
  \item receive an annual interest rate of $i_\text{GBP}=3\%$, resulting in $1.03$m GBP at $t=1$
  \item or, immediately buy $1.23$ USD at an exchange rate $S_{\text{GBP}/\text{USD}}=0.813$, and receive an annual interest rate of $i_\text{USD}=5\%$, resulting in 1.2915m USD at $t=1$. 
  Then, convert the USD with the exchange rate at $t=1$, say $S_{\text{GBP}/\text{USD}}=0.7974$, and get $1.03$m GBP.
\end{itemize}

If UIP holds, despite the higher interest rate of the USD, the investor will be indifferent because the exchange rate between currencies offset the spread between interest rates. 
Mathematically, UIP is stated as 
\begin{align*}
1+R^{(i)} = (1+R^{(j)})\frac{\mathbb{\sf E}S_{t+k}}{S_t},
\end{align*}
where $R^{(i/j)}$ is the interest rate payable on asset $i/j$ from time $t$ to $t+k$, and $S_t$ is the exchange rate ar time $t$. \\

Now the question is both $R^{(i/j)}$ are not known in advance due to the lack of a liquid bond market that investors can secure the future payoff by holding a cryptocurrency. 
However, the good news is that we have the observable historical short rate, $r_t^{(i)}$, and exchange rate $S_t^{(i/j)}$. 
The UIP condition in this case have to be adjusted to incorporate the fact that the investor consider also the uncertainty of the domestic and foreign short rate, i.e.\footnote{The AAVE and Compound interest rate are compounded every second, which is close enough to model the short rate payoff in a continuous compounding scheme. }

\begin{align*}
  \mathbb{\sf E} \left(\exp{\int_0^T r^{(i)}_t \text{d}t}\right) = \mathbb{\sf E}\left(\exp\left(\int_0^T r^{(j)}_t\text{d}t\right)\frac{S_{T}}{S_t}\right).
  \end{align*}

The above UIP condition open quite some questions
\begin{enumerate}
  \item It seems necessary to model the joint dynamics of the foreign short rate and the exchange rate, such that the R.H.S. of the above equation can be evaluated.
  \item Under which measure should we take the expectation of both sides? Any criteria of choosing the measure? No-arbitrage? 
  \item When will the above equation hold? When will not?
  \item If the condition does not hold, are there any arbitrage opportunities? 
  \item What is the dynamics of $r^{(i/j)}$? For further development, we might want a parametrised stochastic model such that we can
   (i) perform measure change, (ii) price bonds, swaps, or any other derivatives easily, (iii) capture the interest rate dynamics nicely.
  \item Another idea: Can we use machine learning method to get a good enough estimate of the L.H.S. and R.H.S. separately, and make a statistical argument over the difference between L.H.S. and R.H.S.?
  \begin{align*}
    \hat{\mathbb{\sf E}} \left(\exp{\int_0^T r^{(i)}_t \text{d}t}\right) = \hat{\mathbb{\sf E}}\left(\exp\left(\int_0^T r^{(j)}_t\text{d}t\right)\frac{S_{t+k}}{S_t}\right) + \varepsilon_{t, k}.
  \end{align*}

\end{enumerate}

\subsection{A crypto interest rate model}
This idea is motivated by a conversation in an interview with a team of quant working for a centralised crypto exchange (CEX).
 The pricing of crypto derivatives requires a "risk-free" rate for each underlying,
 e.g. a "risk-free" rate of holding BTC, another "risk-free" rate of holding ETH. \\
 
I put quotation marks for the term risk-free rate because there is no consensus to what is the risk-free rate in the crypto market. 
 In practice, there are four ways of getting a risk-free rate by viewing the risk-free rate as the opportunity cost of investing the 
 holdings to the derivative market instead of earning a risk-free rate in another market. 
 The four ways of getting a risk-free are referring to calibrating the risk-free rate to three different markets:
 \begin{enumerate}
  \item DeFi lending protocols, e.g. AAVE and Compound,
  \item DeFi staking protocols, e.g. ETH POS, AQRU,
  \item Futures traded on CEX,
  \item Perpetual futures traded on CEX.
 \end{enumerate} 

The first two markets are not exactly risk-free since trading on DeFi protocols expose oneself to platform risk.
 Although DeFi lending protocols enforce over-collateralisation, 
 lenders still expose themselves to gap risk since there is a chance that the collateral value jump through the value of the borrowings and there is no way to force the borrower to pay back.\\ 

The third and forth markets suffer from the platform risk. 
 However, if a trader manage/hedge her exposure on CEX with other instruments traded on the same CEX, 
 then we can argue that the platform risk is somehow offset. 
 The perpetual futures market does not exempt carry trade traders from losses since funding rate can go negative. 
 It turns out, after considering the pros and cons of the markets, 
 the futures market traded on CEX is commonly used by the team of quant to get the risk-free rate to price their derivatives, which is also a common practice in the traditional market.  \\

The key idea here is to have a CEX futures market centric view, i.e. considering the CEX futures market as risk-free,
 and to study the differences of the rate dynamics among the markets. 
 In order to do so, a common interest rate model for crypto is needed. 
 It will be a challenge to have a single interest rate model that fits the rate dynamics in all the markets.
 For example, the DeFi lending/staking rate is always positive, but the funding rate of perpetual futures can go negative.  
 A possible solution is to discount the DeFi lending/staking rates by the corresponding default intensity.

\subsection{A model for perpetual futures}
Suppose the price of the perpetual futures, $F_t$, is reverting to its underlying, $S_t$, with some Brownian noise involved, i.e.
\begin{align*}
\dd F_t = \kappa \left(\delta S_t - F_t \right)\dd t + \epsilon \dd W_t,
\end{align*}
where $\kappa>0$ is the mean reverting speed, $\delta \in \mathbb{R}$ is the spot-futures spread, and $\epsilon>0$ is the volatility.

We derive the solution of the perpetual futures by applying Ito's lemma to $e^{\kappa t}F$:

\begin{align*}
\dd e^{\kappa t} F_t &= \left(\kappa e^{\kappa t}F_t+ \kappa \left(\delta S_t - F_t\right)e^{\kappa t}\right)\dd t + \epsilon e^{\kappa t}\dd W_t\\
                      &= \kappa \delta S_t e^{\kappa t}\dd t + \epsilon e^{\kappa t}\dd W_t.
\end{align*}

The solution is obtained by integrating both sides from time $0$ to $T$,
\begin{align*}
  e^{\kappa T} F_T &= F_0 +  \kappa \delta \int_0^T S_t e^{\kappa t}\dd t + \int_0^T \epsilon e^{\kappa t}\dd W_t.
  \end{align*}

The expected value of $F_T$ under a risk-neutral measure $\mathbb{Q}$ is
\begin{align*}
  e^{\kappa T} \mathbb{\sf E}_\mathbb{Q} \left(F_T\right) &= F_0 +  \kappa \delta \int_0^T \mathbb{\sf E}_\mathbb{Q}\left(S_t\right) e^{\kappa t}\dd t\\
                                                          &= F_0 +  \kappa \delta e^{rT}S_0 \int_0^T  e^{\kappa t}\dd t\\
                                                          &= F_0 +  \delta e^{rT}S_0 \left(e^{\kappa T} - 1\right)\\
               \mathbb{\sf E}_\mathbb{Q} \left(F_T\right) &= e^{-\kappa T}\left(F_0 - \delta e^{rT}S_0\right)  + \delta e^{rT}S_0.                              
\end{align*}

Suppose $e^{-rt} F_t$ is a martingale under $\mathbb{Q}$ (i.e. $\mathbb{\sf E}_\mathbb{Q} \left(F_T\right)= e^{rT}F_0$ ), 
\begin{align*}
  e^{r T} F_0 &= e^{-\kappa T}\left(F_0 - \delta e^{rT}S_0\right)  + \delta e^{rT}S_0.
\end{align*}

By such, we work out the formula for the spread $\delta$:
\begin{align*}
  \delta &= \frac{F_0}{S_0}\frac{\left(e^{rT}-e^{-\kappa T}\right)}{e^{rT}\left(1-e^{-\kappa T}\right)}\\
         &= \frac{F_0}{S_0}\frac{\left(1-e^{-(\kappa+r) T}\right)}{1-e^{-\kappa T}}
\end{align*}

Observations
\begin{enumerate}
\item The spread $\delta $ depends on $\kappa$, $r$ (risk-free rate), and $T$ (can be considered as maturity)
\item If $r > 0$, then $\delta$ is a time-increasing function
\begin{align*}
\frac{\partial \delta}{\partial T} &= \frac{F_0}{S_0}\frac{T e^{-r T}}{e^{\kappa T}-1}>0
\end{align*}
\item If $r=0$, then $\delta  = F_0/S_0$, meaning the spread is no longer depending on maturity
\item If we send $T\rightarrow \infty$, then $\delta = F_0/S_0$ asymptotically. 
\end{enumerate}
Now, say the spread is a time-dependant deterministic function with a form
\begin{align*}
\delta(t) = Ce^{dt},
\end{align*}
where $C>0$ and $d \in \mathbb{R}$ are constants to be determined. One can think of $d$ as the rate of divergence between the spot and futures. 

The price dynamics of the perpetual futures is 
\begin{align*}
  \dd F_t = \kappa \left(\delta(t) S_t - F_t \right)\dd t + \epsilon \dd W_t.
  \end{align*}

Again, by apply Ito's lemma to $e^{\kappa t}F_t$ and integration, we get the solution of $F_T$
\begin{align*}
e^{\kappa T}F_T & = F_0 + \kappa C \int_0^T S_t e^{(\kappa + d) t}\dd t + \int_0^T \epsilon e^{\kappa t}\dd W_t.
\end{align*}

Taking expectation under $\mathbb{Q}$ of both sides yields, 
\begin{align*}
  e^{\kappa T}\mathbb{\sf E}_\mathbb{Q}\left(F_T\right) & = F_0 + \kappa C S_0 e^{rT} \int_0^T e^{(\kappa + d) t}\dd t\\
  e^{(\kappa + r) T}F_0 & = F_0 + \kappa C S_0 e^{rT} \frac{e^{(\kappa + d) T} - 1}{\kappa + d}
\end{align*}

\subsection{Connection between the volatility of spot and level of borrowing rate}
This is motivated by the observation that stable coins have higher borrowing rate, and other coins have lower. \\

Say $B^{(i/j)}(T) = \exp\int_0^T r^{(i/j)}_t \dd t$ are the lending accounts levels if one lend out $i/j$ coins, 
 the investor should be indifference in investing to $i$ or $j$ if the following condition holds.
 \begin{align*}
  \mathbb{\sf E}\left(B^{(i)}(T)\right) 
  &= \mathbb{\sf E}\left(B^{(j)}(T)S_T/S_0\right)\\
  &= \mathbb{\sf Cov}\left(B^{(j)}(T),\ S_T/S_0\right) + \mathbb{\sf E}\left(B^{(j)}(T)\right)\mathbb{\sf E}\left(S_T/S_0\right)\\
  &= \mathbb{\sf Corr}\left(B^{(j)}(T),\ S_T/S_0\right) \sigma\left(B^{(j)}(T)\right) \sigma\left(S_T/S_0\right) + \mathbb{\sf E}\left(B^{(j)}(T)\right)\mathbb{\sf E}\left(S_T/S_0\right).
\end{align*}

Observation: If $\sigma\left(S_T/S_0\right)$ went up, $\sigma\left(B^{(j)}(T)\right)$ must go down for the equation to hold (while keeping all other variables fixed).

\section{Literatures}
\subsection{Uncovered Interest Parity and its variants}
\cite{cappiello2007uncovered}
\begin{enumerate}
  \item This paper proposes an extension of UIP called the Uncovered Return Parity (URP)
  \item The URP condition is 
  \begin{align*}
    \mathbb{\sf E}\left(R_{t+1}\frac{S_{t+1}}{S_t}m_{t+1}\Big|\mathcal{F}_t\right)=1,
  \end{align*}
  where $R_{t+1}$ is the gross return on a foreign asset denominated in a foreign currency, and $S_{t+1}$ is the spot exchange rate, defined as the number of units of domestic currency exchanged for one unit of foreign currency. 
  \item The R.H.S. (=1) of the above equation stemmed from definition of stochastic discount factor, see Section 3.1 of \cite{back2010asset}. 
  \item Then the authors assume that there exist a foreign risk-free bond (which we do not have that in the cryptomarket) and yield the following
  \begin{align*}
    \mathbb{\sf E}\left(\frac{S_{t+1}}{S_t}m_{t+1}\Big|\mathcal{F}_t\right) = \frac{1}{R_{f,t}}.
  \end{align*}
  \item The remaining paper is about estimation of URP. The authors estimate the SDF via GMM. 
\end{enumerate}

\subsection{Affine term structure models}
\cite{anderson2010affine}
\begin{enumerate}
  \item The paper extends the affine class of term structure models to describe the joint dynamics of exchange rates and interest rates
\end{enumerate}

\subsection{Interest rate derivatives in the crypto space}
{\bf Inter-Protocol Offered Rate (IPOR)} \url{https://docs.ipor.io/}
\begin{enumerate}
  \item The IPOR company offers an interest rate benchmark (weighted average of DeFi interest rate) with the same name that 
  summarizes the lending and borrowing interest rates of crypto loan platforms.
  \item The company offers trading of fixed income derivatives, e.g. interest rate swap. 
  The pricing and transaction are based on the IPOR rate and automated market maker. 
  \item The main derivatives traded on IPOR is cancellable interest rate swaps.
  \item IPOR uses Hull-White jump-diffusion model for rate simulation and Longstaff-Schwartz method for pricing (the cancellable part)
  \item Criticisms
  \begin{enumerate}
    \item Complex product designs: cancellable swaps + Hull-White-jump-model + Longstaff-Schwartz 
    \item Lack of market-involved pricing mechanism: the AMM takes the spread calculated only by the Hull-White model + Longstaff-Schwartz. 
    \item Spread calculation often results in high spread that prohibits transactions. 
    \item Max tenor is too short (90-day longest) 
    \item See \url{https://scapital.medium.com/ipor-a-postmortem-for-the-interest-rate-swap-pioneer-5dc8492c2f7c}
  \end{enumerate}
\end{enumerate}




\section{Backgrounds}
\subsection{Short rate models}
A quote from Section 10.1 of \cite{Shreve2004a} nicely summarise what is short rate traditionally:
\begin{displayquote}
  "The interest rate (sometimes called the short rate) is an {\color{red} idealization} corresponding to the shortest maturity yield or perhaps the overnight rate offered by the government, depending on the particular application."
\end{displayquote}

\begin{remark}In the crypto world, instead of using liquidly traded bonds and fixed-income products to infer the short rate (mainly its dynamics) and price more complex products,
 the short rate itself is directly observable and impacts the growth rate of borrowing and lending accounts,
 i.e. the growth of crypto borrowing and lending accounts is a {\color{red} realization} of the crypto short rate.
 Beware that {\it all} borrowings in a lending pool, disregard of the identity of borrower and starting date of the borrowings (there is no maturity in crypto loans),
 are growing at the {\it{same}} crypto borrowing rate. \end{remark}

The simplest model for fixed income markets begin with a stochastic differential equation for the interest rate
\begin{align*}
  \dd r_t = \beta(t, r_t)\dd t + \gamma(t, r_t)\dd W_t.
\end{align*}

The zero-coupon bond pricing formula is
\begin{align*}
B(t,T) = \E_\mathbb{Q}\left[
e^{-\int_t^T r_s \dd s}\big|\mathcal{F}_t
\right],
\end{align*}
provided that $B(T,T)=1$. 

Since $r_t$ is governed by a SDE, it is a Markov process and we must have
\begin{align*}
B(t,T) = f(t, r_t)
\end{align*}

By Feynman-Kac, we obtain the partial differential equation
\begin{align*}
\partial_t f(t,r) + \beta(t,r)\partial_r f(t,r)+ \frac{1}{2}\gamma^2(t,r)\partial_{rr}f(t,r) = r f(t,r),\  
f(T, r)=1.
\end{align*}

{\bf Hull-White model}\\
In the Hull-White model, the evolution of the interest rate is given by 
\begin{align*}
\dd r_t = \left(a(t)-b(t)r_t\right)\dd t \sigma(t) \dd W_t,
\end{align*}
where $a(t)$, $b(t)$, and $\sigma(t)$ are non-random positive functions of time.
The PDE for the zero-coupon bond price becomes
\begin{align*}
  \partial_t f(t,r) + \left(a(t)-b(t)\right)\partial_r f(t,r)+ \frac{1}{2}\sigma^2(t)\partial_{rr}f(t,r) = r f(t,r),\  
  f(T, r)=1.
  \end{align*}
By ansatz, the solution of the above PDE has a form 
\begin{align*}
f(t,r) = \exp\left(-r C(t,T)- A(t,T)\right).
\end{align*}

Let's work out the partial derivatives:
\begin{align*}
  \partial_t f &= \left(-r \partial_t C(t,T) - \partial_t A(t,T)\right) f\\
  \partial_r f &= -C(t,T)f\\
  \partial_{rr} f &= C^2(t,T)f.
\end{align*}

Substitute into the PDE gives
\begin{align*}
0=\Big[
&\left(
-\partial_t C+ b(t)C-1
\right)r\\
&-\partial_t A - a(t)C+\frac{1}{2}\sigma^2(t)C
\Big]f(t,r).
\end{align*}
 
Since $f$ is nonzero and this equation must hold for all $r$, 
\begin{align*}
  -\partial_t C+ b(t)C-1 &=0\\
  -\partial_t A - a(t)C+\frac{1}{2}\sigma^2(t)C &=0.
\end{align*}

Since $f(r,T)=1$, $C(T,T)=A(T,T)=0$. The solution of the above equations is 
\begin{align*}
  C(t,T) &= \int_t^T \exp\left(-]int_t^s b(v)\dd v\right)\dd s\\
  A(t,T) &= \int_t^T \left(a(s) C(s,T)-\frac{1}{2}\sigma^2(s)C^2(s,T)\right)\dd s.
\end{align*}


\bibliographystyle{abbrvnamed} %
\bibliography{finance} %
\end{document}

%%% Local Variables: 
%%% mode: latex
%%% TeX-master: t
%%% End: 
