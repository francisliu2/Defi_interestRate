%%
%% $Id: article.tex,v 1.1 2008/09/20 10:19:28 natalie Exp $
%% $Source: /Users/natalie/cvs/tex/templates/article.tex,v $
%% $Date: 2008/09/20 10:19:28 $
%% $Revision: 1.1 $
%%

%\documentclass[a4paper,11pt,BCOR1cm,DIV11,headinclude]{scrbook}
% bei 12pt ist DIV 12 default, bei 11pt ist es DIV 10
% Textbereiche 
% DIV 10: 147*207.9mm, DIV 11: 152.73*216mm, DIV 12:157.50*222.75
% DIV 13: 161.54*228.46mm, DIV 14: 165*233.36mm

\def\deftitle{Notes on Dynamics for cryptocurrency derivatives}
% \def\defauthor{N.\ Packham}
% \def\defauthor{nat}
\def\defauthor{}

%% option: largefont
\documentclass[square]{article} %
%% options: vscreen, garamond, wnotes, savespace
\usepackage[vscreen]{nat}
\usepackage[longnamesfirst,sort&compress]{natbib}
\usepackage{booktabs}

\bibpunct{(}{)}{;}{a}{,}{,}
\usepackage{amsfonts,amssymb,amsthm} %
\usepackage{mathrsfs}
\usepackage[tbtags]{amsmath} %
\usepackage{bm}
\usepackage{bbm}
\usepackage{tabularx,ragged2e}
\usepackage[table,xcdraw]{xcolor}
\usepackage{subfig}
\usepackage{enumitem}
\usepackage{csquotes}


\newcolumntype{C}{>{\Centering\arraybackslash}X}
\newcolumntype{s}{>{\hsize=.2\hsize \Centering\arraybackslash}X}
% \usepackage{fullpage}
\usepackage{footnote}
\makesavenoteenv{tabular}

\usepackage{cleveref}
\newenvironment{delayedproof}[1]
 {\begin{proof}[\raisedtarget{#1}Proof of \Cref{#1}]}
 {\end{proof}}
\newcommand{\raisedtarget}[1]{%
  \raisebox{\fontcharht\font`P}[0pt][0pt]{\hypertarget{#1}{}}%
}
\newcommand{\proofref}[1]{\hyperlink{#1}{proof}}


\usepackage{graphicx,color}
\graphicspath{{./pics/}}
\definecolor{BrickRed}{rgb}{.625,.25,.25}
\providecommand{\red}[1]{\textcolor{BrickRed}{#1}}
\definecolor{markergreen}{rgb}{0.6, 1.0, 0}
\definecolor{darkgreen}{rgb}{0, .5, 0}
\definecolor{darkred}{rgb}{.7,0,0}
\definecolor{darkorange}{rgb}{1,0.3,0}
\definecolor{darkblue}{rgb}{0,29,245}
%\definecolor{orange}{rgb}{239, 133, 54}
%\definecolor{lightblue}{rgb}{59, 188, 175}

\providecommand{\marker}[1]{\fcolorbox{markergreen}{markergreen}{{#1}}}
\providecommand{\natp}[1]{\textcolor{darkred}{#1}}
\providecommand{\mj}[1]{\textcolor{darkred}{#1}}
\providecommand{\francis}[1]{\textcolor{darkgreen}{#1}}

\theoremstyle{plain}
\newtheorem{theorem}{Theorem}%[section]
\newtheorem{proposition}[theorem]{Proposition}
\newtheorem{corollary}[theorem]{Corollary} %%
\newtheorem{lemma}[theorem]{Lemma} %%
\theoremstyle{definition} %%
\newtheorem{definition}{Definition}
\newtheorem{remark}[theorem]{Remark}
\newtheorem{remarks}{Remarks}
\newtheorem{condition}[theorem]{Condition}
\newtheorem{example}[theorem]{Example}
\newtheorem{assumption}{Assumption}
\setlength{\parindent}{0pt}

\usepackage{amsmath}
%%
%% Mathematical Definitions for Interest Rate Theory
%%

%% GENERAL SETTINGS
\unitlength1cm

%% =============================================================================
%% BASIC MATHEMATICAL SETS AND SPACES
%% =============================================================================
%% =============================================================================
%% PROBABILITY MEASURES AND EXPECTATIONS
%% =============================================================================
\newcommand{\E}{{\mathbb{\sf E}}}
\newcommand{\P}{{\mathbb{P}}}
\newcommand{\p}{{\bf P}}
\newcommand{\q}{{\bf Q}}

\providecommand{\R}{{\mathbb{R}}}
\providecommand{\N}{{\mathbb N}}
\def\Z{{\mathbb Z}}
\def\Q{{\mathbb Q}}
\def\I{{\mathbb I}}
\def\M{{\mathbb M}}
\newcommand{\T}{{\mathbb{T}}}
\newcommand{\Fb}{{\mathbb{F}}}
%% Conditional expectations under various measures
\newcommand{\Eq}{{\mathbb{E}}_{{\bf Q}}}
\newcommand{\Eqn}{{\mathbb{E}}_{{\bf Q}_N}}
\newcommand{\Eqm}{{\mathbb{E}}_{{\bf Q}_M}}
\newcommand{\EqT}{{\mathbb{E}}_{{\bf Q}_T}}
\newcommand{\EqTe}{{\mathbb{E}}_{{\bf Q}_{T_1}}}
\newcommand{\EqTz}{{\mathbb{E}}_{{\bf Q}_{T_2}}}
\newcommand{\EqSe}{{\mathbb{E}}_{{\bf Q}_{S^1}}}
\newcommand{\EqSz}{{\mathbb{E}}_{{\bf Q}_{S^2}}}
%% Almost sure convergence notation
\newcommand{\pas}{\text{{\bf P}--a.s.}}
\newcommand{\paa}{\text{{\bf P}--a.a.}}
\newcommand{\qas}{\text{{\bf Q}--a.s.}}

%% Specific probability measures
\newcommand{\qn}{{\bf Q}_N}
\newcommand{\qm}{{\bf Q}_M}
\newcommand{\qT}{{\bf Q}_T}
\newcommand{\qTe}{{\bf Q}_{T_1}}
\newcommand{\qTz}{{\bf Q}_{T_2}}
\newcommand{\qS}{{\bf Q}_S}
\newcommand{\qSe}{{\bf Q}_{S^1}}
\newcommand{\qSz}{{\bf Q}_{S^2}}
\newcommand{\qs}{{\q_{\rm Swap}}}

%% Miscellaneous
\newcommand{\e}{{\bf e}}
%% =============================================================================
%% FILTRATIONS AND SIGMA-ALGEBRAS
%% =============================================================================
\newcommand{\F}{{\cal F}}
\newcommand{\G}{{\cal G}}
\newcommand{\A}{{\cal A}}
\newcommand{\Hc}{{\cal H}}
\renewcommand{\H}{\ensuremath{\mathcal H}}
\def\filtration#1{{\ensuremath\mathcal{#1}}}
\providecommand{\Fsigma}{\ensuremath{\mathcal \F_\infty^\sigma}}
%% =============================================================================
%% DIFFERENTIAL NOTATION
%% =============================================================================
\newcommand{\dP}{{\rm d}{\bf P}}
\newcommand{\du}{{\rm d}u}
\newcommand{\dd}{{\rm d}}
\newcommand{\df}{{\rm \bf DF}}
\providecommand{\dx}{\ensuremath{\frac{\partial}{\partial x}}}
\providecommand{\dt}{\ensuremath{\frac{\partial}{\partial t}}}
\providecommand{\dy}{\ensuremath{\frac{\partial}{\partial y}}}
%% =============================================================================
%% STATISTICAL AND FINANCIAL NOTATION
%% =============================================================================
\providecommand{\Ncdf}{{\rm N}}
\newcommand{\n}{{\rm n}}
\newcommand{\emb}{\bf \em}
\newcommand{\1}{\textbf{1}}
\newcommand{\fx}{{\rm fx}}
\newcommand{\V}{{\rm Var}}
\providecommand{\var}{\ensuremath{\text{Var}}}
\providecommand{\cov}{\ensuremath{\text{Cov}}}
\newcommand{\Om}{{\Omega}}
%% =============================================================================
%% MATHEMATICAL OPERATORS AND FUNCTIONS
%% =============================================================================
\providecommand{\limn}{\ensuremath{\lim_{n\rightarrow\infty}}}
\providecommand{\qv}[2]{\ensuremath{\langle #1,#1\rangle_{#2}}}
\newcommand{\argmax}{\operatornamewithlimits{argmax}}
\newcommand{\argmin}{\operatornamewithlimits{argmin}}
\providecommand{\vec}[1]{\ensuremath{\bm #1}}
\providecommand{\vecb}[1]{\ensuremath{\bm #1}}
\providecommand{\abs}[1]{\ensuremath{\lvert#1\rvert}}
\providecommand{\norm}[1]{\ensuremath{\lVert#1\rVert}}

%% =============================================================================
%% THEOREM ENVIRONMENTS
%% =============================================================================
\ifx\prop\undefined
\newtheorem{prop}{Proposition}[section]
\fi
\newtheorem{theo}[prop]{Theorem}
\newtheorem{lem}[prop]{Lemma}
\newtheorem{cor}[prop]{Corollary}
\newtheorem{defi}[prop]{Definition}

%% =============================================================================
%% LIST FORMATTING
%% =============================================================================
\providecommand{\labelenumi}{{\rm (\roman{enumi})}}
\setlength{\labelsep}{0.3cm}
\setlength{\leftmargin}{10cm}
\setlength{\labelwidth}{5cm}

%% =============================================================================
%% STOCHASTIC PROCESS TERMINOLOGY
%% =============================================================================
\providecommand{\cadlag}{c\`adl\`ag }
\providecommand{\cadlagns}{c\`adl\`ag}
\providecommand{\caglad}{c\`agl\`ad }
\providecommand{\cad}{c\`ad}
\providecommand{\cag}{c\`ag}
\providecommand{\levy}{L\'evy\ }
\providecommand{\levyns}{L\'evy}
\providecommand{\levyito}{L\'evy-It\^o\ }
\providecommand{\levykhinchin}{L\'evy-Khinchin\ }
\providecommand{\ito}{It\^o }
\providecommand{\itos}{It\^o's\, }
%% =============================================================================
%% FUNCTION SPACES
%% =============================================================================
\providecommand{\D}{\ensuremath{D(\R_+,\R)}}
\providecommand{\Dsig}{\ensuremath{D(\R_+, \R_+\setminus\{0\}})}
\providecommand{\Dd}{\ensuremath{D(\R_+,\R^d)}}
\providecommand{\C}{\ensuremath{C(\R_+,\R)}}
\providecommand{\Cd}{\ensuremath{C(\R_+,\R^d)}}
\providecommand{\rpos}{\ensuremath{{[0,\infty)}}}}

\def\Mc{{\mathcal M}}
\def\tp{\tilde{\p}}
%% =============================================================================
%% MEASURE THEORY AND INTEGRATION
%% =============================================================================
\providecommand{\borel}[0]{\ensuremath{\mathcal{B}}}
\providecommand{\intinf}[0]{\ensuremath{\int_{-\infty}^\infty}}
\providecommand{\intpos}[0]{\ensuremath{\int_0^\infty}}
\providecommand{\intneg}[0]{\ensuremath{\int_{-\infty}^0}}
\providecommand{\dynkin}[0]{\ensuremath{\mathcal D}}

%% =============================================================================
%% CONDITIONAL EXPECTATION AND PROCESSES
%% =============================================================================
\providecommand{\ce}[2]{\ensuremath{\E(#1|\filtration{#2})}}
\providecommand{\inv}[1]{\ensuremath{#1^{(-1)}}}
\providecommand{\os}[2]{\ensuremath{#1^{(#2)}}}
\providecommand{\pos}[2]{\ensuremath{h_{#1}(#2)}}
\providecommand{\poslong}[3]{\ensuremath{h_{#1, #2}(#3)}}
\providecommand{\variation}[2]{\ensuremath{\rm V_{#1}(#2)}}

%% =============================================================================
%% UTILITY COMMANDS
%% =============================================================================
\providecommand{\todo}[1]{\footnote{#1}}

%% =============================================================================
%% PROCESS CLASSES AND STOCHASTIC INTEGRATION
%% =============================================================================
%% Classes of stochastic processes
\providecommand{\classfv}{\ensuremath{\mathscr V}}
\providecommand{\classv}{\ensuremath{\mathscr V}}
\providecommand{\classh}{\ensuremath{\mathscr H^2}}
\providecommand{\classhloc}{\ensuremath{\mathscr H^2_{\rm loc}}}
\providecommand{\classm}{\ensuremath{\mathscr M}}
\providecommand{\classmloc}{\ensuremath{\mathscr M_{\rm loc}}}
\providecommand{\classl}{\ensuremath{L^2}}
\providecommand{\classlloc}{\ensuremath{L^2_{\rm loc}}}
\providecommand{\classa}{\ensuremath{\mathscr A}}
\providecommand{\classaloc}{\ensuremath{\mathscr A_{\rm loc}}}
\providecommand{\classalocpos}{\ensuremath{\mathscr A_{\rm loc}^+}}
\providecommand{\classp}{\ensuremath{\mathscr P}}
\providecommand{\classo}{\ensuremath{\mathscr O}}
\providecommand{\classs}{\ensuremath{\mathscr S}}
\providecommand{\classsp}{\ensuremath{\mathscr S_p}}
\providecommand{\classu}{\ensuremath{\mathscr U}}
\providecommand{\nullset}{\ensuremath{\mathscr N}}

%% Stochastic integration
\providecommand{\stint}{\ensuremath{\cdotp}}

%% =============================================================================
%% FINANCE-SPECIFIC NOTATION
%% =============================================================================
%% CPO distribution
\providecommand{\cpo}{\ensuremath{{\rm CPO}}}
\providecommand{\sigd}{\ensuremath{\mathscr D}}

%% Credit spreads
\providecommand{\s}{{\bf s}}

%% State spaces
\providecommand{\sX}{\ensuremath{\mathcal X}}
\providecommand{\sY}{\ensuremath{\mathcal Y}}

\sloppy
\begin{document}
\setlength{\boxlength}{0.95\textwidth} %
\title{\large{\bf\deftitle}} %
\author{{\normalsize\bf\defauthor}}%
\thispagestyle{empty}
\addtocounter{page}{1}
\maketitle
\begin{abstract}
 To be filled. 
\end{abstract}
% \keywords{keywords here} %%
% \jel{jel here} %%
\vspace{.5cm}
\def\contentsname{Contents}
\tableofcontents
%%
\vspace{.5cm}
\section{Ideas}
\subsection{Uncovered interest parity in the absence of bond}
This section is mostly taken/copied from section 6.1 of \cite{gudgeon2020defi} for idea generation. 
Uncovered interest parity (UIP) normally appear in the context of foreign exchange between two countries: domestic and foreign. 
An investor has the choice of whether to hold domestic or foreign assets. 
If the condition of UIP holds, a risk-neutral investor should be indifferent between holding the domestic or foreign assets because the exchange rate is expected to adjust such that returns are equivalent. \\

Example
An investor starting with 1m GBP at $t=0$ could either:
\begin{itemize}
  \item receive an annual interest rate of $i_\text{GBP}=3\%$, resulting in $1.03$m GBP at $t=1$
  \item or, immediately buy $1.23$ USD at an exchange rate $S_{\text{GBP}/\text{USD}}=0.813$, and receive an annual interest rate of $i_\text{USD}=5\%$, resulting in 1.2915m USD at $t=1$. 
  Then, convert the USD with the exchange rate at $t=1$, say $S_{\text{GBP}/\text{USD}}=0.7974$, and get $1.03$m GBP.
\end{itemize}

If UIP holds, despite the higher interest rate of the USD, the investor will be indifferent because the exchange rate between currencies offset the spread between interest rates. 
Mathematically, UIP is stated as 
\begin{align*}
1+R^{(i)} = (1+R^{(j)})\frac{\mathbb{\sf E}S_{t+k}}{S_t},
\end{align*}
where $R^{(i/j)}$ is the interest rate payable on asset $i/j$ from time $t$ to $t+k$, and $S_t$ is the exchange rate ar time $t$. \\

Now the question is both $R^{(i/j)}$ are not known in advance due to the lack of a liquid bond market that investors can secure the future payoff by holding a cryptocurrency. 
However, the good news is that we have the observable historical short rate, $r_t^{(i)}$, and exchange rate $S_t^{(i/j)}$. 
The UIP condition in this case have to be adjusted to incorporate the fact that the investor consider also the uncertainty of the domestic and foreign short rate, i.e.\footnote{The AAVE and Compound interest rate are compounded every second, which is close enough to model the short rate payoff in a continuous compounding scheme. }

\begin{align*}
  \mathbb{\sf E} \left(\exp{\int_0^T r^{(i)}_t \text{d}t}\right) = \mathbb{\sf E}\left(\exp\left(\int_0^T r^{(j)}_t\text{d}t\right)\frac{S_{T}}{S_t}\right).
  \end{align*}

The above UIP condition open quite some questions
\begin{enumerate}
  \item It seems necessary to model the joint dynamics of the foreign short rate and the exchange rate, such that the R.H.S. of the above equation can be evaluated.
  \item Under which measure should we take the expectation of both sides? Any criteria of choosing the measure? No-arbitrage? 
  \item When will the above equation hold? When will not?
  \item If the condition does not hold, are there any arbitrage opportunities? 
  \item What is the dynamics of $r^{(i/j)}$? For further development, we might want a parametrised stochastic model such that we can
   (i) perform measure change, (ii) price bonds, swaps, or any other derivatives easily, (iii) capture the interest rate dynamics nicely.
  \item Another idea: Can we use machine learning method to get a good enough estimate of the L.H.S. and R.H.S. separately, and make a statistical argument over the difference between L.H.S. and R.H.S.?
  \begin{align*}
    \hat{\mathbb{\sf E}} \left(\exp{\int_0^T r^{(i)}_t \text{d}t}\right) = \hat{\mathbb{\sf E}}\left(\exp\left(\int_0^T r^{(j)}_t\text{d}t\right)\frac{S_{t+k}}{S_t}\right) + \varepsilon_{t, k}.
  \end{align*}

\end{enumerate}

\subsection{A crypto interest rate model}
This idea is motivated by a conversation in an interview with a team of quant working for a centralised crypto exchange (CEX).
 The pricing of crypto derivatives requires a "risk-free" rate for each underlying,
 e.g. a "risk-free" rate of holding BTC, another "risk-free" rate of holding ETH. \\
 
I put quotation marks for the term risk-free rate because there is no consensus to what is the risk-free rate in the crypto market. 
 In practice, there are four ways of getting a risk-free rate by viewing the risk-free rate as the opportunity cost of investing the 
 holdings to the derivative market instead of earning a risk-free rate in another market. 
 The four ways of getting a risk-free are referring to calibrating the risk-free rate to three different markets:
 \begin{enumerate}
  \item DeFi lending protocols, e.g. AAVE and Compound,
  \item DeFi staking protocols, e.g. ETH POS, AQRU,
  \item Futures traded on CEX,
  \item Perpetual futures traded on CEX.
 \end{enumerate} 

The first two markets are not exactly risk-free since trading on DeFi protocols expose oneself to platform risk.
 Although DeFi lending protocols enforce over-collateralisation, 
 lenders still expose themselves to gap risk since there is a chance that the collateral value jump through the value of the borrowings and there is no way to force the borrower to pay back.\\ 

The third and forth markets suffer from the platform risk. 
 However, if a trader manage/hedge her exposure on CEX with other instruments traded on the same CEX, 
 then we can argue that the platform risk is somehow offset. 
 The perpetual futures market does not exempt carry trade traders from losses since funding rate can go negative. 
 It turns out, after considering the pros and cons of the markets, 
 the futures market traded on CEX is commonly used by the team of quant to get the risk-free rate to price their derivatives, which is also a common practice in the traditional market.  \\

The key idea here is to have a CEX futures market centric view, i.e. considering the CEX futures market as risk-free,
 and to study the differences of the rate dynamics among the markets. 
 In order to do so, a common interest rate model for crypto is needed. 
 It will be a challenge to have a single interest rate model that fits the rate dynamics in all the markets.
 For example, the DeFi lending/staking rate is always positive, but the funding rate of perpetual futures can go negative.  
 A possible solution is to discount the DeFi lending/staking rates by the corresponding default intensity.

\subsection{A model for perpetual futures}
Suppose the price of the perpetual futures, $F_t$, is reverting to its underlying, $S_t$, with some Brownian noise involved, i.e.
\begin{align*}
\dd F_t = \kappa \left(\delta S_t - F_t \right)\dd t + \epsilon \dd W_t,
\end{align*}
where $\kappa>0$ is the mean reverting speed, $\delta \in \mathbb{R}$ is the spot-futures spread, and $\epsilon>0$ is the volatility.

We derive the solution of the perpetual futures by applying Ito's lemma to $e^{\kappa t}F$:

\begin{align*}
\dd e^{\kappa t} F_t &= \left(\kappa e^{\kappa t}F_t+ \kappa \left(\delta S_t - F_t\right)e^{\kappa t}\right)\dd t + \epsilon e^{\kappa t}\dd W_t\\
                      &= \kappa \delta S_t e^{\kappa t}\dd t + \epsilon e^{\kappa t}\dd W_t.
\end{align*}

The solution is obtained by integrating both sides from time $0$ to $T$,
\begin{align*}
  e^{\kappa T} F_T &= F_0 +  \kappa \delta \int_0^T S_t e^{\kappa t}\dd t + \int_0^T \epsilon e^{\kappa t}\dd W_t.
  \end{align*}

The expected value of $F_T$ under a risk-neutral measure $\mathbb{Q}$ is
\begin{align*}
  e^{\kappa T} \mathbb{\sf E}_\mathbb{Q} \left(F_T\right) &= F_0 +  \kappa \delta \int_0^T \mathbb{\sf E}_\mathbb{Q}\left(S_t\right) e^{\kappa t}\dd t\\
                                                          &= F_0 +  \kappa \delta S_0 \int_0^T  e^{(r+\kappa) t}\dd t\\
                                                          &= F_0 +  \frac{\kappa}{r+\kappa} \delta S_0 \left(e^{(r+\kappa) T} - 1\right)\\
\end{align*}

Suppose $e^{-rt} F_t$ is a martingale under $\mathbb{Q}$ (i.e. $\mathbb{\sf E}_\mathbb{Q} \left(F_T\right)= e^{rT}F_0$ {\color{red} This assumption disregards the funding cost of holding and shorting the perpetual futures}), 
\begin{align*}
e^{(r+\kappa)T}F_0 &= F_0 +  \frac{\kappa}{r+\kappa} \delta S_0 \left(e^{(r+\kappa) T} - 1\right)\\
\left(e^{(r+\kappa)T}-1\right)F_0 &=  \frac{\kappa}{r+\kappa}\delta S_0 \left(e^{(r+\kappa) T} - 1\right)\\
\delta &= \frac{r+\kappa}{\kappa}\frac{F_0}{S_0}
\end{align*}

In fact, we arrive with the same conclusion if we assume $\delta$ is time dependant (but still deterministic), i.e. $\delta(t): t \rightarrow \mathbb{R}^+$.
 To see that, we start with 
 \begin{align*}
  e^{(r+\kappa) T} F_0 &= F_0 +  \kappa  S_0 \int_0^T \delta(t) e^{(r+\kappa) t}\dd t.
\end{align*}

Differentiating w.r.t. $T$ and evaluating at time $T$ yields,
\begin{align*}
  (r+\kappa) e^{(r+\kappa) T} F_0 &= \kappa S_0 \delta(T) e^{(r+\kappa) T}\\
  \delta(T) &= \frac{r+\kappa}{\kappa}\frac{F_0}{S_0}
\end{align*}

Observations
\begin{enumerate}
\item The spread $\delta $ depends on $\kappa$ and $r$ (risk-free rate)
\item If $r=0$, then $\delta  = F_0/S_0$
\item If $\kappa \rightarrow \infty$, then $\delta  = F_0/S_0$.
\item If $\kappa \rightarrow 0^+$, then $\delta  = F_0/S_0$, (Since $\dd F_t = \varepsilon \dd W_t$, the spread $\delta$ no longer has a meaning.)
\end{enumerate}

% check also https://arxiv.org/pdf/2310.11771

\subsection{Connection between the volatility of spot and level of borrowing rate}
This is motivated by the observation that stable coins have higher borrowing rate, and other coins have lower. \\

Say $B^{(i/j)}(T) = \exp\int_0^T r^{(i/j)}_t \dd t$ are the lending accounts levels if one lend out $i/j$ coins, 
 the investor should be indifference in investing to $i$ or $j$ if the following condition holds.
 \begin{align*}
  \mathbb{\sf E}\left(B^{(i)}(T)\right) 
  &= \mathbb{\sf E}\left(B^{(j)}(T)S_T/S_0\right)\\
  &= \mathbb{\sf Cov}\left(B^{(j)}(T),\ S_T/S_0\right) + \mathbb{\sf E}\left(B^{(j)}(T)\right)\mathbb{\sf E}\left(S_T/S_0\right)\\
  &= \mathbb{\sf Corr}\left(B^{(j)}(T),\ S_T/S_0\right) \sigma\left(B^{(j)}(T)\right) \sigma\left(S_T/S_0\right) + \mathbb{\sf E}\left(B^{(j)}(T)\right)\mathbb{\sf E}\left(S_T/S_0\right).
\end{align*}

Observation: If $\sigma\left(S_T/S_0\right)$ went up, $\sigma\left(B^{(j)}(T)\right)$ must go down for the equation to hold (while keeping all other variables fixed).

\subsection{Optimal DeFi lending portfolio}
AAVE and Compound offers a range of cryptos to be lend out to earn interest. 
 Say you have some USDT and would like to invest on a portfolio of loans, what is the optimal portfolio weights?

Let's look at the payoff of a simple portfolio of two loans first.
 Say the investor has $1$ USDT, she converts $1-p$ amount of USDT to, say, ETH, and invest both coins to the corresponding lending pool at time $0$. 
 At time $T$, she converts back the ETH principle and interest back to USDT. 
\begin{align*}
p\exp\left(\int_0^T r^{(\text{USDT})}_t \dd t\right) 1(\tau^{(\text{USDT})} > T) 
+ (1-p)\exp\left(\int_0^T r^{(\text{ETH})}_t \dd t\right)\frac{S^{(\text{ETH})}_T}{S^{(\text{ETH})}_0}1(\tau^{(\text{ETH})} > T)\\
= p I^{(\text{USDT})}(T)1(\tau^{(\text{USDT})} > T)  + (1-p)I^{(\text{ETH})}(T)\frac{S^{(\text{ETH})}_T}{S^{(\text{ETH})}_0}1(\tau^{(\text{ETH})} > T)
 \end{align*}
 Here,  $r^{(\text{coin})}_t$ is the lending rate of the coin at time $t$, $S^{(\text{ETH})}_t$ is the ETH price in USDT at time $t$, and 
 $\tau^{(coin)}$ is the default event time of the coin which cost all the principle and interest of the corresponding coin. 

Assume the investor's utility is in a constant absolute risk aversion (CARA) form with parameter $\gamma$, 
 her utility of investing to the loan portfolio is 
 \begin{align*}
  u(p) = -\gamma^{-1}\exp\left\{ -\gamma\left( p I^{(\text{USDT})}(T)1(\tau^{(\text{USDT})} > T)  + (1-p)I^{(\text{ETH})}(T)\frac{S^{(\text{ETH})}_T}{S^{(\text{ETH})}_0}1(\tau^{(\text{ETH})} > T)\right)\right\}.
 \end{align*}

 We are interested in maximising the investor's expected utility by searching for the optimal $p$, i.e.
 \begin{align*}
  p^* = \argmax_{p \in [0, 1]} \mathbb{\sf E}\left\{u(p)\right\}.
 \end{align*}

Upon defining the dynamics of $S$, $I$s, and $\tau$s, we can solve for $p^*$ by HJB equation, similar to the Merton's portfolio problem. 

\subsection{Modelling the wealth process}
For simplicity (which we must discuss if this makes sense), 
suppose the dynamics of 
$r^{(1)}_t$, $r^{(2)}_t$, and $S_t$ are as follows.
 \begin{align*}
  \dd S_t &= \mu S_t \dd t + \sigma S_t \dd W_t \\ 
  \dd r_t^{(1/2)} &= \kappa^{(1/2)}\left(\theta^{(1/2)}-r_t^{(1/2)}\right)\dd t + \sigma^{(1/2)}\dd W^{(1/2)}_t,
 \end{align*}
where $\mu$ is the drift of the ETH price, $\sigma$ is ETH price's volatility,
$\nu \geq 0$ is the elasticity, 
$W_t$ and $W^{(1/2)}$s are independent brownian motions, 
$\kappa^{(1/2)}$s and $\theta^{(1/2)}$ are the mean reversion rates and long term mean of lending rate of USDT and ETH. 

Let $X_t$ denote the total wealth of the investor at time $t$, and let
\begin{itemize}
\item $p_t$ represent the proportion of wealth invested in the USDT lending pool at time $t$
\item $1-p_t$ represent the proportion of wealth invested in the ETH lending pool at time $t$
 \end{itemize}

The change in wealth due to the accruing of interest in the USDT lending pool is
\begin{align*}
\dd \left[p_t X_t\right] &= p_t X_t r^{(1)}_t \dd t - p_t X_t \dd N^{(1)}_t\\
&= p_t X_t \left(r^{(1)}_t\dd t - \dd N^{(1)}_t\right),
\end{align*}
where $N^{(1)}$ is the counting process for the default of USDT lending pool. 
$\dd N^{(1)}_s = 1$ when there is a default happen at time $s$.\\


The change in wealth due to the accruing of interest in the ETH lending pool is a bit tricky. 
We get that step by step:
\begin{enumerate}
\item Say at time $t-$, the investor has $(1-p_t)X_{t-}$ USDT worth ETH in the lending pool
\item This is equivalent of having $(1-p_t)X_{t-}\frac{1}{S_{t-}}$ ETH in the lending pool
\item Since $r^{(2)}_t$ is a foreign rate (accuring ETH interest), the investor's ETH in the lending pool change due to interest rate is
\begin{align*}
  \dd \left[(1-p_t)X_{t-}\frac{1}{S_{t-}}\right] = (1-p_t)X_{t-}\frac{1}{S_{t-}} \left(r^{(2)}_t\dd t - \dd N^{(2)}_t\right),
\end{align*}
where $N^{(2)}$ is the counting process for the default of ETH lending pool.
\item We are interested in the change of wealth in USDT. By Ito's product rule
\begin{align*}
  \dd \left[(1-p_t)X_{t-}\frac{1}{S_{t-}}S_t\right]
  &= (1-p_t)X_t \left[
  (r^{(2)}_t + \mu) \dd t + \sigma \dd W_t - \dd N_t^{(2)}
  \right]
\end{align*}
\end{enumerate}

Therefore, the overall change in wealth is
\begin{align*}
\dd X_t = p_t X_t \left(r^{(1)}_t \dd t- \dd N^{(1)}_t\right)
+ (1-p_t)X_t\left[
(r^{(2)}_t+\mu)\dd t + \sigma \dd W_t - \dd N^{(2)}_t
\right].
\end{align*}

Grouping terms, we have
\begin{align*}
  \dd X_t &= \left[
  p_t X_t r^{(1)}_t + (1-p_t)X_t (r^{(2)}_t + \mu)
  \right]\dd t \\
  &+(1-p_t)X_t \sigma \dd W_t \\
  &-p_t X_t \dd N^{(1)}\\
  &-(1-p_t) X_t \dd N^{(2)}\\
  &= X_t\left[
  p_t  \left(
    r^{(1)}_t - r^{(2)}_t - \mu
    \right)
    + r^{(2)}+\mu
  \right]\dd t\\
  &+(1-p_t)X_t \sigma \dd W_t \\
  &-p_t X_t \dd N^{(1)}\\
  &-(1-p_t) X_t \dd N^{(2)}
  \end{align*}

Next, we introduce the value function. 
\begin{align*}
V(t, x, r_1, r_2, s) = \sup_{p \in [0,1]}\mathbb{\sf E}_t\left[
U\left(X_T\right) \big| X_0=x, r^{(1)}_0=r_1, r^{(2)}_0=r_2, S_0 = s
\right],
\end{align*}
where $U(\cdot)$ is the utility function. \\

The HJB equation to solve the optimization problem above is
\begin{align*}
0&=\partial_t V + \sup_{p\in[0,1]}
\Big\{
x\left[p \left(r_1 - r_2 - \mu \right)+r_2 + \mu \right]\partial_X V\\
&+ \frac{1}{2} (1-p)^2x^2\sigma^2 \partial_X^2 V \\ 
&+\lambda^{(1)}\left(V(x-px, r_1, r_2, s) - V\right)\\
&+\lambda^{(2)}\left(V(x-(1-p)x, r_1, r_2, s) - V\right)
\Big\}\\
&=\partial_t V + H,
\end{align*}
where $H$ is the Hamiltonian. \\

\subsubsection{Constant absolute risk aversion utility function}
In this section, we focus on the constant absolute risk aversion (CARA) utility 
\begin{align*}
U(x) = -e^{-\gamma x},
\end{align*}
where $\gamma>0$ is a constant that respresents the degree of risk preferenece.\\

The terminal condition of $V$ is
\begin{align*}
  V(T, x, r_1, r_2, s) = U(x) = -e^{-\gamma x}.
\end{align*}


By observing the terminal condition of $V$ and the HJB equation, we postulate the form of $V(t, x, r_1, r_2, s)$, 
\begin{align*}
V(t, x, r_1, r_2, s) = -e^{-\gamma x}f(t, r_1, r_2, s),
\end{align*}
where $f$ is a function depends on time, the lending rates, and the ETH price. \\

The partial derivatives of $V$ are
\begin{align*}
\partial_t V &= -e^{-\gamma x}\partial_t f\\
\partial_x V &= -\gamma V\\
\partial^2_x V &= \gamma^2 V.
\end{align*}

The change in $V$ due to defaults are
\begin{align*}
V(t, x-px, r_1, r_2, s) - V &= V\left(e^{\gamma px}-1\right) \\
V(t, x-(1-p)x, r_1, r_2, s) - V &= V \left(e^{-\gamma x (p-1)}-1\right) \\
\end{align*}


Plugging everything into $H$, we have
\begin{align*}
H &= \sup_{p\in[0,1]}
\Big\{
x\left[p \left(r_1 - r_2 - \mu \right)+r_2 + \mu \right](-\gamma)V\\
&+ \frac{1}{2} (1-p)^2x^2\sigma^2 \gamma^2 V \\ 
&+\lambda^{(1)} V\left(e^{\gamma px}-1\right) \\ 
&+\lambda^{(2)} V\left(e^{-\gamma x (p-1)}-1\right)
\Big\}.
\end{align*}

The first order condition is
\begin{align*}
  \partial_p H &= 
  -x \left(r_1 - r_2 - \mu \right)\gamma V\\
  &- (1-p^*)x^2\sigma^2 \gamma^2 V \\ 
  &+\lambda^{(1)} V\gamma xe^{\gamma p^*x} \\ 
  &+\lambda^{(2)} V\left(-\gamma x e^{-\gamma x (p^*-1)}\right) = 0.
  \end{align*}

Therefore, $p^*$ satisfies the following equation
\begin{align*}
  \lambda^{(1)} e^{\gamma p^*x}
  - \lambda^{(2)} e^{-\gamma x (p^*-1)} = (r_1 - r_2 - \mu) + (1-p^*)x\sigma^2 \gamma.  
\end{align*}


\subsubsection{Constant relative risk aversion utility function}
In this section, we focus on the constant relative risk aversion utility fuction (CRRA)
\begin{align*}
  U(x) = \frac{x^\gamma}{\gamma}, 
\end{align*}
where $\gamma > 0$ is a constant that represents the degree of risk preference. \\

The terminal condition of $V$ is 
\begin{align*}
V(T, x, r_1, r_2, s) = U(x) = \frac{x^\gamma}{\gamma}.
\end{align*}

By observing the terminal condition of $V$ and the HJB equation, we postulate that $V$ has the form
\begin{align*}
V(t, x, r_1, r_2, s) = \frac{x^\gamma}{\gamma}f(t, r_1, r_2, s),
\end{align*}
where $f$ is a function that depends on time, the lendings rates, and the ETH price. \\

The partial derivatives of $V$ are
\begin{align*}
\partial_t V &= \frac{x^\gamma}{\gamma}\partial_t f\\
\partial_x V &= \frac{\gamma}{x}V\\
\partial^2_xV &= \left(\frac{\gamma^2 - \gamma}{x^2}\right)V.
\end{align*}

The change in $V$ due to defaults are
\begin{align*}
V(x-px) - V(x)&=\left((1-p)^\gamma - 1\right) V\\
V(x-(1-p)x) - V(x) &= (p^\gamma-1) V.
\end{align*}

Plugging everything into $H$, we have
\begin{align*}
H = \sup_{x\in[0,1]}
\Big\{
&\left(p(r_1 - r_2 - \mu) + r_2 + \mu\right)\gamma V\\
&+\frac{1}{2}(1-p)^2\sigma^2 (\gamma^2-\gamma)V \\ 
&+ \lambda^{(1)}\left[(1-p)^\gamma - 1 \right]V\\
&+ \lambda^{(2)}(p^\gamma -1)V
\Big\}
\end{align*}

The optimal $p^*$ satisfies the first order condition, i.e.
\begin{align*}
0=r_1-r_2 - \mu - (1-p^*)\sigma^2 (\gamma - 1) - \lambda^{(1)}(1-p^*)^{\gamma-1} + \lambda^{(2)}(p^*)^{\gamma-1}.
\end{align*}





\section{Literatures}
\subsection{Uncovered Interest Parity and its variants}
\cite{cappiello2007uncovered}
\begin{enumerate}
  \item This paper proposes an extension of UIP called the Uncovered Return Parity (URP)
  \item The URP condition is 
  \begin{align*}
    \mathbb{\sf E}\left(R_{t+1}\frac{S_{t+1}}{S_t}m_{t+1}\Big|\mathcal{F}_t\right)=1,
  \end{align*}
  where $R_{t+1}$ is the gross return on a foreign asset denominated in a foreign currency, and $S_{t+1}$ is the spot exchange rate, defined as the number of units of domestic currency exchanged for one unit of foreign currency. 
  \item The R.H.S. (=1) of the above equation stemmed from definition of stochastic discount factor, see Section 3.1 of \cite{back2010asset}. 
  \item Then the authors assume that there exist a foreign risk-free bond (which we do not have that in the cryptomarket) and yield the following
  \begin{align*}
    \mathbb{\sf E}\left(\frac{S_{t+1}}{S_t}m_{t+1}\Big|\mathcal{F}_t\right) = \frac{1}{R_{f,t}}.
  \end{align*}
  \item The remaining paper is about estimation of URP. The authors estimate the SDF via GMM. 
\end{enumerate}

\subsection{Affine term structure models}
\cite{anderson2010affine}
\begin{enumerate}
  \item The paper extends the affine class of term structure models to describe the joint dynamics of exchange rates and interest rates
\end{enumerate}

\subsection{Interest rate derivatives in the crypto space}
{\bf Inter-Protocol Offered Rate (IPOR)} \url{https://docs.ipor.io/}
\begin{enumerate}
  \item The IPOR company offers an interest rate benchmark (weighted average of DeFi interest rate) with the same name that 
  summarizes the lending and borrowing interest rates of crypto loan platforms.
  \item The company offers trading of fixed income derivatives, e.g. interest rate swap. 
  The pricing and transaction are based on the IPOR rate and automated market maker. 
  \item The main derivatives traded on IPOR is cancellable interest rate swaps.
  \item IPOR uses Hull-White jump-diffusion model for rate simulation and Longstaff-Schwartz method for pricing (the cancellable part)
  \item Criticisms
  \begin{enumerate}
    \item Complex product designs: cancellable swaps + Hull-White-jump-model + Longstaff-Schwartz 
    \item Lack of market-involved pricing mechanism: the AMM takes the spread calculated only by the Hull-White model + Longstaff-Schwartz. 
    \item Spread calculation often results in high spread that prohibits transactions. 
    \item Max tenor is too short (90-day longest) 
    \item See \url{https://scapital.medium.com/ipor-a-postmortem-for-the-interest-rate-swap-pioneer-5dc8492c2f7c}
  \end{enumerate}
\end{enumerate}

\subsection{Time-changed and fractional interest rate models}

\cite{shokrollahi2021time}
\begin{enumerate}
  \item Beware that the work is not peer-reviewed. 
  \item They propose a time-changed fractional short rate model 
  \begin{align*}
    \dd r(T_\alpha(t)) = \mu \dd T_\alpha(t) + \sigma \dd B^H(T_\alpha (t)),
  \end{align*}
  where $\mu$ and $\sigma$ are constants, $B^H$ is a fractional Brownian motion with Hurst parameter $H\in[0.5,\ 1)$. 
  $T_\alpha(t)$ is the inverse 
  $\alpha$-stable subordinator with $\alpha \in(0,1)$ defined as 
  \begin{align*}
    T_\alpha(t) = \inf\left\{\tau > 0 : U_\alpha(\tau)>t\right\},
  \end{align*}
  where $\left\{\right\}$ is a $\alpha$-stable Levy process with non-negative increments and Laplace transform
  \begin{align*}
    \mathbb{\sf E}\left(\exp(-u U_\alpha(t))\right)= \exp(-ru^\alpha).
  \end{align*}
  Here, $T_\alpha(t)$ is assumed to be independent of $B^H$. 
  \item The motivation of modelling beyond Markov models is the ongoing financial empricial evidence. 
  \item It is also sensible to argue that the short rates are driven by macroeconomic variables, like 
  domestic gross products, supply and demand rates or volatilities exhibit long range dependence. 
  \item The price of a zero-coupon bond is given by 
  \begin{align*}
    P(r,t,T) = \exp\left[-r f_2(T-t) + f_1(T-t)\right],
  \end{align*}
  where 
  \begin{align*}
f_1(T-t) &= \frac{H \sigma^2}{\Gamma(\alpha)^{2H}}\int^{T-t}_0 (T-v)^{2\alpha H -1}v^2 \dd v\\
f_2(T-t) & = T-t. 
\end{align*}
\end{enumerate}

\section{Backgrounds}
\subsection{Short rate models}
A quote from Section 10.1 of \cite{Shreve2004a} nicely summarise what is short rate traditionally:
\begin{displayquote}
  "The interest rate (sometimes called the short rate) is an {\color{red} idealization} corresponding to the shortest maturity yield or perhaps the overnight rate offered by the government, depending on the particular application."
\end{displayquote}

\begin{remark}In the crypto world, instead of using liquidly traded bonds and fixed-income products to infer the short rate (mainly its dynamics) and price more complex products,
 the short rate itself is directly observable and impacts the growth rate of borrowing and lending accounts,
 i.e. the growth of crypto borrowing and lending accounts is a {\color{red} realization} of the crypto short rate.
 Beware that {\it all} borrowings in a lending pool, disregard of the identity of borrower and starting date of the borrowings (there is no maturity in crypto loans),
 are growing at the {\it{same}} crypto borrowing rate. \end{remark}

The simplest model for fixed income markets begin with a stochastic differential equation for the interest rate
\begin{align*}
  \dd r_t = \beta(t, r_t)\dd t + \gamma(t, r_t)\dd W_t.
\end{align*}

The zero-coupon bond pricing formula is
\begin{align*}
B(t,T) = \E_\mathbb{Q}\left[
e^{-\int_t^T r_s \dd s}\big|\mathcal{F}_t
\right],
\end{align*}
provided that $B(T,T)=1$. 

Since $r_t$ is governed by a SDE, it is a Markov process and we must have
\begin{align*}
B(t,T) = f(t, r_t)
\end{align*}

By Feynman-Kac, we obtain the partial differential equation
\begin{align*}
\partial_t f(t,r) + \beta(t,r)\partial_r f(t,r)+ \frac{1}{2}\gamma^2(t,r)\partial_{rr}f(t,r) = r f(t,r),\  
f(T, r)=1.
\end{align*}

{\bf Hull-White model}\\
In the Hull-White model, the evolution of the interest rate is given by 
\begin{align*}
\dd r_t = \left(a(t)-b(t)r_t\right)\dd t \sigma(t) \dd W_t,
\end{align*}
where $a(t)$, $b(t)$, and $\sigma(t)$ are non-random positive functions of time.
The PDE for the zero-coupon bond price becomes
\begin{align*}
  \partial_t f(t,r) + \left(a(t)-b(t)\right)\partial_r f(t,r)+ \frac{1}{2}\sigma^2(t)\partial_{rr}f(t,r) = r f(t,r),\  
  f(T, r)=1.
  \end{align*}
By ansatz, the solution of the above PDE has a form 
\begin{align*}
f(t,r) = \exp\left(-r C(t,T)- A(t,T)\right).
\end{align*}

Let's work out the partial derivatives:
\begin{align*}
  \partial_t f &= \left(-r \partial_t C(t,T) - \partial_t A(t,T)\right) f\\
  \partial_r f &= -C(t,T)f\\
  \partial_{rr} f &= C^2(t,T)f.
\end{align*}

Substitute into the PDE gives
\begin{align*}
0=\Big[
&\left(
-\partial_t C+ b(t)C-1
\right)r\\
&-\partial_t A - a(t)C+\frac{1}{2}\sigma^2(t)C
\Big]f(t,r).
\end{align*}
 
Since $f$ is nonzero and this equation must hold for all $r$, 
\begin{align*}
  -\partial_t C+ b(t)C-1 &=0\\
  -\partial_t A - a(t)C+\frac{1}{2}\sigma^2(t)C &=0.
\end{align*}

Since $f(r,T)=1$, $C(T,T)=A(T,T)=0$. The solution of the above equations is 
\begin{align*}
  C(t,T) &= \int_t^T \exp\left(-\int_t^s b(v)\dd v\right)\dd s\\
  A(t,T) &= \int_t^T \left(a(s) C(s,T)-\frac{1}{2}\sigma^2(s)C^2(s,T)\right)\dd s.
\end{align*}


\bibliographystyle{abbrvnamed} %
\bibliography{finance} %
\end{document}

%%% Local Variables: 
%%% mode: latex
%%% TeX-master: t
%%% End: 
